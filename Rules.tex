\raggedright
\part{Regeln}
\chapter{Allgemeine Regeln}
\section{Zeiteinheiten}
\begin{itemize}
\item eine Kampfrunde(KR) dauert 6 Sekunden.
\item eine Woche hat 6 Tage(5 Arbeitstage und 1 Feiertag).
\end{itemize}

\section{Proben}
\label{AllgProben}

Für eine Probe werden immer 6 Würfel geworfen und in 5 Kategorien sortiert:
\begin{description}
\item[1] Gilt als -2 Erfolge
\item[2] Gilt als -1 Erfolg
\item[3] Wird ignoriert(+-0)
\item[3..9] Gilt als 1 Erfolg
\item[>9] Gilt als 2 Erfolge
\end{description}
Es werden alle Erfolge zusammengezählt:
\begin{description}
\item[$\geq$6 Erfolge] Probe 'perfekt' erfolgreich
\item[$\geq$1 Erfolge] Probe erfolgreich.
\item[0 Erfolge] Akteur kann wählen:
\begin{description}
\item Probe erfolgreich aber dauert wesentlich länger(*5 min. 30 Minuten)
\item Macht doch nix(Probe quasi annulliert).
\item Probe erfolgreich aber es passieren 'Schusselfehler' (z.b. Verteidigungslücke, hält nur 1 Kampf, ...).
\end{description}
\item[-1..-6 Erfolge] Probe fehlgeschlagen.
\item[$\leq$-7 Erfolge] Epischer Fehlschlag.
\end{description}

Die Art der Würfel werden durch Spezialisierungen und Situations-/Schwierigkeitsboni/-mali beeinflusst: $Spezialisierungspunkte+Boni-Mali=Probenpunkte$ die Probenpunkte beeinflussen die Würfel:
\begin{description}
\item[-6] 6w4
\item[-5 \& Attribut$\geq$1] 5w4 \& 1w6
\item[... \& Attribut$\geq$1] ...
\item[0 \& Attribut$\geq$1] 6w6
\item[... \& Attribut$\geq$2] ...
\item[6 \& Attribut$\geq$2] 6w8
\item[... \& Attribut$\geq$3] ...
\item[12 \& Attribut$\geq$3] 6w10
\item[... \& Attribut$\geq$4] ...
\item[18 \& Attribut$\geq$4] 6w12
\item[... \& Attribut$\geq$5] ...
\item[24 \& Attribut$\geq$5] 6w20
\end{description}

\subsection{Attributszeug V2 TODO}
Probenpunkte=Attribut+Fähigkeit+Spezialisierung.

\begin{description}
\item[Attribut] 0 bis 6
\item[Fähigkeit] bis zu Attribut*2; generelle Fähigkeit(Fingerfertigkeit/Redegewandtheit/...); je 3-5+ pro Attribut
\item[Attribut] bis zu Fähigkeit*2; konkrete Anwendung/Besonderheit(Schlösser knacken/Taschendiebstahl/Verführen/Überzeugen/Überreden/... ); je 3-5(+) pro Fähigkeit
\end{description}

Kosten Fähigkeit=2*Spezialisierung
Kosten Attribut=2*Fähigkeit

\subsection{Lange Proben/Gruppenprobe TODO}
Mehrere Proben um 1 größeres Ziel zu erreichen(z.b. einen Raum durchsuchen)

Jede Probe:
\begin{itemize}
\item Epische Fehlschlag: Gesamtprobe scheitert
\item Erzeugt Erfolge in Höhe von Probenerfolge-(<bisher gemachte Proben>/X)(abgerundet) X ist eine vom Spielleiter festgelegte 'Effektivität'
\item Einzelproben können zusätzliche Effekte generieren(vor allem Kritische Erfolge/Fehlschläge).
\item Sobald die Gesamterfolge den Wert eines epischen Fehlschlags erreicht scheitert die Probe(epischer Fehlschlag).
\item Sobald genügend Erfolge verfügbar sind passieren Sachen(z.b. ab 8 Erfolgen findet die Gruppe das versteckte Artefakt).
\end{itemize}

\subsection{Übliche Boni/Mali}

........

\textit{\textbf{Beispiel:} 
Schlucki der Zwerg hat Konstitution 3 und Zähigkeit 7. Er will in einer Taverne Zwergenschnaps trinken und muss eine einfache Zähigkeitsprobe ablegen um nicht umzukippen. Er würfelt insgesamt mit 3w12 und 1w8 und hat somit garantiert 4 Erfolge, hat also keinerlei Probleme mit dem Schnaps. }


\subsection{Unterstützung/Gruppenprobe}
(TODO)

Bei einer Unterstützung erhält der Unterstützte die Hälfte(aufgerundet) der Spezialisierungspunkte des Unterstützers als Bonus für seine Probe.

\section{Größenkategorien}
Alle Größenwerte werden in Kategorien eingeteilt. (Jede Kategorie ist dabei $\approx1,5$ mal größer als de vorherige.) Diese Kategorien werden bei bedarf auch benutzt um andere Zahlensteigerungen zu beschreiben(z.b. Masse). Objekte und Personen werden der nächstbesten Größenkategorie zugeordnet.
\begin{tabular}{|c|c|c|}
\hline 
Kategorie & Maximalgröße & Bemerkung \\ 
\hline 
0 & 0,2m & Gilt in Rechnungen als $\frac{1}{4}$ \\ 
\hline 
1 & 0,3m & Gilt in Rechnungen als $\frac{1}{3}$\\ 
\hline 
2 & 0,5m &\\ 
\hline 
3 & 0,7m & Gilt in Rechnungen als $\frac{2}{3}$\\ 
\hline 
4 & 1m &\\ 
\hline 
5 & 1,5m& \\ 
\hline 
6 & 2m& \\ 
\hline 
7 & 3m& \\ 
\hline 
8 & 5m &\\ 
\hline 
9 & 7m& \\ 
\hline 
10 & 10m &\\ 
\hline 
+6 & *10& \\ 
\hline 
18 & 200m & Größe eines \textbf{H-Hohn}\\ 
\hline 
\end{tabular} 

\section{Regeneration}
\label{vital}
(TODO)

Es gibt 2 Möglichkeiten Leben wiederherzustellen. Entweder mit Unterstützung durch Magie (oder Medizin), oder indem man einfach wartet. Für jede Stunde ohne große Aktivität(kein Kampf oder schwere Arbeit also z.b. rumsitzen, schlafen, durch die Stadt bummeln, ...) regeneriert man $Konstitution+Ausdauer$ LP.

Vitale Charaktere regenerieren doppelt so viel. Außerdem haben sie auch mehr Leben(12+(6 o. 1d12) pro Konstitutionspunkt).

\section{Erschöpfung}
(TODO)

Charaktere werden durch Zauber, Kämpfe, (körperliche) Arbeit und langem wach bleiben erschöpft. Bei solchen Aktionen können/werden 'Erschöpfungspunkte' generiert.

Erschöpfungspunkte pro Aktion:
\begin{itemize}
\item[1] pro Kampfrunde(zeigt erst nach dem Kampf Wirkung)
\item[1] pro (Zauber)probe(außerhalb von Kämpfen) um diese zu 'beschleunigen'
\item[1] pro 'unbeschleunigtem' Zauber
\item[1(0)] pro rennen Aktion(während eines Kampfs)
\item[2(1)] pro sprinten Aktion(während eines Kampfs)
\item[1] pro 10min Harte Arbeit(z.b. Schmieden)
\item[1] pro 30min mittlere Arbeit
\item[1] pro 1h leichtere Arbeit(z.b. Marschieren)
\end{itemize}

Wenn ein Charakter bestimmte Erschöpfungsschranken überschreitet bekommt er Mali:
\begin{description}
\item[$>(Ausdauer+1)*5$:] -1 auf körperliche Attribute
\item[$>(Ausdauer+1)*10$:] -2 auf körperliche und geistige Attribute
\item[$>(Ausdauer+1)*15$:] -3 auf alle Attribute
\item[$>(Ausdauer+1)*20$:] -4 auf alle Attribute
\item[...:] -... auf alle Attribute
\end{description}
Hinweis: Wenn ein Körperliches Attribut auf 0 fällt sind damit verbunden Aktionen nicht mehr möglich.

Zum abbauen von Erschöpfungspunkten gibt es folgende Möglichkeiten:
\begin{itemize}
\item[1] pro 1h Ruhepause
\item[+1] pro 1h ohne schlechtes Bedingungen(Regen,Wind,...)
\item[+1] pro 1h beim Schlafen
\item[+1] pro 1h beim Schlafen in Sicherheit/mit Wachen
\item[+1] pro 1h beim Schlafen pro 'Luxusstufe'(Schlafsack, gute Matratze, Bett, ...).
\item[+1] pro 1h bei reichhaltiger Verpflegung.
\item[X] durch Aufputschmittel(magisch oder chemisch/pflanzlich)
\end{itemize}

\section{Erfahrungspunkte}
(TODO)

Im verlauf eines Abenteuers werden Erfahrungspunkte(EP) verdient. Mit EP können verschiedene Charakterverbesserungen erlernt werden.
\begin{enumerate}
\item[1+EP] einen Zauber erlernen
\item[3 EP] 1 Spezialisierungspunkt(SpezP)
\item[12 EP] 1 Attributspunkt(AttrP)
\item[21 EP] 1 Spezialspezialisierung(1 freies Spezialisierungsfeld des Charakterblattes befüllen, erlaubt entsprechende Proben, Start mit Spezialisierung 0)
\item[42 EP] 1 Spezialtalent
\end{enumerate}
	
\section{Bewegung}
(TODO)

%TODO Hex feld bewegung!
Es gibt 5 Arten der Bewegung:
\begin{enumerate}
\item Wandern: $(Konstitution + Ausdauer) * 3 * $Größe km pro Tag(~8-10h).
\item Gewaltmarsch: $Wandern * 1,5$, stark erschöpft am Abend.
\item Gehen: $(\frac{Fitness + 5}{3}) * (Größe)$ pro Kampfrunde.
\item Schleichen: $\frac{Gehen}{2}$.
\item Rennen: $Gehen * 2 + \frac{Akrobatik}{2}$ pro Kampfrunde.
\item Sprinten: $Gehen * 3 + Akrobatik$ pro Kampfrunde.
\end{enumerate}
Alle Bewegungsgeschwindigkeiten sind abhängig von der Körpergröße, diese werden aufgerundet auf eins der folgenden Werte: $0,5m; 0,66m; 1m; 1,5m; 2m; 3m; 4,5m; 7m$. Für Geschwindigkeitsproben können diese Größen als Modifikation der Probenerfolge betrachtet werden.

Je nachdem wie sich ein Charakter über was für Gelände bewegen will kann eventuell eine Probe gefordert werden um die Bewegung auszuführen.(beim Schleichen,Rennen\& Sprinten erhöht sich die Schwierigkeit)

Beispielschwierigkeiten für verschiedene Gelände:
\begin{tabular}{rc}
\multicolumn{1}{c}{Gelände} & Minimal notwendige Proben(Basis)\\
\hline
Wald: & 1 Trivial\\
Sandwüste: & 2 Trivial\\
lose Felsen: & 1 Einfach\\
Sumpf: & 1 Mittel\\
Treibsand: & 1 Schwer\\
\end{tabular}

\section{Schleichen}
(TODO)

Um dich unerkannt zu bewegen ist eine Heimlichkeitsprobe erforderlich, die Schwierigkeit dieser Probe wird Maßgeblich vom Gelände beeinflusst(z.b. Hausboden:einfach, quietschende Dielen: mittel, ...).

Ein Gegner hat die Optionen den Schleichenden wahrzunehmen(z.b. Wachposten oder generell Leute die in der nähe sind), dazu wird eine Wahrnehmungsprobe ausgeführt. Je nachdem was die entsprechenden Personen gerade machen wird die Probenschwierigkeit gewählt(Wachen auf Patrouille: einfach, Leute ohne Wachabsichten: mittel, aktiv abgelenkte/betrunkene: schwer)

Sollte der Gegner sich des Schleichenden schon bewusst sein(zumindest, dass er da ist) wird stattdessen eine Suchenprobe durchgeführt(Schwierigkeit -1).

Alternativregel für große Gruppen: pro Person einer Gruppe, vor der man sich verstecken will muss ein Erfolg generiert werden, kritische Erfolge für wachsame/suchende Personen einer Gruppe.

\section{Glücksproben}
(TODO)

Bei Glücksproben wird ein w20 geworfen, besonders glückliche Menschen(Glück>1, oder Glücksbonus) dürfen mehrere w20 werfen und das beste Ergebnis nutzen.

Anders als bei anderen Proben können auch Zwischenwerte(3,5,7,...) zu einem leicht anderen Ergebnis führen, je besser die Probe ausfällt desto besser ist das Ergebnis für den Charakter.
\chapter{Charisma Regeln}
(TODO)

Es gibt verschiedene Möglichkeiten andere Charaktere zu beeinflussen, diese haben jedoch unterschiedliche Nebeneffekte, von denen einige Dauerhafter Natur sind.

\section{Unerkannt}
\label{Unerkannt}
Mit dem Spezialtalent Unerkannt können bist du ein sehr unauffälliger Typ und andere werden Probleme sich an dein genaues Aussehen zu erinnern(halt so'n Durchschnittstyp). Dadurch sind alle gesellschaftlichen Effekte nur temporär(solange sie dich noch wahrnehmen können), dies betrifft sowohl durch Beeinflussung hervorgerufene Effekte(Einschüchtern/Verführen) als auch andere Effekte(Schlechtes Benehmen bis hin zu Diebstahl). Effekte die auf dich wirken bleiben unverändert.

\section{Rhetorik}
\begin{description}
\item[Anwendungen:] Überzeugen/Überreden/Lügen/Verhandeln($\pm 10\%$ pro kritisch(inklusive 0-fach kritisch))/...

Allgemein eben alles was mit Worten zu tun hat.

\item[Gegenprobe:] Menschenkenntnis

\item[Erfolg:] Der Gegner ist überzeugt.

\item[(Kritischer) Fehlschlag:] Der Gegner ist definitiv vom Gegenteil überzeugt.
\end{description}

\section{Einschüchtern}
\begin{description}
\item[Probe:] $Einschüchtern(Charisma) + \frac{Fitness}{2} w20s$

\item[Gegenprobe bei Humanoiden:] $Menschenkenntnis(Charisma) + \frac{Fitness}{2} w20s$

\item[Gegenprobe allgemein:] $Willenskraft(Konstitution) + \frac{Fitness}{2} w20s$

\item[Erfolg:] Der Gegner ist verängstigt und versucht möglichst schnell und dauerhaft von dir wegzukommen.

\item[(Kritischer) Fehlschlag:] Du hast angst vor deinem Gegner und wirst dich ihm nicht mehr widersetzen.
\end{description}

\section{Verführen}
\begin{description}
\item[Beschränkung:] nur gegen anders geschlechtliche(m$\rightarrow$w oder w$\rightarrow$m)

\item[Gegenprobe:] Verführen

\item[Dekowert:]Die Probe wird um X erschwert wenn gilt:$Dekowert<2^X*gegnerischer\ Dekowert$.

Basis Dekowert=(Verführen+1)*2

\item[Homosexualität:] Verführen nur gegen gleichgeschlechtliche, Verführung von Homosexuellen gegen Heterosexuelle oder umgekehrt: -2 Schwierigkeit bei der Verteidigung.

\item[Erfolg:] Der Gegner ist in dich verliebt und wird versuchen dir nah zu bleiben (potentielle Anhänglichkeit).

\item[(Kritischer) Fehlschlag:] Du bist in deinen Gegner verliebt und machst dir (e.v. falsche) Hoffnungen auf eine Beziehung.
\end{description}

\section{Täuschung}
\begin{description}
\item[Anwendungen:] Für optische/Akustische Täuschungen(vorgeben jemand anders zu sein). Beschreibt nicht wie gut man mit Worten täuschen kann!

\item[Probe:] Heimlichkeit mit dem Würfelpool von Charisma(Das Talent Unerkannt gibt einen Schwierigkeitsbonus)

\item[Gegenprobe:] Menschenkenntnis mit dem Würfelpool von Wahrnehmung

\item[Erfolg:] Gegner nimmt dir deine Imitation ab.

\item[(Kritischer) Fehlschlag:] Gegner durchschaut die Täuschung und betrachtet dich als Feind.
\end{description}

\chapter{Kampf Regeln}

In jeder Kampfrunde hat jeder Charakter 3 Aktionen:
\begin{description}
\item[Haupt] Aktionen die den Oberkörper betreffen(Hände/Waffen/...)
\item[Bewegung] Aktionen die den Unterkörper betreffen(Ortsveränderung/...)
\item[Reaktion] Alles was eine Reaktion auf andere beinhaltet(Verteidigung/Aktion auf ein Ereignis verzögern/Kombinationsaktionen mit anderen ausführen) oder ein bestimmtes Verhalten von anderen annehmen/mehrere Ziele betreffen sollen(AOE angriffe)
\end{description}

Jede dieser Aktionen kann unabhängig verwendet werden oder in Kombination. Beispiele:
\begin{description}
\item[Haupt] Angriff
\item[Bewegung] Gehen
\item[Reaktion] Verteidigung(Blocken)
\item[Bewegung+Reaktion] Ausweichrolle
\item[Haupt+Reaktion] Schwertwirbel(AOE)
\item[...] ...
\end{description}

Spezialfähigkeiten können unter bestimmten Bedingungen weitere Aktionen einbringen:
\begin{description}
\item[Reaktiv] 2 Reaktionen wenn maximal leichte Rüstung getragen wird.
\item[Beidhändig] 2 Hauptaktionen wenn 2 einhändige Objekte gehalten werden.
\end{description}

\section{Angriff}
Bei einem Angriff macht der Angreifer eine Probe mit einem Malus der dem gegnerischen Skilllevel entspricht. Bsp. Pfeilangriff auf einen Gegner mit 16 Punkten in Ausweichen => Malus 16.

Bei einer 0 kann der Angreifer wählen ob er sich selbst eine Blöße gibt(Malus von TODO auf seine Verteidigung in der kommenden Runde) oder nicht trifft.

Bei einem Epischen Fehlschlag bekommt der Gegner einen freien Gegenangriff(falls sinnvoll).

\subsection{Bereichsangriffe}
Anders als bei normalen Angriffen würfelt bei Bereichsangriffen der Verteidiger auf seinen Verteidigungsskill mit einem Malus entsprechend dem Skill des Angreifers.

\subsection{Schaden}
TODO Schanden: 
1. Grund*Schadenserfolge/Rüstung-Zäh
(2. Grund+*?-Rüstung(-Zäh?))
Grund=dx

Im falle eines Treffers wird Schaden in Höhe von $Basisschaden*(Rüstungswert+Schadenserfolge-1)$(min. 0) verursacht.
\subsubsection{Waffenschaden}
Normale Waffen haben einen Waffenschaden äquivalent zu ihrer Größenkategorie. Eine 1m große Waffe würde damit z.b. 4 Schaden verursachen. 

Zusätzlich hat jede Waffe auch eine Form:
Schadensänderungsoptionen:
	1. mehr Basisschaden -> steilere Kurve
	2. 'Bonuserfolge' -> nach oben verschobene Kurve(Basierend auf Basisschaden)
	3. Bonusschaden -> konstant nach oben verschobene Kurve, Kein Rüstungsblock möglich
	4. Rüstwert aka Maluserfolge -> Reduktion Schaden 'unabhängig' von Waffe
	5. Zäh aka Malusschaden -> Reduktion/verhinderung schaden vorallem gegen Kleineren Waffen
=> Schwert: Hieb/Stich Schaden gegen ungerüstet
=> Streitkolben: Stumpf Schaden gegen Panzerung
=> Axt: Hieb Schaden gegen Panzerung
=> Hammer: Stumpf Schaden gegen Panzerung
\begin{tabular}{|r|l|}
\hline Form & Eigenschaft\\
\hline Klinge & Schnittschaden; Schaden+1\\
\hline Knüppel & Stumpfer Schaden; Gegnerische Rüstung+3\\
\hline Spitze & Stichschaden; ???\\
\hline ... & ...\\
\hline
\end{tabular}
\subsubsection{Faustkampf}
Faustkampfwaffen gelten als Größenkategorie 1 Waffen(oder eigene Größenkategorie-5 für Größere Wesen) ohne weitere besondere Eigenschaften. Für normale Menschen(Kategorie 6) ergibt sich somit ein Schaden von 1.


TODO: Initiative basiert oder Phasen basiert?
Eine Kampfrunde besteht aus 4 Phasen:
\begin{description}
\item[1. Manöverphase:] Support-/Bewegungsaktionen nutzen.
\item[2. Hauptphase der Spieler:] Angriffe der Spieler auswürfeln.
\item[3. Hauptphase der Gegner:] Angriffe der Gegner auswürfeln.
\item[4. Endphase:] Kampfrunde beenden \& Wirkung der Angriffe auswerten.
\end{description}
\section{Manöverphase}
Für jeden Charakter wird gesammelt/ausgewertet was er mit seiner Supportaktion machen will.
Die Supportaktion der Kampfrunde kann z.b. folgendes sein:
\begin{itemize}
\item Bewegen(gehen).
\item Spezielles Kampfmanöver ausführen.
\item Spezielles Kampfmanöver unterstützen(2. Aktion).
\item Nachladen ausführen(2. Aktion).
\item Zaubern(2. Aktion).
\item (einfache) Interaktionen ausführen(z.b. unverschlossene Tür öffnen).
\item Reden.
\end{itemize}
Falls die Supportaktion für andere Aktionen mitbenutzt wird gibt es entsprechend keine Auswertung in dieser Phase.

Falls der Charakter sich für eine Bewegung entscheidet(auch wenn andere Aktionen mit betroffen sind(rennen/sprinten)):
\begin{description}
\item[Nahkampfdistanz eines Gegners betreten:] Standarddistanz=max(eigene Waffenreichweite, gegnerische Waffenreichweite).
\item[Nahkampfdistanz eines Gegners verlassen:] Verteidigung nur über Ausweichen(nicht Abwehr) in dieser Runde.
\end{description}

\section{Hauptphase}
Die Hauptaktion/Primärfokus der Kampfrunde kann z.b. folgendes sein:
\begin{itemize}
\item Angriff durchführen.
\item Spezielles Kampfmanöver ausführen.
\item Nachladen ausführen(kann weitere Aktionen benutzen).
\item Zaubern(benutzt weitere Aktionen).
\item (Komplexe) Interaktionen ausführen.
\item ...
\end{itemize}
Bei einem Angriff wird generell eine Angriffsprobe(Nahkampf) gegen eine Verteidigungsprobe ausgeführt.

\subsection{Nahkampfdistanzen}
Jede Waffe hat eine andere optimale Nahkampfdistanz, diese leitet sich aus der Größenkategorie der Waffe ab(eine 1,5m Waffe hat eine um 1 höhere optimale Nahkampfdistanz verglichen mit einer 1m Waffe).

Für Verteidigungsproben(Abwehr \& Schildabwehr) gibt es einen Malus in Höhe von $max(0,Waffendistanz-Kampfdistanz)*2$.

Für Angriffsproben gibt es einen Malus in Höhe von $|Kampfdistanz-Waffendistanz|*2$.

Beispiel 1: Kampf von einer 0,33m Waffe(A) gegen eine 0,66m Waffe(B). Die beiden Kämpfer halten eine Kampfdistanz die einer 0,5m Waffe entspricht.
Greift A B an bekommt A er -2 Spezialisierungspunkte auf seinen Angriff während B -2 Spezialisierungspunkte auf seine Verteidigung erhält.
Greift B A an bekommt B er -2 Spezialisierungspunkte auf seinen Angriff während A seine normale Verteidigung nutzen darf.

Wenn ein Schild getragen wird gilt bei Verteidigungsproben immer die Waffendistanz des Schildes statt der der eigenen Waffe.

Um die Nahkampfdistanz(für die nächste Runde) zu ändern gibt es verschiedene Möglichkeiten:
\begin{itemize}
\item Spezielle (bewegungs-)Manöver.
\item Bei einem Angriffswurf kann jeder Erfolg genutzt werden um die Distanz um 1 zu verringern.
\item Bei einem Verteidigungswurf(Ausweichen)kann jeder Erfolg genutzt werden um die Distanz um 1 zu erhöhen.
\end{itemize}

\subsection{Abwehraktion}
Es gibt verschiedene Verteidigungsmöglichkeiten, eine Verteidigung erfordert jedoch immer die Defensivaktion, weshalb jede Verteidigungsprobe immer gegen alle Angriffe der Kampfrunde gilt(bei jedem Angriff gelten aber alle für den Angriff relevanten Würfe):

\begin{tabulary}{\textwidth}{|C|C|C|C|}
\hline 
Aktion & Ausweichen & Abwehr & Schildabwehr\\ 
\hline 
\hline 
Probe & Ausweichen & Abwehr & Ausweichen \textbf{oder} Abwehr \\ 
\hline 
Schwierigkeit gegen Nahkampf & $+ 1$ & $\pm 0$ & $\pm 0$ \\ 
\hline 
Schwierigkeit gegen Fernkampf & $\pm 0$ & $+ 1$ & $\pm 0$ \\ 
\hline 
Schwierigkeit gegen größere Gegner(nur bei 'gezielten Angriffen') & \multicolumn{3}{c|}{$- 1$ pro Größenkategorie Differenz.} \\ 
\hline
Gegen Tech. Nahkampf & \multicolumn{2}{c|}{Todo} & ? \\ 
\hline 
Konteraktion Gegenschlag(nur mit Nahkampfwaffe) & (leichter) Treffer & Gegner entwaffnen & Schildschlag(1 Runde betäubt) \\ 
\hline 
Konteraktion Angriff umlenken(Verfälschung der Angriffsrichtung um Ziel auszuwählen) & Ja($\pm 0^\circ$ Verfälschung) & Ja($\pm 30^\circ$ Verfälschung). & Ja($\pm 90^\circ$ Verfälschung) \\ 
\hline 
Tech. Nahkampf Konter & - & Umlenkung(Tech. trifft den Gegner) & Nahkampf Konter? \\ 
\hline 
Voraussetzung & - & Waffe o. Metallarmschienen & Schild \\ 
\hline 
\end{tabulary}

Bei Konteraktionen gilt: ein normaler kritischer Erfolg generiert einen leichten Treffer auf das Ziel der Konteraktion/löst den beschriebenen Effekt aus, ein 2-fach kritischer Erfolg generiert einen normalen Treffer oder verdoppelt den Effekt, ein X-fach kritischer Erfolg generiert einen X-3-fach kritischen Treffer oder ver-x-facht den Effekt.

\subsubsection{Kleinerer Gegner}
Bei einem Angriff durch einen größerer Angreifer wird die Verteidigungsprobe für Fernkampf um 1 Stufe(4 Punkte) erleichtert pro Größenkategorie unterschied.

\subsubsection{Mehrere Gegner}
Wenn mehrere Angreifer ein Ziel angreifen wird die Verteidigungsprobe um 1 Stufe(4 Punkte) erschwert.
Wenn die Angreifer das Ziel umzingelt(Angriffswinkel $>120^\circ$, mindestens 3 Angreifer) haben wird die Verteidigungsprobe nochmals um 1 Stufe erschwert.

Die Verteidigungsprobe erhält außerdem einen Malus von 2 pro zusätzlichen Angreifer gegen den Verteidigt werden soll.

Man muss sich nicht gegen alle Gegner verteidigen(wodurch z.b. der Malus durch umzingeln vermieden, bzw. die Angriffsboni reduziert werden können), alle gegen die man sich nicht verteidigt können einen allerdings angreifen als ob man sich überhaupt nicht verteidigt(Probe ohne Gegenprobe \ref{ohneGegner}).

Wenn man sich gegen '0' Angreifer(z.b. durch Verteidigungsboni) wird die Schwierigkeit um 1 Stufe reduziert.

\paragraph{Gruppenangriff}
TODO
\paragraph{Gruppenverteidigung}
TODO


\section{Endphase}
Effekte(Schaden, Magie, ...) die in der Kampfrunde erzeugt werden gelten normalerweise erst für die nächste Kampfrunde. Schaden \& Co können/sollten zwar schon während der Kampfrunde berechnet und aufgeschrieben werden die Konsequenzen davon(Tod, Bewusstlosigkeit, ...) gibt es jedoch erst in der Endphase.

Alle Teilnehmer bekommen außerdem 1 Erschöpfungspunkt(zusätzlich zu eventuellen Erschöpfungspunkten durch spezielle Aktionen).

\section{Manöver}
\label{SupportKampfaktionen}
Manöver werden unterteilt in Standardmanöver und Spezialmanöver.

Standardmanöver sind immer/für jeden verfügbar:
\begin{description}
\item[Gehen:] ($Sup$)
\item[Rennen:] ($Atta + Sup$) +1 auf Abwehrschwierigkeit.
\item[Sprinten:] ($Atta + Sup + Def$)

\item[Stürmen:] ($Sup$) Reduziert die Nahkampfdistanz(für die nächste Runde) auf die eigene Waffenreichweite. Alle Distanzmali auf Angriffsproben werden auf die Verteidigungsproben übertragen.
\item[Wegspringen:] ($Sup$) Erhöht die Nahkampfdistanz(für die nächste Runde) auf die eigene Waffenreichweite. Alle Distanzmali auf Angriffsproben werden auf die Verteidigungsproben übertragen.

\item[Normaler Angriff:] ($Atta$) Kritische Erfolge werden genutzt um kritische Treffer zu erzeugen.
\item[Gezielter Angriff:] ($Atta$) Gegenprobe wird um die Rüstungsschwierigkeit verbessert dafür wird der Rüstungswert ignoriert(sonst wie Normaler Angriff).
\item[Schneller angriff:] ($Atta + Sup$) Jeder kritische Erfolg erzeugt 1 normalen Treffer. Jeder Erfolg erzeugt 1 leichten Treffer. Gegnerische Misserfolge werten leichte Treffer zu normalen Treffern auf. Eigene Misserfolge werten Treffer ab oder entfernen 1 leichten Treffer.
\item[Heftiger Angriff:] ($Atta + Sup$) Normale Erfolge werden genutzt um kritische Treffer zu erzeugen. Misserfolge gleichen sich aus und wandeln Kritische Erfolge in normale Erfolge bzw. neutralisieren normale Erfolge. Verteidigungsproben der Runde sind erschwert um 1 Stufe.

\item[Konter:] ($Sup??$) Ausführung einer Konteraktion(siehe Tabelle) bei kritischem Defensiverfolg.

\item[Nahschuss:] ($Atta$) ??
\item[Hastiges Nachladen:] ($Atta + Sup + Def$) Keine anderen Aktionen(inklusive Verteidigungen) während des Nachladens möglich. Probe notwendig, Erfolge reduzieren die Nachladezeit, Misserfolge erhöhen die Nachladezeit.
\item[Zielen:] ($Sup$) Bonus auf den nächsten gezielter Angriff(+2 Spez).

\item[Angriff mit 2 Waffen(Erfordert Spezialtalent Beidhändigkeit):]($Atta + Sup + Def(optional)$) Ermöglicht einen Angriff mit beiden Waffen, entweder auf 2 Gegner oder als '2 Angreifer' gegen einen Gegner. Die Schwierigkeit des Angriffs erhöht sich um 1.
Abwehraktionen werden ausgeführt als gebe es 1 Angreifer weniger.
Schaden ist mit 'normalen' Waffen um 3 statt 1 Kraft reduziert, mit 'großen' Waffen um 6 statt 3.
Große Fernkampfwaffen sind nicht nutzbar.

\item[Schnelle Reaktion(Spezialtalent):] ($Speziell$) Keine Überraschungsangriffe/Schleichangriffe auf dich möglich, Waffe ziehen als Unterstützungsaktion statt Hauptaktion.
\end{description}

Spezialmanöver benötigen spezielles Training(ähnlich wie Zauber) deshalb:
\begin{itemize}
\item Der Spieler beschreibt was er vorhat bzw. was die Aktion seines Charakters bewirken soll und wie sie aussehen soll(z.b. dreifacher Salto rückwärts auf den Kopf des Gegners mit Schwert voraus und ich nenne es den 'Todesdreher').
\item Falls die Aktion und Wirkung sinnvoll sind wird eine Manöverprobe festgelegt(meist eine passende Kampfprobe, z.b. Nahkampf, mittel\& komplexer) um das Manöver richtig auszuführen.
\item Je nach Aktion und Wirkung können weitere Proben notwendig sein(bzw. kann das Ergebnis entsprechend weiter/wiederverwendet werden).
\item Falls die Manöverprobe erfolgreich ist kann der Charakter die Aktion ausführen.
\item Falls die Manöverprobe kritisch fehlschlägt ...
\item Falls die Manöverprobe kritisch erfolgreich ist(2 extra Krits) hat der Charakter das Manöver sogar direkt gelernt(ohne EP-Kosten).
\item Spezialmanöver können für EP gelernt werden(siehe Zauber).
\item Ein einmal erfolgreich durchgeführtes Manöver kann der Charakter einfacher lernen(siehe 'lernen mit Buch' bei Zaubern).
\item Erlernte Manöver benötigen keine Manöverprobe mehr.
\item Proben für das Manöver sind für gelernte Manöver leichter als für ungelernte Manöver.
\end{itemize}
Beispiele für Spezialmanöver(Zu lernen wie Zauber Stufe 0):
\begin{description}
\item[Rundumschlag:] ($Atta + Sup$) Trifft alle die sich innerhalb der eigenen Waffenreichweite befinden.
\item[Finte:] ($Atta$) Normaler oder gezielter Angriff um 1 Stufe erschwert für beide Seiten.
\item[Umhauen:] ($Atta + Sup$) Kein schaden aber ein Erfolg haut den Gegner um(muss erst wieder aufstehen). Jeder kritische Erfolge zwingen den Gegner min. 1 Runde liegenzubleiben.
\item[Entwaffnen:] ($Atta + Sup$) Je nach Waffe gibts eine Erschwernis(+1 für normale Waffen, +3 für kleine Waffen, +2 für zweihändig geführte Waffen). Kritische Erfolge schleudern die Waffe weiter weg.

\item[2 Pfeile gleichzeitig schießen:] TODO
\end{description}

\chapter{Magie Regeln}
\section{Lore}
Ein Zauber ist eine Komplexe Magische Matrix die bestimmte Effekte erzeugt wenn Mana durch sie hindurch geleitet wird.

Um die Komplexität eines Zaubers begreifbar zu machen werden Zauber normalerweise in 'Zauberworte' zerlegt bei denen die Effekte die sie erzeugen bekannt sind und die relativ frei kombiniert werden können um andere Zauber zu erschaffen.

Ein Zauber kann man sich also etwa vorstellen wie ein elektronisches Gerät das irgendwas tut wenn es mit Energie versorgt wird.
Die Zaubermatrix wäre dann quasi die Platine dieses Geräts(Die beim normalen Zaubern als 1-Weg Objekt konstruiert wird).
Die 'Zauberworte' sind quasi vorgefertigte Schaltplanteile die dann kombiniert werden können.

\section{Regeln}
\label{Kampfmagier}
Eine Zauberaktion benötigt normalerweise alle Aktionen.

Für Kampfmagier braucht eine Zauberaktion nur noch die Hauptaktion und 1 beliebige andere Aktion(Bewegung oder Reaktion).

Alle anderen können die Schwierigkeit der Zauberaktion verdoppeln um 1 Aktion weniger für die Zauberaktion zu verwenden.

Durch langes Training mit auf sie gewirkte Zauber haben Kampfmagier eine bessere Kontrolle über ihren Magieschild und können Spezialisierungspunkte darin investieren.

\subsection{Zaubern}
Zauber zu wirken erfordert die Kombination verschiedener 'Zauberwörter' zu einem Gesamtzauber.
Für jedes Wort ist eine Zauberaktion erforderlich mit einer entsprechenden Magieverstand Probe die je nach Wort eine Zauberschwierigkeit hat.

Jedes Wort wird dabei einer bestimmten 'Magieschule' zugeordnet und hat eine Zauberschwierigkeit. Proben werden entsprechend mit der Spezialisierung der 'Magieschule' und einem Malus entsprechend der Zauberschwierigkeit durchgeführt.

Jede einzelne Probe beeinflusst den Gesamtzauber:
\begin{itemize}
\item Jeder epische Fehlschlag lassen den Gesamtzauber (episch) scheitern.
\item Normale Fehlschläge können ausgeglichen werden wenn die anderen Wörter entsprechend gut gewirkt werden.
\item andere Erfolge/Fehlschläge werden aufsummiert.
\item Alle 3 Runden nach 'Erschaffung' benötigt jedes Zauberwort 1 zusätzlichen Erfolg um es zu stabilisieren(außer die Worte werden auf einem 'Zaubermedium' konstruiert).
\item Sobald die summe der Erfolge/Fehlschläge den Wert eines epischen Fehlschlags erreicht scheitert der Gesamtzauber(episch).
\item Sobald die summe der Erfolge/Fehlschläge mehr als 2 runden hintereinander kleiner als 0 ist zerfällt die Zaubermatrix.
\item Sobald alle Proben gewirkt wurden gilt der Zauber als Erfolg wenn sie Summe der Erfolge/Fehlschläge >0(Erfolg) ist.
\item Ein nicht erfolgreicher Zauber kann unvorhergesehene Konsequenzen erzeugen sobald er mit Mana durchflutet wird.
\item Episch fehlgeschlagene Zauber entziehen dem Wirker direkt Mana und erzeugen unvorhergesehene(meist schlechte) Konsequenzen.
\end{itemize}

Eine fertige Zaubermatrix wird schließlich durch Mana aktiviert und damit auch direkt 'verbraucht'(Ausnahme 'Zaubermedien').

TODO: magische Macht Probe?

\subsubsection{Zaubermedien/Gegenstandsverzauberungen}
Zauber können auch auf 'Zaubermedien' geschriebene werden um die Zaubermatrix über längere Zeit stabil zu halten.
'Zaubermedien' sind üblicherweise Magiekristalle 1 Kristall kann nur eine Begrenzte menge an Wörtern aufnehmen und (meist) nur begrenzt oft aktiviert werden. Ein Zauberwort auf einen Magiekristall zu schreiben erfordert spezielles Werkzeug und 1 Minute statt 1 Runde pro Wort.
Außerdem können Magiekristalle nicht neubeschrieben werden(Zauber werden quasi eingraviert). Sollte der Magiekristall noch nicht 'voll' sein kann der Zauber aber noch entsprechend erweitert werden oder gar ein 2. Zauber hinzugefügt werden.
Sobald ein Zauber seine maximale Anzahl Benutzungen verbraucht hat ist dieser Teil des Magiekristall dauerhaft nicht mehr nutzbar.

\subsection{Zaubereffekte}
Für einen Zauber der einen Effekt auf andere Lebewesen ausüben soll wird eine Probe(siehe Kampf) mit der Zaubermacht des Anwenders gegen den Magieschild des Zieles durchgeführt(auch bei Positiven Effekten solange es nicht explizit anders angegeben ist).
Wenn das Ziel die Probe gewinnt hat seine Magieresistenz den Zauber abgeblockt.

Für (direkte) magische Effekte gilt zudem immer der Rüstungswert 'ohne Rüstung' als Multiplikator es seiden es gibt einen expliziten magischen Rüstungswert an der Rüstung.

Indirekte magische Effekte(z.b. Telekinesewurf eines Felsen oder ein Feuerball) gelten als 'normale' Fernkampfangriffe denen entsprechend ausgewichen werden kann/Rüstung wirken. Für diese ist der Magieschild des Ziels irrelevant und sie können auch eine nicht Zaubermacht Probe des Anwenders erfordern.

\subsection{Mehrere Wörter}
Wenn mehr als 1 Wort pro Runde erzeugt werden soll muss eine Zauberprobe mit Schwierigkeit $X*Summe der Wortschwierigkeiten$ ausgeführt werden(mit X=Anzahl gleichzeitig gewirkter Wörter).

\subsection{Fehlschläge TODO}
\begin{TODO}
seltene Änderung des Magiebereichs -> Hälfte im gleichen bereich hälfte speziell/2 Würfel/...
\end{TODO}
\begin{tabulary}{\textwidth}{|c|C|}
\hline 
Fehlschläge (1w20) & Effekt (immer ungezielt wenn nicht näher definiert) \\ 
\hline 
\hline 
1 & Dein Zauber. \\ 
\hline 
2 & Jede Mag. Tech wird aufgeladen \\ 
\hline 
3 & • \\ 
\hline 
4 & zufällige Heilung \\ 
\hline 
5 & • \\ 
\hline 
6 & Zufällige Explosionen \\ 
\hline 
7 & • \\ 
\hline 
8 & Alles wird sauber \\ 
\hline 
9 & • \\ 
\hline 
10 & Drachenillusion(alle) \\ 
\hline 
11 & Gegenstände fliegen durch die Luft. \\ 
\hline 
12 & • \\ 
\hline 
13 & (Elementar)Beschwörung \\ 
\hline 
14 & Feuerball auf dich \\ 
\hline 
15 & zufälliger Schutz \\ 
\hline 
16 & Aktiven Zauber werden aufgelöst. \\ 
\hline 
17 & Licht \\ 
\hline 
18 & • \\ 
\hline 
19 & Du(+zufällige andere) verwandelst dich(nutzlose Erscheinung)\\ 
\hline 
20 & 'Angriff' auf dich. \\ 
\hline 
\end{tabulary} 

\begin{tabulary}{\textwidth}{|c|C|}
\hline 
Ungezielt(1w20) & Effekt \\ 
\hline 
\hline 
1 & Zauber wird auf alle Verbündeten des Ziels gewirkt. \\ 
\hline 
2 & Zufällig triffst du das richtige Ziel \\ 
\hline 
3 & \multirow{3}{*}{Ein Verbündeter des Ziels.} \\ 
4 & \\ 
5 & \\ 
\hline 
6 & \multirow{12}{10cm}{\center Umgebung/Neutral:\linebreak Niedrigere Zahlen haben positivere Nebenwirkungen.\linebreak Höhere Zahlen haben negativere Nebenwirkungen.} \\ 
7 & \\ 
8 & \\ 
9 & \\ 
10 & \\ 
11 & \\ 
12 & \\ 
13 & \\ 
14 & \\ 
15 & \\ 
16 & \\ 
\hline 
17 & \multirow{3}{*}{Ein Gegner des Ziels} \\ 
18 & \\ 
19 & \\ 
\hline 
20 & Zauber wird auf alle Gegner des Ziels gewirkt. \\ 
\hline 
\end{tabulary}

\subsection{Lebensdauer der Magie TODO}
Normalerweise ist ein Zauber eine einmalige Angelegenheit und erzeugt einen bestimmten Effekt, dies ist spontane Magie, bei der die Zaubermatrix für das Wirken des Zaubers 'verbraucht' wird. Die Zaubermatrix destabilisiert sich dabei durch das zuführen Magischer Kraft und der gewünschte Effekt tritt normalerweise auf wenn die Matrix zerbricht und die enthaltene Energie gezielt freigibt.

Einige Zauber haben jedoch eine stabilere Zaubermatrix, diese werden temporäre Zauber genannt, bei diesen tritt der Effekt auf solange die Zaubermatrix intakt und befüllt ist. Solche Zauber benötigen konstante Aufmerksamkeit um die Zaubermatrix über längere Zeit aufrechtzuerhalten. Eine Zuführung Magischer Energie nach der Aktivierung können diese Zauber allerdings nur sehr selten verkraften. Sobald der Zauberer sich nicht mehr um die Zaubermatrix kümmert(sei es durch Unfähigkeit oder weil er sie einfach nicht mehr braucht), zerfällt sie relativ schnell(innerhalb der nächsten Runde).

Eine dritte Kategorie sind sogenannte permanente Zauber, diese sind in vielen Punkten den temporären Zaubern sehr ähnlich, mit dem unterschied, das sich nach einer gewissen Zeit (bestimmten Anzahl an Runden) die Zaubermatrix stabilisiert, sodass sie auch ohne das Zutun von irgendjemandem erhalten bleibt, solche stabilisierten Matrizen lassen sich nur durch Spezialzauber auflösen.
Während eine permanente Magie noch in der Anfangsphase ist gelten jedoch die gleichen Regeln wie für temporäre Magie.

\chapter{Technik Regeln - Anwendung TODO}

Jeder durch Technik verursachte Schaden ist tödlich wenn es bei der Waffe nicht anders definiert ist, anders als bei Magischem und Nahkampfschaden kann man sich bei technischen Angriffen nicht zurückhalten, um den Gegner nicht zu töten....

\section{Gegenstände allgemein}
\subsection{Eigenschaften von Gegenständen}
Jeder Gegenstand kann bestimmte Eigenschaften besitzen:

\begin{itemize}
\item Stufe angegeben als Technikstufe-Qualitätsstufe bestimmt den Schwierigkeitsgrad für Konstruktion/Analyse/Verbesserung des Gegenstands.
\item Spezial listet Spezialfunktionen des Gegenstands auf, diese können reguläre Aktionen(Angriff/Nachladen/...) ersetzen, erweitern oder dauerhaft ändern.
\item Supportaktionen listet Spezialfunktionen des Gegenstands auf, diese benötigen mindestens eine Supportaktion um aktiviert zu werden.
\item Magazin/Verbrauch wenn im Magazin weniger ist als Verbraucht werden müsste kann das Gerät/Spezialfunktion nicht benutzt werden. Wenn nichts angegeben ist sind beide Werte 1.
\item Nachladen Zeit, die investiert werden muss bis das Magazin erneuert wurde einsetzbar ist. Ist für Technische Geräte relevant. Wenn nichts angegeben ist kann das Gerät nicht(normal) Nachgeladen werden.
\end{itemize}

\subsection{Haltbarkeit}
Jeder Gegenstand(Schwert, Tür, Mauer, ...) hat 10 Haltbarkeitspunkte(\textbf{HP}) wenn mit dem Gegenstand eine Aktion ausgeführt wird, für die er nicht geeignet ist(Schwert als Brechstange, Gewehr als Keule, Schaden, ...) wird eine Haltbarkeitsprobe durchgeführt. Eine Haltbarkeitsprobe hat triviale Schwierigkeit und nutzt die Haltbarkeitsspezialisierung des Gegenstands. Standardmäßig hat jeder Gegenstand 2 Haltbarkeitswürfel und 0 Spezialisierungspunkte.

Je nachdem wie unpassend die Aktion war/wie leicht sie zu Beschädigungen führt wird eine Erfolgszahl festgelegt (z.b. 3 für leicht unpassende Aktionen, 15 für mutwillige Zerstörung(Taschenuhr aus 20 m Höhe auf den Boden werfen)). Wenn die Probe scheitert verliert der Gegenstand HP in Höhe der Differenz von benötigten zu erreichten Erfolgen.

Wenn ein Gegenstand die Hälfte seiner HP verloren hat können keine Spezialaktionen mehr ausgeführt werden, außerdem werden wichtige Werte halbiert(halber Schaden bei Waffen). Durch eine Reparatur kann jedoch die volle Funktionsfähigkeit wiederhergestellt werden. Für Schusswaffen gilt jeder weitere Schuss als unpassende Aktion(3-5 je nach Beschädigung).

Wenn ein Gegenstand auf 0 HP sinkt zerbricht er völlig/wird komplett unbrauchbar, eine Reparatur ist dann kaum mehr möglich.

\subsection{Reparatur}
Es gibt 3 Varianten einen Gegenstand zu reparieren. Für die ersten beiden gilt: wenn die Ausführung scheitert muss eine erneute Haltbarkeitsprobe(7) gemacht werden.\begin{enumerate}
\item durch Magie
\item \label{Reperatur} durch eine normale Reparatur: Mechanik + Spezialerfolge(je nach Typ z.b. Chemie für Explosivwaffen) Schwierigkeit je nach Schaden und Technikstufe(1h Basiszeit, passendes Werkzeug) um den Gegenstand zu reparieren in der folgenden Tabelle wird angegeben, wie schwer es ist den Gegenstand um je 1 HP zu reparieren. Für eine vollständige Reparatur eines beschädigten Gegenstands muss eine Probe mit der Erfolgssumme aller Schadenswerte von 1 HP bis X HP geschafft werden.

\begin{description}
\item[1 HP Schaden:] 1 Technikstufe-2 Mechanikerfolge \& 2 Technikstufe-2 Spezialerfolge
\item[2 HP Schaden:] 2 Technikstufe-2 Mechanikerfolge \& 1 Technikstufe-1 Spezialerfolge
\item[3 HP Schaden:] 1 Technikstufe-1 Mechanikerfolge \& 2 Technikstufe-1 Spezialerfolge
\item[4 HP Schaden:] 2 Technikstufe-1 Mechanikerfolge \& 1 Technikstufe Spezialerfolge
\item[5 HP Schaden:] 1 Technikstufe Mechanikerfolge \& 2 Technikstufe Spezialerfolge
\item[6 HP Schaden:] 2 Technikstufe Mechanikerfolge \& 3 Technikstufe Spezialerfolge
\item[7 HP Schaden:] 3 Technikstufe Mechanikerfolge \& 1 Technikstufe+1 Spezialerfolge
\item[8 HP Schaden:] 1 Technikstufe+1 Mechanikerfolge \& 2 Technikstufe+1 Spezialerfolge
\item[9 HP Schaden:] 2 Technikstufe+1 Mechanikerfolge \& 3 Technikstufe+1 Spezialerfolge
\item[10 HP Schaden:] 3 Technikstufe+1 Mechanikerfolge \& 4 Technikstufe+1 Spezialerfolge
\end{description}
\item Einen Handwerker bezahlen(nur möglich bei weniger als 10HP Schaden): TODO!?
\begin{description}
\item[Kosten:] $Schaden*10\%$ des Grundpreises(abzüglich der Materialkosten(bis zu 50\%))
\item[Zeit:] $Schaden*10\%$ der Basisherstellungszeit
\end{description}
\end{enumerate}

\subsection{Wartung}
Nach Spielleiterentscheid können für Gegenstände Wartungskosten anfallen(bis zu 1\% pro Monat), wenn diese nicht (rechtzeitig) bezahlt werden verlieren die entsprechenden Gegenstände 1HP. Die Wartungskosten können u.a. auch Munitionskosten ect. mit abdecken.

\section{Fernkampf}

Siehe Kampfregeln, Schusswaffen als Angriffsprobe. Ein Schuss abzugeben benötigt eine Haupt oder Supportaktion.

Beidhändigkeit: Auch mit Fernkampfwaffen/kombiniert möglich. Bei 2 Fernkampfwaffen darf für jede bis zu 2 mal geschossen werden. Bei Nutzung 'Normaler' Waffen gibt es einen Malus von 1 Würfel aufs Zielen. Große Fernkampfwaffen benötigen mindestens Stärke 3 und geben einen Malus von 3 Würfeln. Das Nachladen muss für jede Waffe getrennt durchgeführt werden.

Eine bestimmte Körperstelle anzuvisieren(so es mit der Waffe überhaupt möglich ist) erhöht die Schwierigkeit des Angriffs um 1, dadurch kann z.b. eine tödliche Wunde in eine dauerhafte Verkrüppelung verwandelt werden(wenn Medizinischer Support vorhanden ist), es kann auch zu Schadensänderungen/Nebeneffekten führen.

Jede Waffe verfügt über eine Magazingröße, diese beschreibt wie oft geschossen werden kann bevor nachgeladen werden muss.

Jede Waffe hat eine Nachtladedauer, die angibt wie viele Aktionen zum Nachladen benötigt werden, pro Nachladerunde muss mindestens die Hauptaktion benutzt werden, es kann optional auch die Supportaktion oder die Defensivaktion mit benutzt werden(um bis zu 3 Nachladeaktionen pro Runde auszuführen).

Jede Waffe hat einen Schaden, dieser ist normalerweise fix(pro Kugel).

Sollte sich das Ziel im Nahkampf befinden besteht die Möglichkeit andere Kämpfer zu treffen, so auch wenn das Ziel durch Personen/Gegenständen gedeckt ist, in beiden Fällen kann der Spielleiter eine Erschwernis(z.b. mindestens X Erfolge um das Ziel zu treffen) festlegen, bei einem 'Fehlschuss' wird dann potentiell die Person/der Gegenstand getroffen.

Jede Waffe hat eine Reichweite für jeden Gegner, der näher dran ist wird die Schwierigkeit für einen Schuss um 1 erhöht. Ist das Ziel weiter als $2^X*Reichweite$ entfernt wird die Schwierigkeit um X+1 erhöht.

\section{Nahkampf}

Spezielle Nahkampfwaffen(z.b. ein Explosivhammer o. Federspeer) haben die Möglichkeit Spezialfunktionen zu aktivieren um den Angriff zu verbessern. Das aktivieren muss allerdings noch vor dem ausführen der Angriffsprobe angekündigt werden(außer mit dem Spezialtalent Schnelle Reaktion, mit dem keine Ankündigung nötig ist). Sollte der der Angriff fehlschlagen schlägt auch der Technische Angriff fehl.

Für den Technischen Angriff wird eine Probe auf sonstige Waffen durchgeführt gegen die Verteidigungsprobe des Gegners. Manche Waffen erfordern eine Mindesterfolge, die für eine erfolgreiche Aktivierung notwendig sind.

Jede Waffe verfügt über eine begrenzte Anzahl von Aktivierungen, und einer Nachtladezeit(üblicherweise nur außerhalb eines Kampfes möglich).

Auch Fallenstellen/-entschärfen wird mit dieser Spezialisierung geregelt, je nachdem wie geschickt der Fallensteller beim aufstellen der Falle vorgegangen ist ist es schwerer sie zu entdecken, sie zu entschärfen oder auf sie zu reagieren(Ausweichen als Def.Aktion wenn man sie bemerkt hat).

\section{Rüstungen}

Das Rüstungsgeschick wird benutzt um den Umgang mit Technologien zu beschreiben, die der Verteidigung dienen(Rüstungen, Schilde, Burgtore, ...).

Eine normale Rüstung hat folgende Werte:

Rüstungsklasse: Wenn ein Treffer erzielt wurde wird die Rüstungsklasse zu den Verteidigungserfolgen hinzugefügt und diese Summe anschließend verwendet um die Schwere des Treffers(leichter/normaler/X-Fach-kritisch) zu bestimmen. Rüstung kann einen Treffer jedoch nicht verhindern also selbst mit RK+Verteidigungserfolge=2*Angriffserfolge würde noch ein leichter Treffer generiert werden.

Minimale Anzahl von Basiserfolgen, welche bei jeder Verteidigungsaktion(auch bei unterlassener Verteidigung) erzeugt wird, sollte ein Probe weniger als die angegebenen Erfolge erzielen werden stattdessen diese Basiserfolge benutzt. 
Diese Basiserfolge können niemals einen kritischen Verteidigungserfolg generieren.

Rüstungsschutz: Schadensreduktion, diese besteht meist aus Relativwerten($\frac{1}{10}$, ...) es können jedoch auch zusätzlich Absolutwerte(als $+X$ mit $X\in\mathbb N_{> 0}$) angegeben werden, diese werden unabhängig vom Grundschaden bei Schaden subtrahiert. Absolutwerte sind meistens bei magisch verbesserter Ausrüstung zu finden.

Primäre Voraussetzungen/Sekundäre Voraussetzung/Tertiäre Voraussetzung: min. Wert in Gegenstände + Tech. Geschick zum tragen/verwenden der Rüstung bzw. um Mali zu reduzieren.

Primäre Mali: Negative Effekte durch Tragen der Rüstung. Dieser Malus ist aktiv solange die 3. Voraussetzung nicht erfüllt ist.

Sekundäre Mali: Negative Effekte durch Tragen der Rüstung. Dieser Malus ist aktiv solange die 2. Voraussetzung nicht erfüllt ist.

\section{Sonstiges}

Technik kann für vieles verwendet werden, alles außer den oben beschriebenen wird aktuell unter sonstiges zusammengefasst(z.b. der Umgang mit Lampen/Werkzeug/...).

Auch der Umgang mit medizinischem Gerät(Verbandskasten, ...) fällt in diese Kategorie. Heilung mit regulärer Medizin dauert länger als eine Magische Heilung wird aber ähnlich eruiert(z.b. Magie verstand Probe fällt weg, Technische Geschick(Sonstiges) Probe ersetzt die Zaubermachtprobe als Zeitprobe(1h, Verbände, mit Misserfolgen)).

\section{Technik Regeln - Erzeugung und Analyse}
Für die Herstellung(/Analyse) sind folgende Sachen nötig:
\begin{itemize}
\item Werkzeug
\item Material(oder ein zu analysierender Gegenstand)
\item Zeit(eine Einheit ist bei Analyse weniger($\sim \frac{1}{2}$h) als bei Herstellung($\sim 8$h))
\end{itemize}

Als nächstes wird die Technikstufe und Komplexität des Gegenstands bestimmt: TODO(abhängig von AP Kosten?)

Anschließend wird die eigentliche Probe durchgeführt, dabei gilt:
\begin{itemize}
\item Erfolg: wird benutzt um Komplexität zu erreichen
\item zusätzliche Erfolge: Reduzieren die Dauer um 1 Basiszeiteinheit oder halbiert die Zeit(je nach dem was weniger bringt)(normalerweise gilt Komplexität * Basiszeiteinheit)
\item Kritischer Erfolg: wie Erfolg + zusätzlicher Erfolg
\item Misserfolg: Erhöht die Dauer um 1 Basiszeiteinheit(siehe zusätzlicher Erfolg)
\end{itemize}

TODO: MAC Guyver/Kosten

Mit dem Spezialtalent Mac Guyver werden:
\begin{itemize}
\item Zeitproben um 50\% reduziert(oder auf Minutenbasis statt Stundenbasis berechnet wenn Probenschwierigkeit+1 in Kauf genommen wird)
\item oder Materialkosten um 50\% reduziert(oder auf nahezu 0 gesetzt wenn Probenschwierigkeit+1 in Kauf genommen wird)
\item oder Werkzeugerfordernisse um 50\% reduziert(oder auf nahezu 0 gesetzt(funktioniert mit irgendwelchem Werkzeug) wenn Probenschwierigkeit+1 in Kauf genommen wird)
\end{itemize}

\subsection{Alt}
Erzeugung und Analyse kann/wird über einen längeren Zeitraum durchgeführt, es werden also unterbrechbare Zeitproben, mit Bestrafung der Misserfolge gemacht.

Normalerweise benötigt jede Probe mindestens 1h Zeit und passendes Werkzeug(z.b. Werkstatt/Labor) als Katalysator.

Zusätzlich zu den pro Ressourcen pro Probe können zusätzlich Gesamtressourcen benötigt werden(ein X-Kg Klumpen Arglardor um eine Arglardorwaffe zu schmieden, ...).

Mit dem Spezialtalent Mac Gyver werden Erfordernisse der einzelnen Zeitproben etwas reduziert(entweder alle etwas(um $\approx25\%$) oder eine stärker($\approx50\%$)). 
Zusätzlich darf durch Erhöhung der Probenschwierigkeit um 1 je eine Voraussetzung stark reduziert werden($\approx90\%$).
Es wird dann z.b. nur noch eine Basisausstattung an Werkzeug(nur noch Hammer\& Zange statt Werkstatt/Schmiede) bzw. Ressourcen benötigt.
Die Gesamtressourcen können sich entweder zusammen mit den Probenressourcen reduzieren oder als Konstante erhalten bleiben, diese Entscheidung obliegt dem Spielleiter.

Die Proben werden immer auf Technikwissen durchgeführt die Spezialisierungen richten sich dabei nach den im technischen Gerät benutzten Grundprinzipien. Magische und elektrische Geräte können nur von entsprechenden Meistertechnikern verstanden werden.
Ein technisches Gerät, dass auf mehreren Grundprinzipien aufbaut benötigt auch entsprechend aufgeteilte Erfolge(z.b. Revolver Kombination aus Mechanik und Chemie benötigt von beidem je 10 Erfolge, also auch mindestens 1 Mechanik- und 1 Chemieprobe).

Die Probenschwierigkeit entspricht der Technikstufe des entsprechenden Gegenstands.
Die Menge der Erfolge für einen Erfolg ist von dem Projekt und der Qualitätsstufe des Projekts abhängig.

Üblicherweise benötigte Erfolge um ein Projekt erfolgreich abzuschließen(z.t. können weniger ausreichen um Zwischenergebnisse zu erhalten bzw. mehr Erfolge das Ergebnis verbessern. Bsp.: nach der Analyse einer Falle werden weitere Erfolge gesammelt mit dem Ergebnis, dass ein weiterer versteckter Auslöser entdeckt wird.):
\begin{itemize}
\item Analyse: $\frac{Qualitätsstufe}{2}$
\item Reparieren: Je nach Beschädigung siehe \ref{Reperatur}
\item Bau: 2x Qualitätsstufe.
\item Verbesserung: (Analyse, falls nicht schon gemacht) + Bau
\end{itemize}

\section{Magietechnik}
Mittles Magietechnik lassen sich Zauber in Gegenständen 'speichern'. Dies kann auf 2 verschiedene Weisen passieren: entweder wird nur eine Zaubermatrix im Gegenstand gespeichert(z.b. Zauberstab), dadurch entfällt beim Anwenden des Zaubers die Verstandprobe, es wird immer die im Gegenstand gespeicherte Matrix verwendet.
Die Machtprobe ist allerdings weiterhin notwendig(siehe Magieregeln) und Machtverbesserungen können normal ausgewählt werden.

Die 2. Option ist zusätzlich einen Magiekristall einzusetzen, der letztlich die Machtprobe übernimmt(falls genug Energie gespeichert ist).

Regeltechnisch wird beim erschaffen von Magietechnik 1 bzw. 2 Magietechnik Proben durchgeführt.

Die 1. Probe bestimmt die Zaubermatrix, e.v. kann eine zusätzliche Verstandprobe(leichter als normalerweise) gefordert sein, falls komplexere Zauber gespeichert werden sollen. Erfolge der Magietechnikprobe können genutzt werden um die Zaubermatrix zu 'verbessern'.

Die optionale 2. Probe bestimmt wie die Zaubermatrix mit Energie versorgt wird. Erfolge der Magietechnikprobe können genutzt werden um 'Machtverbesserungen' des Zaubers zu kaufen. Die menge der benutzten Erfolge beeinflusst auch wieviel Energie die Aktivierung des Gegenstands braucht.

Für die 2. Probe gilt:
\begin{description}
\item[1 benutzter Erfolg:] +4 Kapazitätseinheiten verbrauch
\item[1 benutzter kritischer Erfolg:] +6 Kapazitätseinheiten verbrauch
\item[1 unbenutzter Erfolg:] -2 Kapazitätseinheiten verbrauch
\item[1 unbenutzter kritischer Erfolg:] -3 Kapazitätseinheiten verbrauch
\item[1 Misserfolg:] +2 Kapazitätseinheiten verbrauch
\end{description}

Die Energiekristalle können jederzeit von jedem wieder nachgeladen werden. Dafür wird eine Magische Macht(Angriff) Probe gegen 12 ausgeführt:
\begin{description}
\item[1 Erfolg:] +5 Kapazitätseinheiten wiederhergestellt.
\item[1 Misserfolg:] 1 Kapazitätseinheit wiederhergestellt.
\end{description}

\chapter{Regelanhang: Sonderregeln}
\section{Trolle TODO?}
Trolle sind keine Fleischlichen sondern Mineralische Geschöpfe. Sie werden bis zu 7000 Jahre alt sind sehr selten. Die meisten Trolle bestehen aus Gesteinen, einige wenige bestehen aber auch aus Metallen oder Kristallen, diese sind aber auch deutlich kleiner als ihre gesteinischen Brüder. Trolle essen üblicherweise(zumindest zum teil) was sie sind(Ein Granittroll frisst Granit um Material für sein Wachstum zu erhalten).
\begin{description}
	\item [0-700 Jahre] Entstehungsphase
	\item [700-1500 Jahre] Wachstumsphase(0,5m zu 2m), erste selbstwahrnehmung/Bewegungen
	\item [1500-2500 Jahre] Jugendphase(2m-3m), Welt erkunden, Abenteuer erleben. 
	\item [2500-6000 Jahre] Weisheitsphase(<10m), Rückzug zu anderen Trollen, Philosophieren, ...
	\item [6000-7000 Jahre] Endphase(bis zu 20m), Troll findet seinen 'Endfelsen' und wird langsam eins mit ihm.
	\item [Geschlechtslos]
	\item [Mineralisch:] Körper funktioniert anders als bei anderen. Essen Mineralien/Erze/Steine/... können sich nur sehr schwer in 'Fleischsäcke' hineinversetzen.(gegenseitige Verständnislosigkeit)
	\item[Temperierter Geist:]\ 
	\begin{description}
	\item [$\bm{<\text{-}30^\circ C}$:] Geistige Attribute +1 je 1 vollen Punkten
	\item [$\bm{<\text{-}15^\circ C}$:] Geistige Attribute +1 je 2 vollen Punkten
	\item [$\bm{<\ \ 0^\circ C}$:] Geistige Attribute +1 je 3 vollen Punkten
	\item [$\bm{>\ 10^\circ C}$:] Geistige Attribute -1 je 3 vollen Punkten
	\item [$\bm{>\ 25^\circ C}$:] Geistige Attribute -1 je 2 vollen Punkten
	\item [$\bm{>\ 40^\circ C}$:] Geistige Attribute auf 0
	\end{description}
	\item [Haut] Die Haut eines Trolls ist härter als die eines Fleischsacks, daher ist die bei der Charaktererstellung ausgewählte Rüstung für den Troll seine Haut(je nach Rüstung ist dies dann eben Granit/Sandstein/...).
	\item[Größe:] 2-3m
\end{description}

