\documentclass[a4paper]{report}
%umlaute
\usepackage{lmodern}
\usepackage[english,ngerman]{babel}
\usepackage[utf8]{inputenc}
\usepackage[T1]{fontenc}
%math
\usepackage{amsmath}
\usepackage{amsfonts}
\usepackage{amssymb}
\usepackage{bm}
\usepackage{array}
%tabelle
\usepackage{multirow}
\usepackage{tabularx}
\usepackage{tabulary}
\usepackage{diagbox}
\usepackage{array}
\newcolumntype{P}[1]{>{\centering\arraybackslash}p{#1}}
%Zeug
\usepackage{parskip}
\usepackage{geometry}
\usepackage{pdflscape}
\usepackage{fullpage}
\usepackage{rotating}
\usepackage{booktabs}
\usepackage{wrapfig}
\usepackage{xcolor}
\usepackage{setspace}
\graphicspath{{./Bilder/}}%Bilder
%tolle macros
\usepackage{xkeyval}
\usepackage{xstring}
\usepackage{forloop}
\usepackage{xnewcommand}
\usepackage{xargs}

\usepackage{multicol}%equip multicol
\usepackage{enumitem}%better controll for descriptions/other lists

\usepackage{hyperref, bookmark}
\hypersetup{
    linktoc=all,     %set to all if you want both sections and subsections linked
    hidelinks
}

%hide stuff

\usepackage{comment}
	\newif\ifDevelop
	%\Developtrue
	\Developfalse
	\newif\ifSpecial
	%\Specialtrue
	\Specialfalse
	\newif\ifNice
	%\Nicetrue
	\Nicefalse
	\newif\ifChars
	\Charstrue
	%\Charsfalse
	\ifDevelop
		\includecomment{Images}
		\includecomment{characterSheets}
		\includecomment{Beispiele}
		\includecomment{specialEquip}
		\includecomment{hiddenWorld}
		\includecomment{TODO}
	\else
		\ifChars
			\includecomment{characterSheets}
		\else
			\excludecomment{characterSheets}
		\fi
		\excludecomment{TODO}
		\ifSpecial
			\includecomment{specialEquip}
			\includecomment{hiddenWorld}
		\else
			\excludecomment{specialEquip}
			\excludecomment{hiddenWorld}
		\fi
		\includecomment{Beispiele}
		\ifNice
			\includecomment{Images}
		\else
			\excludecomment{Images}
		\fi
	\fi
	\newcommand{\hidden}[1]{}
	\begin{hiddenWorld}
		\renewcommand{\hidden}[1]{#1}
	\end{hiddenWorld}
	\newcommand{\Image}[1]{}
	\begin{Images}
		\renewcommand{\Image}[1]{#1}
	\end{Images}
%set chapter spacing
\usepackage{titlesec}
\titleformat{\chapter}[display]   
{\normalfont\huge\bfseries}{\chaptertitlename\ \thechapter}{20pt}{\Huge}   
\titlespacing*{\chapter}{0pt}{-40pt}{10pt}

%define gnewcommand as a global newcommand
\usepackage{etoolbox}
\makeatletter
\def\gnewcommand{\g@star@or@long\gnew@command}
\def\grenewcommand{\g@star@or@long\grenew@command}
\def\g@star@or@long#1{% 
  \@ifstar{\let\l@ngrel@x\global#1}{\def\l@ngrel@x{\long\global}#1}}
\def\gnew@command#1{\@testopt{\@gnewcommand#1}0}
\def\@gnewcommand#1[#2]{%
  \kernel@ifnextchar [{\@gxargdef#1[#2]}%
                {\@argdef#1[#2]}}
\let\@gxargdef\@xargdef
\patchcmd{\@gxargdef}{\def}{\gdef}{}{}
\let\grenew@command\renew@command
\patchcmd{\grenew@command}{\new@command}{\gnew@command}{}{}
\makeatother

\begin{document}

\makeatletter
\define@cmdkeys{CharacterSheet}[Char@]{LPMax,MPMax}[\ \ \ \ \ ]
\define@cmdkeys{CharacterSheet}[Char@]{MPDesc}[]

\define@cmdkeys{CharacterSheet}[CharAttr@]{St,Ge,Ko,Wa,Ch,AW,TW,TG,MV,MM}[\ \ \ \ \ \ ]

\define@cmdkeys{CharacterSheet}[CharSpez@]{Leb,Mag,
										StKr,StAb,StAk,
										GeNa,GeAu,GeHe,
										KoZa,KoAu,KoWi,
										WaSu,WaWa,WaSc,
										ChRh,ChEi,ChVe,ChMe,
										AWMo,AWGe,AWAk,AWUn,AWFr,
										TWMe,TWCh,TWDr,TWMa,TWEl,
										TGSc,TGAn,TGRu,TGSo,
										MVAn,MVIl,MVVS,MVVW,MVEl,MVBe,
										MMAn,MMIl,MMVS,MMVW,MMEl,MMBe, MMSh}
										[\ \ \ \ ]

\define@cmdkeys{CharacterSheet}[Char@]{Name, Rasse, Alter, Gewicht, Size,
				Eigenschaften,
				Beschreibung,
				Talente,
				Gold, Silber, Kupfer,
				InvGrAnz, InvGr, InvNorAnz, InvNor, InvKleinAnz, InvKlein, InvSonst,
				Zauber,
				Rus, RusRS, RusRK,RusMali,RusVor}[\ \ \ \ \ \ ]

\define@cmdkeys{CharacterSheet}[Char@]{Titel}[Charakter Blatt]


\newcommand{\FillCell}[1]{
	\setstretch{0}\begin{normalsize}#1 \end{normalsize}
}

\newcommand{\CharacterSheetUser}{
\setkeys{CharacterSheet}{LPMax,MPMax,MPDesc}
\setkeys{CharacterSheet}{St,Ge,Ko,Wa,Ch,AW,TW,TG,MV,MM}
\setkeys{CharacterSheet}{Leb,Mag,
							StKr,StAb,StAk,
							GeNa,GeAu,GeHe,
							KoZa,KoAu,KoWi,
							WaSu,WaWa,WaSc,
							ChRh,ChEi,ChVe,ChMe,
							AWMo,AWGe,AWAk,AWUn,AWFr,
							TWMe,TWCh,TWDr,TWMa,TWEl,
							TGSc,TGAn,TGRu,TGSo,
							MVAn,MVIl,MVVS,MVVW,MVEl,MVBe,
							MMAn,MMIl,MMVS,MMVW,MMEl,MMBe, MMSh}
\setkeys{CharacterSheet}{Titel, Name, Rasse, Alter, Gewicht, Size,
				Eigenschaften,
				Beschreibung,
				Talente,
				Gold, Silber, Kupfer,
				InvGrAnz, InvGr, InvNorAnz, InvNor, InvKleinAnz, InvKlein, InvSonst,
				Zauber,
				Rus, RusRS, RusRK,RusMali,RusVor}
}

\newcommand{\txtfld}[3][1cm]{
	\TextField[borderstyle=I,value=#3,borderwidth=0,width=#1,height=0.8\baselineskip]{#2:}
}
\newcommand{\txtfldSilent}[3][1cm]{
	\TextField[borderstyle=I,value=#3,borderwidth=0,width=#1,name=#2,height=0.8\baselineskip]{}
}
\newcommand{\txtfldBig}[4][1cm]{	\TextField[borderstyle=I,value=#3,borderwidth=0,multiline=true,width=#1,name=#2,height=(0.8+2-1)*\baselineskip]{}
}

\newcommand{\valueTab}[5]{	\multirow{#1}{3cm}{\begin{large}\txtfld{#2}{}/\txtfldSilent[0.5cm]{#2Max}{#5}\end{large}} & \txtfld{#3}{#4}\\}
\newcommand{\charTab}[5]{	\multirow{#1}{3cm}{\txtfld{#2}{#4}} & \txtfld{#3}{#5}\\}
\newcommand{\charTabExtraLine}[2]{\cline{3-3}&	&	\txtfld{#1}{#2}\\}
\newcommand{\charTabExtraFreeLine}[1]{\cline{3-3}&	&	\txtfldSilent[2.8cm]{Spez#1}{}:\txtfldSilent{Spez#1Wert}{}\\}
\newcommand{\rot}[1]{\begin{turn}{90}#1\end{turn}}

\newcommand{\CharacterSheet}[1]{
\setkeys{CharacterSheet}{#1}
\newpage
\newgeometry{left=0mm,right=0mm,top=1.5mm,bottom=1.5mm}
\begin{samepage}\begin{landscape}
\thispagestyle{empty} %% Remove header and footer.
\begin{minipage}[t]{0.29\paperheight}
\begin{large}\vspace{-1.5mm}
\setlength\tabcolsep{2pt}\begin{tabular}[t]{|c|l|l|}
\hline 
&	\valueTab{1}{LP}{Lebenskraft}{\CharSpez@Leb}{\Char@LPMax}
	\cline{2-3}
&	\valueTab{1}{MP}{Magiekraft}{\CharSpez@Mag}{\Char@MPMax}
\hline 
\multirow{12}{*}{\begin{turn}{90}\large \textbf{\underline{Körperlich}}\end{turn}}&
	\charTab{3}{Fitness}{Kraft}{\CharAttr@St}{\CharSpez@StKr}
	\charTabExtraLine{Abwehr}{\CharSpez@StAb}
	\charTabExtraLine{Akrobatik}{\CharSpez@StAk}
	\cline{2-3}
&	\charTab{3}{Geschick}{Nahkampf}{\CharAttr@Ge}{\CharSpez@GeNa}
	\charTabExtraLine{Ausweichen}{\CharSpez@GeAu}
	\charTabExtraLine{Heimlichkeit}{\CharSpez@GeHe}
	\cline{2-3}
&	\charTab{3}{Konstitution}{Zähigkeit}{\CharAttr@Ko}{\CharSpez@KoZa}
	\charTabExtraLine{Ausdauer}{\CharSpez@KoAu}
	\charTabExtraLine{Willenskraft}{\CharSpez@KoWi}
	\cline{2-3}
&	\charTab{3}{Wahrnehmung}{Suchen}{\CharAttr@Wa}{\CharSpez@WaSu}
	\charTabExtraLine{Wachsamkeit}{\CharSpez@WaWa}
	\charTabExtraLine{Schätzen}{\CharSpez@WaSc}
\hline 
\multirow{14}{*}{\begin{turn}{90}\large \textbf{\underline{Geistig}}\end{turn}}
&	\charTab{5}{Allgemeines Wissen}{Lebewesen}{\CharAttr@AW}{\CharSpez@AWMo}
	\charTabExtraLine{Geschichte}{\CharSpez@AWGe}
	\charTabExtraLine{Aktuelles}{\CharSpez@AWAk}
	\charTabExtraLine{Untergrund}{\CharSpez@AWUn}
	\charTabExtraLine{Sprachen}{\CharSpez@AWFr}
	\cline{2-3}
&	\charTab{4}{Technisches Wissen}{Mechanik}{\CharAttr@TW}{\CharSpez@TWMe}
	\charTabExtraLine{Chemie}{\CharSpez@TWCh}
	\charTabExtraLine{Druck}{\CharSpez@TWDr}
	\charTabExtraFreeLine{4}
	\cline{2-3}
&	\charTab{5}{Magischer Verstand}{Angriff}{\CharAttr@MV}{\CharSpez@MVAn}
	\charTabExtraLine{Illusion}{\CharSpez@MVIl}
	\charTabExtraLine{Verstärkung}{\CharSpez@MVVS}
	\charTabExtraLine{Verwandlung}{\CharSpez@MVVW}
	\charTabExtraFreeLine{3}
\hline 
\multirow{14}{*}{\begin{turn}{90}\large \textbf{\underline{Seelisch}}\end{turn}}
&	\charTab{4}{Technisches Geschick}{Schusswaffen}{\CharAttr@TG}{\CharSpez@TGSc}
	\charTabExtraLine{andere Waffen}{\CharSpez@TGAn}
	\charTabExtraLine{Gegenstände}{\CharSpez@TGRu}
	\charTabExtraFreeLine{2}
	\cline{2-3}
&	\charTab{4}{Magische Macht}{Angriff}{\CharAttr@MM}{\CharSpez@MMAn}
	\charTabExtraLine{Illusion}{\CharSpez@MMIl}
	\charTabExtraFreeLine{1}
	\charTabExtraLine{\textcolor{gray}{Schild}}{\CharSpez@MMSh}
	\cline{2-3}
&	\charTab{4}{Charisma}{Rhetorik}{\CharAttr@Ch}{\CharSpez@ChRh}
	\charTabExtraLine{Einschüchtern}{\CharSpez@ChEi}
	\charTabExtraLine{Verführen}{\CharSpez@ChVe}
	\charTabExtraLine{Menschenkenntnis}{\CharSpez@ChMe}
\hline 
&\txtfldSilent{Attr1}{}&\txtfldSilent[2.8cm]{FreeSpez2}{}:\txtfldSilent{FreeSpez2-1}{}\\
\hline 
&\txtfldSilent{Attr2}{}&\txtfldSilent[2.8cm]{FreeSpez1}{}:\txtfldSilent{FreeSpez1-1}{}\\
\hline 
\end{tabular} 
\end{large}\end{minipage}\nolinebreak
\begin{minipage}[t]{0.7\paperheight}
\begin{center}
\vspace{0mm}
{\LARGE \underline {\Char@Titel}}
\begin{Large}\begin{minipage}[t]{0.34\textwidth}\begin{normalsize}
\setlength\tabcolsep{2pt}\vspace{-0,8cm}\end{normalsize}
\begin{tabulary}{\textwidth}{rp{1cm}L}
Glück:&1&\\
EP:&\txtfldSilent{EP}{}&\\
Erschöpfung:&\txtfldSilent{Erschöpfung}{}&\\
\end{tabulary}\end{minipage}\begin{minipage}[t]{0.66\textwidth}
\begin{tabularx}{\textwidth}[t]{rXrp{0.5cm}rp{2cm}}
Name: &\txtfldSilent[5cm]{Name}{\Char@Name} & Alter:& \txtfldSilent[0.5cm]{Alter}{\Char@Alter} &Größe:&\txtfldSilent[2cm]{Size}{\Char@Size} \\
Rasse: &\multicolumn[6cm]{3}{c}{\FillCell{\txtfldSilent{Rasse}{\Char@Rasse}}} &Gewicht:& \txtfldSilent[2cm]{Gewicht}{\Char@Gewicht} \\
\end{tabularx}

\begin{tabularx}{\textwidth}[t]{rX}
\hline
\multirow{2}{3.5cm}{\raggedleft besondere\newline Eigenschaften:} &\multirow{2}{\linewidth}{\FillCell{\txtfldSilent[10cm]{Eigenschaften}{\Char@Eigenschaften}}}  \\
 \\
\hline
Beschreibung:&\multirow{4}{\linewidth}{\FillCell{\Char@Beschreibung}}\\
 \\
 \\
 \\
\hline
Spezialtalente: &\multirow{2}{\linewidth}{\FillCell{\Char@Talente}}\\
\\
\end{tabularx}
\end{minipage}
\begin{tabularx}{\textwidth}[t]{rrXrXrXrX}
\hline 
\textbf{Geldkristalle}&\emph{Mithril}:&\txtfldSilent{Geld1}{}&\emph{Gold}: &\txtfldSilent{Geld2}{}&\emph{Silber}:&\txtfldSilent{Geld3}{}&\emph{Kupfer}:&\txtfldSilent{Geld4}{}\\
\hline
\textbf{Münzen}&Gold: &\txtfldSilent{Geld5}{\Char@Gold} &Silber: &\txtfldSilent{Geld6}{\Char@Silber} &Kupfer: & \txtfldSilent{Geld7}{\Char@Kupfer} & Zinn: & \txtfldSilent{Geld8}{}\\
\hline 
\end{tabularx}

\begin{tabularx}{\textwidth}[t]{rX||p{5,5cm}}
\multicolumn{2}{c||}{\textbf{Inventar:}}&\multicolumn{1}{c}{\textbf{Zauber:}}\\
\hline 
Groß(\Char@InvGrAnz ): &\multirow{4}{\linewidth}{\FillCell{\Char@InvGr}}&\multicolumn{1}{c}{\multirow{17}{*}{\FillCell{\Char@Zauber}}}\\
&&\\
&&\\
&&\\
\cline{1-2}
Normal(\Char@InvNorAnz ): &\multirow{4}{\linewidth}{\FillCell{\Char@InvNor}}\\
&&\\
&&\\
&&\\
\cline{1-2}
Klein(\Char@InvKleinAnz ): &\multirow{4}{\linewidth}{\FillCell{\Char@InvKlein}}\\
&&\\
&&\\
&&\\
\cline{1-2}
Sonstiges: &\multirow{4}{\linewidth}{\FillCell{\Char@InvSonst}}\\
&&\\
&&\\
&&\\
\cline{1-2}

\multicolumn{2}{c||}{\begin{tabular}{rp{5cm}rp{1cm}rp{1cm}}
Rüstung: &\Char@Rus& RS: &\Char@RusRS & RK: &\Char@RusRK\\
Mali: &\begin{minipage}{5cm}\FillCell{\Char@RusMali}\end{minipage} &\multicolumn{3}{r}{Voraussetzung:}&\Char@RusVor \\
\end{tabular}} \\

%\hline 
\end{tabularx}
\end{Large}

\end{center}
\end{minipage} 

\end{landscape}\end{samepage}
\restoregeometry
}
\makeatother

\makeatletter
\define@cmdkeys{EquipmentItem}[Equip@]{InvGrAnz, InvNorAnz, InvKleinAnz, Rus, RusRS, RusRK, RusMali, RusVor, InvGrAnzMax, InvNorAnzMax, InvKleinAnzMax,
 Kosten, Stufe, Size, Schaden, Name, Spez, Supp, Beschreibung, Nachladen, Reichweite, Mag,  Verbrauch, Preis, Bild,
 MagVerstand,MagMacht, MagDauer}[]

\newcommand{\defaultEquip}{\setkeys{EquipmentItem}{InvGrAnz, InvNorAnz, InvKleinAnz, Rus, RusRS, RusRK, RusMali, RusVor, InvGrAnzMax, InvNorAnzMax, InvKleinAnzMax,
 Kosten, Stufe, Size, Schaden, Name, Spez, Supp, Beschreibung, Nachladen, Reichweite, Mag,  Verbrauch, Preis, Bild,
 MagVerstand,MagMacht, MagDauer}}

\newcommand{\addAndPrintItem}[2]{	
    \expandafter\gnewcommand\csname DB#1\endcsname{
    		\setkeys+{EquipmentItem,CharacterSheet}{#2}}
	\PrintItemFromDB{#1}
}

\newcommand{\addAndPrintSpell}[2]{
    \expandafter\gnewcommand\csname DB#1\endcsname{
    		\setkeys+{EquipmentItem,CharacterSheet}{#2}}
	\PrintItemFromDB{#1}
}

\newcounter{ct}
\newcounter{addOne}
\def\set#1{\setkeys+{EquipmentItem,CharacterSheet}{#1}}
\newcommand{\call}[1]{%
        \csname DB#1\endcsname
}
\newcommand{\NameTo}[2][]{
	\defaultEquip
	\expandafter\set\expandafter{Name=#2}
	\call{#2}
	\expandafter\set\expandafter{#1}
}

\newcommand{\PPrintItemItemsDesc}[3][]{
\IfEq{}{#2}{}{%
	\item[#3:]\begin{description}[nosep]%
	\StrRight{#2}{200000}[\helpStr]%
	\StrCount{\helpStr}{;}[\countOfLines]%
	\setcounter{addOne}{\countOfLines}%
	\addtocounter{addOne}{1}%
	\forloop{ct}{0}{\value{ct} < \value{addOne}}%
	{%
		\StrCount{\helpStr}{;}[\countNewlines]%
		\IfEq{\countNewlines}{0}%
			{\StrRight{\helpStr}{20000}[\currentLine]}%
			{\StrBefore{\helpStr}{;}[\currentLine]}%
		\StrBefore{\currentLine}{:}[\strBeforeDoubleDot]%
		\IfEq{\strBeforeDoubleDot}{}%
			{\item[]}%
			{\item[\emph{\strBeforeDoubleDot}]\StrBehind{\currentLine}{:}[\currentLine]}%
		\currentLine
		\StrBehind{\helpStr}{;}[\helpStr]%
		\IfEq{}{#1}{}{$-$#1}%Optional Maxima
	}%
	\end{description}%
}%
}
\newcommand{\PrintItemFromDB}[2][]{
\NameTo[#1]{#2}
\saveexpandmode
\expandarg
	\vbox{
	\subsection{\Equip@Name}
	\begin{description}[itemsep=3pt, parsep=0pt]
		\PPrintItemItemsDesc{\Equip@Beschreibung}{Beschreibung}
		\PPrintItemItemsDesc{\Equip@Schaden}{Schaden}
		\PPrintItemItemsDesc{\Equip@Reichweite}{Reichweite}
		\PPrintItemItemsDesc{\Equip@Mag}{Magazin}
		\PPrintItemItemsDesc{\Equip@Verbrauch}{Verbrauch pro Anwendung}
		\PPrintItemItemsDesc{\Equip@Nachladen}{Nachladen}
		\PPrintItemItemsDesc{\Equip@RusRK}{Passive Rüstungsklasse}
		\PPrintItemItemsDesc{\Equip@RusRS}{Rüstungsschutz}
		\PPrintItemItemsDesc{\Equip@RusMali}{Mali}
		\PPrintItemItemsDesc{\Equip@RusVor}{Voraussetzungen}
		\PPrintItemItemsDesc{\Equip@Stufe}{Stufe}
		\PPrintItemItemsDesc{\Equip@Preis}{Listenpreis}
		\PPrintItemItemsDesc{\Equip@Kosten}{Kosten bei Erschaffung}
		\PPrintItemItemsDesc{\Equip@Size}{Größe}
		\PPrintItemItemsDesc[\Equip@InvKleinAnzMax]{\Equip@InvKleinAnz}{Kleine Gegenstände}
		\PPrintItemItemsDesc[\Equip@InvNorAnzMax]{\Equip@InvNorAnz}{Normale Gegenstände}
		\PPrintItemItemsDesc[\Equip@InvGrAnzMax]{\Equip@InvGrAnz}{Große Gegenstände}
		\PPrintItemItemsDesc{\Equip@MagDauer}{Lebensdauer des Zaubers}
		\PPrintItemItemsDesc{\Equip@MagVerstand}{Verbesserungen Zauberweben}
		\PPrintItemItemsDesc{\Equip@MagMacht}{Verbesserungen Zaubermacht}
		\PPrintItemItemsDesc{\Equip@Spez}{Spezial}
		\PPrintItemItemsDesc{\Equip@Supp}{Supportaktionen}
		\Image{\IfEq{}{\Equip@Bild}{}{\item \includegraphics[keepaspectratio, width=\linewidth, height=3cm]{\Equip@Bild}}}
	\end{description}
}
\restoreexpandmode
}

\def\CharInvGr{}
\def\CharInvNr{}
\def\CharInvKl{}
\def\CharInvSo{}
\newcommand{\GenerateCharacterRus}[2][]{
	\NameTo[#1,Rus=#2]{#2}
}

\newcommand{\ItemToText}{
	\Equip@Name(%
		\begin{small}%
		\IfEq{}{\Equip@Beschreibung}{}{%
			\Equip@Beschreibung; }%
		\IfEq{}{\Equip@Schaden}{}{%
			\Equip@Schaden; }%
		\IfEq{}{\Equip@Spez; }{}{%
			\Equip@Spez}%
		\end{small})%
}

\newcommand{\addToInv}[2][]{
	\NameToItem[#1]{#2}
	\IfSubStr{\Equip@Size}{Groß}{
		\edef\CharInvGr{\CharInvGr; \ItemToText}
		\setkeys{CharacterSheet}{InvGr=\CharInvGr}
		}{\IfSubStr{\Equip@Size}{Normal}{
			\edef\CharInvNr{\CharInvNr; \ItemToText}
			\setkeys{CharacterSheet}{InvNor=\CharInvNr}
			}{\IfSubStr{\Equip@Size}{Klein}{
				\edef\CharInvKl{\CharInvKl; \ItemToText}
				\setkeys{CharacterSheet}{InvKlein=\CharInvKl}
				}{
					\edef\CharInvGr{\CharInvSo; \ItemToText}
					\setkeys{CharacterSheet}{InvSonst=\CharInvSo}
				}
			}
		}
}
%TODO ADD Eigenschaften/Tallente/Rassen(aka Listenseite)

\makeatother

\title{System Morodax}

\CharacterSheetUser\CharacterSheet{}

\newgeometry{left=0mm,right=0mm,top=10mm,bottom=10mm}
\begin{landscape}
\newpage
\thispagestyle{empty} %% Remove header and footer.

\begin{center}

{\Large \underline {Listen Blatt}}
\vskip 0.3cm

\begin{minipage}[t]{0.65\textwidth}
\begin{tabulary}{\textwidth}[t]{rL}
\multicolumn{2}{c}{Rassen(die verbreitetsten):}\\
\hline
Primärattribut(+Attr): & +1 Bonus auf den Endwert(1 pro Rasse)\\
Gegenattribut(-Attr): & -1 Bonus auf den Endwert(1 pro Rasse)\\
Primärspez.(+Spez): & +1 Bonus auf den Endwert(1 pro Rasse)\\
\hline
Ork($\approx 2,0m$): & +Stärke +Ausdauer\\
Katzenmenschen($\approx 1,7m$): & +Geschick, +Akrobatik\\
Echsenmenschen($\approx 1,8m$): & +Geschick, +Ausdauer\\
Troll($\approx 2,5m$): & +Konstitution, +Einschüchtern, *\\
Mensch($\approx 1,8m$): & +Wahrnehmung, +Heimlichkeit\\
Morodianer($\approx 1,9m$): & +Charisma, +beliebige Mag.Verstand Spez.\\
Zwerg($\approx 1,2m$): & +alle tech.Attribute, -mag.Macht, +Zähigkeit\\
Elf($\approx 1,7m$): & +alle mag.Attribute, -tech.Geschick, +Verführen\\
Gnom($\approx 0,9m$): & +Magietechik(wissen)(inc. Magietechik Spez.)\\
Andere($\approx ?,?m$): &...\\
\\
\end{tabulary}
\begin{tabulary}{\textwidth}[t]{rL}
\multicolumn{2}{c}{Besondere(leere) Spezialisierungen(2):}\\
\hline
\multicolumn{2}{c}{Spezial Fähigkeiten des Charakters}\\
Neue Fähigkeiten:&Erlaubt Proben zu tätigen.\\
z.b. Magietechnik:&Magische Geräte untersuchen/bauen/...\\
Spezialisierte Fähigkeiten:&Erleichterung für Proben\\
z.b. Schwertkampf:&+1 Erfolge für Kampfproben, Konter ab Def.Erfolge=1,5x Angriffserfolge\\
'Allgemeinere' Spezialisierungen:&oft Anwendbar aber Proben sind schwerer\\
z.b. Zauberkenntnis:& Allgemeine Kennriss von Zaubern(kein Zauberweben), Schwierigkeit + 1\\
\\
\end{tabulary}

\begin{tabulary}{\textwidth}[t]{rL}
\multicolumn{2}{c}{Startwerte:}\\
\hline
Attribute:& 40 Punkte(AttrP) zum verteilen(übrige Punkte werden 1:4 in SpezP umgewandelt).\\
Startwerte:& Alle Attribute starten 'Unterdurchschnittlich'.\\
Spezialisierungen:& 100 Punkte(SpezP) zum verteilen(Pro Attribut zw. 1x und 3x der Punkte, die im Attribut vorhanden sind).\\
Startwerte:& Alle Spezialisierungen starten bei 'keine Kenntnisse'.\\
Lebenspunkte:& $Konstitutionswert(w8+8)+Zähigkeitswert(w4)$\\
Ausrüstungspunkte(AP):&5 * (5 oder 1d10).\\
\\
%\end{tabulary}

%\begin{tabulary}{\textwidth}[t]{rL}
\multicolumn{2}{c}{Eigenschaften(bis zu 2):}\\
\hline 
Ungläubiger: & Mag.Verstand(\&Spezialisierungen) für immer auf 0, +1 AttrP, +Spez\\
Idiot:& Tech.Wissen(\&Spezialisierungen) für immer auf 0, +1 AttrP, +Spez\\
Homosexuell: & siehe Regeln\\
Charaktermacke: & +Spezialisierung\\
Bsp. Vegetarier: & +Lebewesen\\
Bsp. Kleptomane: & +Heimlichkeit\\
...: & ...\\
\end{tabulary} 
\end{minipage}\nolinebreak
\begin{minipage}[t]{0.6\textwidth}\hfill
\begin{tabular}[t]{|c|c|c|c|c|}
\hline 
\multicolumn{2}{|c|}{\textbf{Klassifizierung}} & \multirow{2}{*}{\textbf{Wert}} & \multicolumn{2}{c|}{\textbf{Kosten}}\\ 
Attribut & Spezialisierung & & \textbf{AttrP} & \textbf{SpezP} \\ 
\hline 
\hline 
\multirow{2}{*}{Nicht Verfügbar} & \multirow{2}{*}{keine Kenntnisse} & \multirow{2}{*}{0} & - & - \\
\cline{4-5} 
&&&\multirow{2}{*}{0}&\multirow{2}{*}{1}\\
\cline{1-3} 
\multirow{2}{*}{Schlecht(max 3)} & \multirow{2}{*}{Grund Kenntnisse} & \multirow{2}{*}{1} & & \\
\cline{4-5} 
&&&\multirow{2}{*}{1}&\multirow{2}{*}{2}\\
\cline{1-3} 
\multirow{2}{*}{Unterdurchschnittlich} & \multirow{2}{*}{Grund Kenntnisse} & \multirow{2}{*}{2} & & \\
\cline{4-5} 
&&&\multirow{2}{*}{2}&\multirow{2}{*}{3}\\
\cline{1-3} 
\multirow{2}{*}{Durchschnitt(min 5)} & \multirow{2}{*}{Erweiterte Kenntnisse} & \multirow{2}{*}{3} & & \\
\cline{4-5} 
&&&\multirow{2}{*}{3}&\multirow{2}{*}{4}\\
\cline{1-3} 
\multirow{2}{*}{Trainiert} & \multirow{2}{*}{Erweiterte Kenntnisse} & \multirow{2}{*}{4} & & \\
\cline{4-5} 
&&&\multirow{2}{*}{4}&\multirow{2}{*}{5}\\
\cline{1-3} 
\multirow{2}{*}{Experte} & \multirow{2}{*}{Ausgebildet} & \multirow{2}{*}{5} & & \\
\cline{4-5} 
&&&\multirow{2}{*}{5}&\multirow{2}{*}{6}\\
\cline{1-3} 
\multirow{2}{*}{Genie} & \multirow{2}{*}{Ausgebildet} & \multirow{2}{*}{6} & & \\
\cline{4-5} 
&&&\multirow{2}{*}{6}&\multirow{2}{*}{7}\\
\cline{1-3} 
\multirow{2}{*}{...} & \multirow{2}{*}{Profi} & \multirow{2}{*}{7} & & \\
\cline{4-5} 
&&&\multirow{2}{*}{7}&\multirow{2}{*}{8}\\
\cline{1-3} 
\multirow{2}{*}{...} & \multirow{2}{*}{...} & \multirow{2}{*}{...} & & \\
\cline{4-5} 
&&&...&...\\
\hline 
\end{tabular}
\begin{tabulary}{\textwidth}[t]{rL}
\multicolumn{2}{c}{Spezialtalent(1):}\\
\hline 
$\bullet$ &Beidhändikeit (kann alles mit jeder Hand gleich gut \ref{SupportKampfaktionen})\\
$\bullet$ &Reaktiv (\ref{SupportKampfaktionen})\\
$\bullet$ &Zäher Mistkerl (nur bei Erschaffung, \ref{vital})\\
$\bullet$ &Talentiert: 1EP weniger bei Steigerungen(+20 Spez.Punkte bei Erschaffung) \\
$\bullet$ &Unerkannt (\ref{Unerkannt}) \\
%$\bullet$ &?Meisterschütze (Schnelleres Nachladen/?)\\
$\bullet$ &Mac Gyver(\ref{lange Proben})\\
$\bullet$ &Meistermagier(freischalten eines Basiszaubers ohne Buch(auch nicht Standardzauber wie z.b. Beschwörung), + 1 in Verstand- und Machtspezialisierung) \\
$\bullet$ &Kampfmagier(\ref{Kampfmagier}) \\
$\bullet$ &Nerd: nur bei Erschaffung, Wissensattribute auf 0, Wissensspezialisierungen beliebig wählbar, -3 AttrP, +15 SpezP(Wissen), 1EP weniger pro Spez.Punkt(Wissen) \\
$\bullet$ & ...(seid kreativ) \\
\end{tabulary} 
\end{minipage}

\end{center}
\end{landscape}
\restoregeometry

\newgeometry{left=5mm,right=5mm,top=10mm,bottom=10mm}
\begin{landscape}
\newpage
\thispagestyle{empty} %% Remove header and footer.

\begin{center}
{\Large \underline {Kurzreferenz}}
\vskip 0.3cm
\begin{minipage}[t]{0.45\paperheight}
\begin{description}
\item[Allgemeine Regeln:]\ 
	\begin{description}
	\item[Würfel:] Attributspunkte viele Würfel werden um Spezialisierungspunkte verbessert ($\xrightarrow{0}w20\xrightarrow{1}w12\xrightarrow{1}w8\xrightarrow{1}w4$ oder $-\xrightarrow{3}w12$).
	\item[Erfolge/Misserfolge:]Jeder Würfel der $\leq$ Probenschwierigkeit ist ist ein Erfolg.
	\item[Gegenprobe:]Beide machen eine Probe die Erfolge werden miteinander verglichen(die schlechtesten desjenigen der mehr erfolge hat werden als kritische erfolge aussortiert). Der beste Erfolg des einen gegen den schlechtesten Erfolg des anderen und so weiter. Der bessere Erfolg generiert jeweils 1 Teilerfolg, bei gleichstand gibts nix.
	\item[Passivität:]Wenn der Gegner nicht würfeln kann/will gelten alle Erfolge der Probe als kritische Erfolge.
	\item[Erschöpfung:] TODO.
	\end{description}
\item[Magie Regeln:]\ 
	\begin{description}
	\item[Voraussetzung:] Spezialisierungspunkte in der magischen Macht der Zauberkategorie.
	\item[Proben:] 1 Zauberweben- und 1 Zaubermachtprobe(2 Kampfrunden). Die Gegenprobe darf um je $1w20$ oder $1w12$ erschwert werden je 1 zusätzlichen Erfolg für Verbesserungen zu generieren(bzw. ein Kritischen Erfolg bei $1w12$).
	\item[Zauberweben:](Magischer Verstand)
		
	\textbf{Standardgegenprobe} 1w20 gegen Schwierigkeit 20.

	\textbf{Ungelernte Zauber} erfordern für alles die doppelte Anzahl an Zauberweben Erfolgen und werden gegen 2w20 gewürfelt.
	
	\item[Andauernde(temporäre) Zauber] benötigen in jeder Runde eine zusätzliche Zauberprobe(in der Zauberschwierigkeit). Falls weitere Zauber gewirkt werden sollen gelten die Regeln für mehrere Zauber.
	
	\item[Zaubermacht:] \textbf{Standardgegenprobe} gegen 0w20. Scheitert die Probe passiert nix.

	\item[Mehrere Zauber/Kombinationszauber:] als Spezialisierung wird die schwächere Spezialisierung benutzt Erfolge müssen auf beide Zauber verteilt werden. Pro Extrazauber wird die Gegenprobe um 1w12 erschwert.

	\item[Lebensdauer von Zaubern:]\
	 \begin{itemize}
		\item \textbf{Spontan:} direkter(dauerhafter) Effekt
		\item \textbf{Temporär:} andauernder Effekt
		\item \textbf{Permanent(Runden):} während der Aktivierungszeit wie Temporär, danach (fast) wie Spontan
	\end{itemize}
	\end{description}
\end{description}
\end{minipage}\hfill \begin{minipage}[t]{0.45\paperheight}
\begin{description}
\item[Kampf Regeln:]\ 
	\begin{description}
	\item[Kampfproben](Schwierigkeit 12) gegen gegnerische Abwehr(Nahkampf: 12/Fernkampf: 8) oder Ausweichen(Nahkampf: 8/Fernkampf: 12)
	\item[Nahkampfdistanz ändern] Akrobatikprobe(der der näher ran will) vs. Ausweichenprobe(der der weiter weg will)
	\item[zu nah für die eigene Waffe] gibt einen Spezialisierungsmalus auf Angriff(2*Falsche Distanz) und Verteidigung(Falsche Distanz)
	\item[zu weit weg für die eigene Waffe] gibt einen Spezialisierungsmalus auf Angriff(2*Falsche Distanz)
	\item[Schild] kann alles mit Abwehr oder Ausweichen(immer gegen 12) verteidigen. Überschreibt außerdem für Verteidigungsproben die eigene Waffenreichweite.
	\item[Trefferschaden] Leichte Treffer(kein kritischer Erfolg)= halber Schaden. Kritische Treffer(min 2 kritische Erfolge)= (kritische Erfolge - 1) * Schaden. 
	\item[Rüstung] Endschaden/Rüstungswert - Zähigkeit wird von den LP abgezogen. Wird die Rüstung umgangen zählt Rüstungswert=1.
	\end{description}	
\end{description}
\end{minipage}

\end{center}
\end{landscape}
\restoregeometry

\tableofcontents\newpage
\raggedright
\chapter{Generelles}
Dieses Regelwerk soll keine Vollständige Beschreibung von Morodax darstellen. Es ist vielmehr dazu gedacht einige Grundregeln zu beschreiben  und Beispiele/Orientierungen zur Balance zu liefern. Sollte ein Spieler/Meister neue Elemente hinzufügen wollen(Waffen/Ausrüstung/Zauber/...) soll dieses Dokument Anhaltspunkte zur Generierung eines solchen bieten.

\part{Die Welt}
\chapter{Die Geschichte}
\section{Entstehungsgeschichte/Religion}
Es gibt viele Religionen auf Morodax, sie alle haben eins gemeinsam: der Gott Morodis ist bei aller der (höchste) Gott, der Morodax erschuf bzw. der Welt leben brachte. Morodis wird dabei von vielen Religionen völlig unterschiedlich dargestellt, sowohl vom Aussehen als auch  von der Persönlichkeit her. Morodis selbst war/wird sein ein extrem mächtiger Magier-Cyborg, der eine Zeitreise durchgeführt hat, um aus seiner Zeit zu fliehen. Durch die Zeitreise(oder vielleicht auch schon vorher) hat er seinen Verstand beschädigt, wodurch er eine vielzahl von Persönlichkeiten entwickelt hat. Mithilfe von Illusionsmagie hat er dann öfter sein Aussehen komplett verändert und durch die Technologie der Zukunft konnte er auch bedeutend länger leben als alle anderen in dieser Epoche.

\section{Monde}
\begin{description}
\item[Zielel]\ 
\begin{description}
\item[Drehrichtung] wie unser Mond, parallel zur Rotationsachse der Welt
\item[Umlaufzeit] $\approx 18 Tage$
\item[wahrgenommene Größe] weniger als halb so groß wie unser Mond($\approx\frac{1}{3}$)
\item[Farbe] dunkelrot mit funkelnden(hellrot-glitzern) Stellen, bei Vollmond

 Leuchtet rot wenn er von hinten angestrahlt wird(einen anderen Mond/Sonne verdeckt).
 
 sonnst eher schwer zu sehen.
\item[Besonderheit] Riesiger roter Kristall, gilt als 'Magiemond'(ist in der Tat aus einem den Magiekristallen ähnlichem Material).
\end{description}
\item[Kochbalth]\ 
\begin{description}
\item[Drehrichtung] wie unser Mond, fast senkrecht($< 10^\circ$) zur Rotationsachse der Welt
\item[Umlaufzeit] $\approx 34 Tage$
\item[wahrgenommene Größe] $\approx$ so groß wie unser Mond
\item[Farbe] gelb (Vergleichbar mit Io)
\end{description}
\item[Althah]\ 
\begin{description}
\item[Drehrichtung] anders als unser Mond('(' = zunehmender Mond), fast senkrecht($< 10^\circ$) zur Rotationsachse der Welt
\item[Umlaufzeit] $\approx 33 Tage$
\item[wahrgenommene Größe] $\approx$ so groß wie unser Mond
\item[Farbe] lila
\end{description}
\item[Sariel]\ 
\begin{description}
\item[Drehrichtung] wie unser Mond, $\approx 30^\circ$ zur Rotationsachse der Welt
\item[Umlaufzeit] $\approx 46 Tage$
\item[wahrgenommene Größe] $\approx$doppelt so groß wie unser Mond
\item[Farbe] weiß-blau(gefrorenes Wasser)
\end{description}
\end{description}

\section{Kalender}
Ein Jahr hat 337 Tage bzw. 9 Monate(Jeha(33 Tage), Dailiel(46 Tage), Haelyr(34 Tage), Biniel(46 Tage), Saraq(33 Tage), Pyriel(46 Tage), Selik(34 Tage), Zaliel(19 Tage), Rielach(46 Tage)).

Eine Woche besteht aus 6 Tagen(Hielaph, Marach, Arah, Kusha, Sielyr, Zielah(Feiertag)).

\section{Zeitstrahl}
\begin{description}
	\item[Jahr unbekannt:] Ankunft von Morodis (Technischer Stand entspricht etwa der Stein- bis Bronzezeit).
	\item[Jahr 0:] Morodis verschwindet (aka. stirbt) Technischer Stand entspricht etwa der Antike.
	\item[Jahr 2984:] 'Heute'
	\begin{hiddenWorld}
	\item[Jahr unbekannt:] Abbau von Zielel, da dieser ein hocheffizientrm Magiekristall ist	.
	\end{hiddenWorld}
	\item[Jahr unbekannt:] Morodis Geburt
\end{description}

\section{aktuelle Situation}
Nach Morodis erscheinen auf der Welt(und seinem Wirken) sind die Grenzen zu anderen Dimensionen dünner als dies normalerweise sein sollte, dadurch ist das Magie relativ verbreitet und hin und wieder kommen Wesen aus anderen Dimensionen in unsere Welt(oder umgekehrt).
Diese beiden Effekte führen zu einer relativ hohen Anzahl an Monstern(entweder durch Magie veränderte Tiere oder Wesen aus anderen Dimensionen).

Die hohe Anzahl an Monstern hat unter anderem zur folge, dass Abenteurer gebraucht werden, die gegen die Monster kämpfen um Karawanen oder Siedlungen zu schützen oder einfach um wertvolle Materialien zu erbeuten.

\subsection{Kulturelle Situation}
Es gibt schon seit langem einen (schwelenden) Konflikt zwischen Elfen und Zwergen, in diesem waren die Elfen lange Zeit durch ihre Magie den Zwergen überlegen. Durch die Entwicklung von Arglardor(von anderen auch als Blutstahl bezeichnet, aufgrund seiner sehr dunkel roten Farbe), einem speziellen Magieresistenten Metall, konnten die Zwerge sich jedoch mittlerweile auch einen entscheidenden Vorteil verschaffen. Aufgrund der hohen Herstellungskosten kann dieses Metall jedoch nicht flächendeckend in einem Krieg verwendet werden. Aktuell kommt es daher nicht zu aktiven Auseinandersetzungen zwischen dem Reich der Zwerge und der Elfen, weshalb die Leute es den ruhenden Konflikt nennen.

Im Rest der Welt gibt es verschiedene Rassen, zwischen denen es auch z.t. zu Reibereien und Rassismus kommt. Jedoch gibt es auch das Land der Morodianer, Abkömmlinge von Teufeln und Dämonen, die von Morodis beschworen worden, in dem eine bunte Mischung aller Rassen lebt, die halbwegs funktioniert.

\begin{TODO}
\end{TODO}
\subsection{Das Geld}
Es gibt ein Währungsbündnis, hauptsächlich von den Morodianern initiiert, bei dem die angrenzenden Kulturen mitmachen, die allerdings z.t. eigene Prägungen haben(für Münzen)(vergleichbar mit dem System des Euros).Es gibt folgende Währungseinheiten:

\begin{tabular}{|c||c|c|c|}
\hline
Währung & \multicolumn{2}{c|}{ Wechselkurse } & Bemerkungen\\
\hline
\hline
1 Zinnmünze & - & $\frac{1}{1000}$ Goldmünzen & quasi ausgestorben\\
\hline
1 Kupfermünze & 10 Zinnmünzen & $\frac{1}{100}$ Goldmünzen & $\approx$1 Bier\\
\hline
1 Silbermünze & 10 Kupfermünzen & $\frac{1}{10}$ Goldmünzen & $\approx$1 Übernachtung\\
\hline
1 Goldmünze & 10 Silbermünzen & $\frac{1}{10}$ Goldmünzen & $\approx$1 Dolch\\
\hline
1 'Kupfer'-Geldkristall & 100 Goldmünzen & 100 Goldmünzen & $\approx$1 guter Wagen\\
\hline
1 'Silber'-Geldkristall & 10 'Kupfer'-Geldkristalle & 1K Goldmünzen & $\approx$1 einfache Yacht\\
\hline
1 'Gold'-Geldkristall & 10 'Silber'-Geldkristalle & 10K Goldmünzen & $\approx$1 Villa\\
\hline
1 'Mithril'-Geldkristall & 10 'Gold'-Geldkristalle & 100K Goldmünzen & $\approx$1 Schloss\\
\hline
1 'Adamantium'-Geldkristall & 10 'Mithril'-Geldkristalle & 1M Goldmünzen & $\approx$1 Berg\\
\hline
1 'Morodis'-Geldkristall & 10 'Adamantium'-Geldkristalle & 10M Goldmünzen & $\approx$1 Stadt\\
\hline
\end{tabular}
Die Geldkristalle sind winzige Magiekristalle(kaum größer als Münzen), die speziell angefertigt wurden um ihren Wert zu garantieren. Sie sind zudem fälschungssicher und diebessicher, da \begin{enumerate}
\item Jeder Kristall an seinen Besitzer gebunden ist.
\item Nur durch gegenseitiges Einverständnis kann die Bindung übertragen werden.
\item Manipulationsversuche/Fehlaktivierung/Schaden/... führen zur Aktivierung der Selbstzerstörung des Kristalls(Kristall \& Nutzer erhalten 200(d1) magischen, tödlichen und direkten Schaden(Direktangriff auf LP ohne Abwehrmöglichkeit))
\item Nutzen zur Identifikation den 'magischen Fingerabdruck' des Besitzers(die Bedienhand muss frei von Aglardor sein).
\item Besitzen einen Schutzschild, mit 50('Kupfer')-500('Morodax') Punkten Schutz \& Regeneration von 10('Kupfer')-100('Morodax') Punkten pro Kampfrunde.
\end{enumerate}
%Hinweis: 1 Gold~300€

\chapter{Die Magie}
Für die die sich nicht dem langwierigen Studium der Magie hingeben wollen hier ein Auszug aus der beliebten Buchreihe "Magie für Zwerge": "Man kann sich einen Zauber vorstellen wie ein komplexes Rohsystem wenn man in das eine Ende Dampf einspeist und damit das System unter Druck setzt werden die Rohre mit Dampf befüllt. Je nach Konstruktion der Rohre(oder der angeschlossenen Systeme) können jetzt die verschiedensten Dinge passieren, z.B. könnte eine Düse genutzt werden um den Dampf zu fokussieren und als Waffe zu benutzten oder in ein Rohr könnten viele Löcher eingebaut werden um eine riesige Nebelwolke zu erzeugen."

\section{Magie wirken}
Die Magie von Morodax ist eine allumfassende Energie, alle sind mit der Magie verbunden manche stärker manche schwächer, manche sind sich der Magie bewusst, manche wissen vielleicht gar nicht welches Potential in ihnen schlummert.

Um die Magie verwenden zu können ist eine natürliche Begabung zur Manipulation der Magischen Energien notwendig(Magische Macht) und das wissen darüber wie diese Begabung genutzt werden kann(Magischer Verstand). Dann ist es einem Magier möglich durch Bewegungen die Energien in die gewünschten Bahnen zu lenken, manche Magier schwören auch auf den positiven Einfluss von Zaubersprüchen, manche schaffen es wohl sogar nur durch Zaubersprüche die Magie zu lenken, wahre Meistermagier sprechen sogar davon dass eigentlich nur ein Gedanke benötigt wird und alles andere nur Beiwerk ist um den eigenen Geist zu fokussieren.

Das wirken eines Zaubers beinhaltet 2 Schritte:\begin{enumerate}
\item Das erstellen der Zaubermatrix, hierbei werden die "Bahnen" gelegt durch die später die Magische Energie hindurch geleitet wird. Diese "Bahnen" bestimmen maßgeblich die Art des Effektes den der Zauber erreichen wird/soll.

In der Magietechnik sind diese Bahnen Konstruktiv fixiert, sodass keine Veränderungen hieran vorgenommen werden können(betrifft auch in verzauberte Gegenstände, die einen bestimmten Zauber bereitstellen sollen).
\item Das "befüllen" der Zaubermatrix mit Magischer Energie, hierbei wird die Magische Energie der Umgebung durch den Magier in die Zaubermatrix geleitet. Durch die Menge der Magischen Energie wird die stärke des Effekts bestimmt, bestimmte Zaubereffekte können auch durch die Überlastung einer Matrix erreicht werden(z.b. Explosionen, ...), oder durch die unterschiedliche Verteilung der Magie auf verschiedene Kanäle der Matrix(Dauer vs Stärke eines Lichtzaubers).

Dieser Vorgang ist für den Magier mit einer gewissen Erschöpfung verbunden.

In der Magietechnik werden z.t. Kristalle eingesetzt um Magische Energie zu speichern und bei bedarf in die entsprechende Zaubermatrix einzuleiten. Dadurch kann mitunter eine bessere Kontrolle über die Menge der Magischen Macht ausgeübt werden als dies einem normalen Magier möglich ist.
\end{enumerate}

\section{Zauber Schulen}
\begin{description}

\item[Angriffsmagie:] Reine Magische Energie wird konzentriert und genutzt um Schaden an Feinden zu verursachen, in der modernen Zeit kann diese Energie aber auch von Magischen Werkzeugen benutzt werden.

\item[Illusion:] Täuschungen werden erzeugt um Leute abzulenken, zu verwirren oder gar zu ängstigen. Sie können auch genutzt werden um z.b. Licht zu erzeugen.

\item[Verstärkung:] Klassischer Supportzweig der Magie, beinhaltet Heilung, Schilde und Haushaltsmagie(Säuberung,...).

\item[Verwandlung:] Magie zur (dauerhaften) Änderung von Personen und Gegenständen, Veränderung von Körperlichen Attributen, Verwandlung in Tiere etc..

\item[Elementar:] Manipulation der Elementarmächte(Luft,Erde,Feuer,Wasser). Kann z.b. für Mächtige Angriffsmagie benutzt werden, die schwer zu blocken ist.

\item[Beschwörung:] Erlaubt Beschwörung von Geistern(Seelen z.b. auf einem Schlachtfeld), Skeletten(Überreste), Dämonen(magische Zirkel), ... in Kombination mit Elemenarmagie(Kombinationszauber) sind sogar Elementarbeschwörungen(Element als Fokus) möglich.
\end{description}

\chapter{Technischer Stand}

\section{Magietechnik}

Die Magietechnik wurde ursprünglich von den Elfen erfunden. Diese haben Magie statt sie direkt zu wirken in Gegenstände eingearbeitet, als Verzauberungen, die entweder dauerhafter Natur(magische Kleidung/Rüstung/Waffen/...) sind oder eine manuelle Magiezufuhr('Zauberstäbe'/...), zur Aktivierung der Zauberformel und den zugehörigen Effekten, erfordern.

Später wurde die Magietechnik allerdings von den Gnomen perfektioniert vor allem durch die Nutzung von Magiespeichern, womit auch wenig talentierte, diese Geräte nutzen können, bzw. ein konstanter Energiestrom über längere Zeit aufrechterhalten werden kann(was wiederum viel neue Anwendungsgebiete eröffnet). Mittlerweile haben die Gnome die Elfen in jeder Hinsicht bezüglich der Herstellung von Magietechnik überholt(auch was die Zaubermatrizen angeht).

Die größten Hersteller für Magietechnik sind bei den Gnomen zu finden Doktor Grordbort's magische Werkzeuge \& Waffen und Professor Felmorn's Kristallschmieden sind dort die beiden größten.
Doktor Gorodbort ist spezialisiert auf die Integration von Zaubermatrizen in technische Geräte, während Professor Felmorn auf die Herstellung und Entwicklung der Magiekristalle anführt.
Die beiden arbeiten aktuell gemeinsam an der serienreife des 'magischen Generators', der einen Energiekristall ohne manuelle zufuhr von magischer Energie aufladen kann. Aktuell werden besondere legendäre Magiekristalle ab der Kategorie Groß benötigt um einen magischen Generator zu erschaffen.

Magiekristalle werden in verschiedene Qualitäts-(einfach bis legendär) und Größenkategorien(winzig bis gigantisch) eingeteilt.

Einfache Kristalle werden z.t. künstlich hergestellt, zur Herstellung von Kristallen der Qualitätsstufe besser und höher werden die Überreste von magischen Monstern benötigt.

\begin{tabular}{|c|c|}
\hline
Größe & Verwendung\\
\hline
\hline
winzig & kleine Gegenstände\\
\hline
klein & normale Gegenstände(Handwaffen/Pistolen, ...)\\
\hline
mittel & große Gegenstände(Gewehre, ...)\\
\hline
größer & Vollrüstungen, ...\\
\hline
Groß & mobile bzw. (semi)statische Systeme z.B. Autos/Golems/Belagerungswaffen/...\\
\hline
Gigantisch & spezial(z.b. Versorgung einer Stadt)\\
\hline
\end{tabular}

\section{Technischer Stand nach Völkern}

\vbox{
\subsection{Die Zwerge}
\begin{itemize}
\item \textbf{Bewegungsmittel:} Hauptsächlich Dampfbetriebene Züge. \hidden{Als Verbindung zwischen Zwergenstädten(U-Bahn) mittlerweile hauptsächlich elektrische Züge.}
\item \textbf{Allgemeiner Eindruck:} Steampunk: \hidden{Innerhalb der Städte gibt es meist ein oder mehrere große Industriebezirke, die die Energieerzeugung übernehmen und Abgase gebündelt aus dem Berg heraus leiten. Die Energie wird dann über verschiedene Kanäle(Transmission, Rohre und z.t. auch Kabel) in die verschiedenen Ecken der Stadt transportiert.} Zwerge sind für hochwertige Mechanische Lösungen bekannt. Es gibt Geräte mit eigenen Energieerzeugern((kleine) Dampfmaschine, Handkurbel, ...), es gibt aber auch einige bei denen auswechselbare Energiespeicher(Sprengladung, Druckluftkapsel, etc.) benutzt werden.
\item \textbf{Benutzte Technologien:}\begin{enumerate}
	\item \textbf{Mechanik:} Komplexe mechanische Systeme (Taschenuhren,...) sind üblich.
	\item \textbf{Chemie:} Diverse Sprengstoffe, Gifte, Medikamente, ... sind den Zwergen bekannt.
	\item \textbf{Druck:} Primäre Energiequelle, entweder direkt(z.b. Presslufthammer) oder indirekt über z.b. Dampfmaschinen.
	\item \textbf{Magie:} nicht vorhanden
	\item \textbf{Elektrik:} \hidden{Hohe Verbreitung in größeren Zwergenstädten(insbesondere für Licht und erste Elektrifizierte Züge). Das Stromnetz wurde bisher allerdings nur innerhalb der Berge ausgebaut, weshalb die fortschritte der Zwerge in diesem Bereich }nicht bekannt\hidden{ sind. Mobile Energiequellen(mobile Generatoren oder Batterien) sind eher noch nicht besonders weit entwickelt. Insgesamt ist die Technik in diesem Bereich vergleichbar mit dem Anfang unseres 20. Jahrhunderts.}
\end{enumerate}
\end{itemize}
}

\vbox{
\subsection{Gnome und Co.}
\begin{itemize}
\item \textbf{Bewegungsmittel:} z.t. (Dampfbetriebene) Züge aber auch Kutschen(sowohl mit Magiekraft als auch mit Pferdekraft). Im Gegensatz zu den Zwergen interessieren sich Gnome auch für den Himmel/Sterne was u.a. zur Entwicklung einfacher Luftschiffen führt(Heißluftballons bzw. erste Zeppeline).
\item \textbf{Allgemeiner Eindruck:} Gnome sind eher Bastler und Erfinder als Ingenieure, daher sind die meisten gnomischen Geräte Einzelanfertigungen. Es gibt aber auch einige wenige große gnomische Massenproduzenten.

Die meisten gnomischen Handwerker und Erfinder bedienen sich der Magietechnik.

Insgesamt sind die Gnome technisch relativ hoch entwickelt.
\item \textbf{Benutzte Technologien:}\begin{enumerate}
	\item \textbf{Mechanik:} Etwas weniger komplex als bei den Zwergen aber trotzdem viel verwendet.
	\item \textbf{Chemie:} Zur Erstellung von Medikamenten, ...
	
			Gnomischer Sprengstoff wird mitunter durch Magie verstärkt(winzige Kristallsplitter), dieser ist jedoch deutlich teurer als regulärer Sprengstoff.
	\item \textbf{Druck:} Z.t. werden Dampfmaschinen als Energiequelle verwendet, jedoch eher selten, dementsprechend wird dieser Technikzweig von den Gnomen eher weniger benutzt. Hauptsächlich für die Luftfahrt relevant(Konstruktion der Luftschiffshüllen, Befüllung mit entsprechenden Gasen, ...).
	\item \textbf{Magie:} Vorreiter, wird für so gut wie alles benutzt.
	\item \textbf{Elektrik:} nicht vorhanden.
\end{enumerate}
\end{itemize}
}

\vbox{
\subsection{Die Allianz der Mannigfaltigkeit}
\begin{itemize}
\item \textbf{Bewegungsmittel:} z.t. (Dampfbetriebene) Züge aber auch Kutschen(meist mit Pferden) es sind auch hier erste Luftschiffe(experimentell oder militärisch) vorhanden.
\item \textbf{Allgemeiner Eindruck:} Es gibt technische Einflüsse und Gerätschaften der anderen Völker.
\item \textbf{Benutzte Technologien:}\begin{enumerate}
	\item \textbf{Mechanik:} Ähnlich wie bei den Gnomen.
	\item \textbf{Chemie:} Sprengstoffentwicklung liegt etwas hinter dem zwergischen Standard, Medizin/Gifte sind etwa auf dem gleichen Stand.
	\item \textbf{Druck:} Dampfmaschinen werden vor allem in Städten eingesetzt und häufig von Zwergen gebaut/gewartet.
	\item \textbf{Magie:} Wird in größeren Städten oft eingesetzt, viele Geräte stammen mehr oder weniger von den Gnomen.
	\item \textbf{Elektrik:} im Experimentalstatus.
\end{enumerate}
\end{itemize}
}
		
\vbox{
\subsection{Die Menschen}
\begin{itemize}
\item \textbf{Allgemeiner Eindruck:} Viele Einflüsse von anderen Kulturen, aber insgesamt weniger als bei der Allianz der Mannigfaltigkeit. Technisch somit in allen Punkten etwas unterhalb dieser, dafür ist jedoch vieles von dem was die Menschen verwenden auch von Menschen entwickelt und gebaut worden(was zu einem leicht anderen 'Grundstiel' im Design führt). Insgesamt sehr ähnlich zur Technologie Ende 18. Jhr.
\end{itemize}
}
		
\vbox{
\subsection{Die Elfen}
\begin{itemize}
\item \textbf{Bewegungsmittel:} Pferde, Segelboote oder Füße
\item \textbf{Allgemeiner Eindruck:} Elfen sind meist sehr naturverbunden und können mit Technik wenig anfangen. Die meisten Elfen nutzen Magie für die verschiedensten Aufgaben. Die einzigen technischen Geräte die bei Elfen anzutreffen sind, sind einfache mechanische Konstruktionen(Bögen, eventuell mal eine Armbrust oder Taschenuhr) oder magietechnische Geräte in ihrer Urform(so wie sie ursprünglich von den Elfen entwickelt wurden), was effektiv eine Zaubermatrix in einem Gegenstand ist, die manuell mit Magie befüllt wird(z.B. 'Zauberstäbe').
\item \textbf{Benutzte Technologien:}\begin{enumerate}
	\item \textbf{Mechanik:} rudimentär vorhanden(auch als Importware(Taschenuhren, ...)).
	\item \textbf{Chemie:} In Form von Heiltränken/Salben vorhanden(oft auch in Kombination mit Magie).
	\item \textbf{Druck:} nicht vorhanden.
	\item \textbf{Magie:} In Form von Zauberstäben, etc. vorhanden, jedoch meist ohne Kristalle benutzt. Kristalltechnik zum Teil als Lampen im Einsatz(in Städten).
	\item \textbf{Elektrik:} nicht vorhanden.
\end{enumerate}
\end{itemize}
}		
\section{Technischer Stand nach Anwendungskategorien}

\subsection{Waffentechnik}

Es gibt prinzipiell alles von klassischen Vorderladewaffen über Revolversysteme bis hin zu ersten Repetierwaffen.
Auch werden viele verschiedene Möglichkeiten der Waffenkonstruktion erdacht, neben Feuerwaffen gibt es auch welche, die Luftdruck, Federkraft oder Magie benutzten um Geschosse zu beschleunigen. Es gibt jedoch auch Waffen die unkonventionelle 'Munition' verschießen, so wäre z.b. eine Waffe denkbar, die einen bestimmten Zauber aktiviert.

Neben Schusswaffen wird Technik auch für andere Waffen verwendet z.b. eine Sprengladung an einem Hammer um dessen Schlagkraft zu erhöhen, ein Gefäß gefüllt mit explosivem Material oder Fallen.

\begin{Images}
\subsubsection{Beispiele}
Zuerst die beliebten Druckluftwaffen:

\includegraphics[width=0.333\textwidth]{Bilder/SteamWeapon/examples1.jpg}\nolinebreak
\includegraphics[width=0.333\textwidth]{Bilder/SteamWeapon/examples2.jpg}\nolinebreak
\includegraphics[width=0.333\textwidth]{Bilder/SteamWeapon/engine.jpg}

Das Herzstück dieser Waffen bildet ein Mechanismus, der von den Zwergen entwickelt und vertrieben wird, die meisten Hersteller von Druckwaffen nutzen dieses Modul als Basis.


Magische Waffen werden meist in Form von Strahlenwaffen konstruiert:

\includegraphics[width=0.27\textwidth]{Bilder/MagicWeapons/gun.jpg}\nolinebreak
\includegraphics[width=0.33\textwidth]{Bilder/MagicWeapons/examples.jpg}\nolinebreak
\includegraphics[width=0.4\textwidth]{Bilder/MagicWeapons/engine.jpg}

Es gibt auch 'reguläre' Feuerwaffen, diese sind allerdings seltener als die anderen Varianten.
\end{Images}

\subsection{Rüstungstechnik}

Es gibt wie auch bei Waffentechnik verschiedene Leute, die versuchen die Technologie der aktuellen Zeit zu nutzen um bessere Schutzmaßnahmen zu konstruieren. Von Rüstungen unterschiedlicher Machart über Schilde, die sich 'ausklappen' lassen, um die Vorteile verschiedener Schildtypen zu vereinen, bis hin zu neuen Techniken im Mauerbau wird auch hier viel probiert und experimentiert.

Es gibt sogar Leute, die versuchen Rüstungen durch den Einsatz von Magie zu verbessern, zum Beispiel durch Verzauberungen.

\subsection{Sonstige Technik}
Taschenuhren, Feuerzeuge aber auch Verbandsmaterial und vieles mehr besorgt sich der moderne Abenteurer um für jede Situation gewappnet zu sein.

\raggedright
\part{Regeln}
\chapter{Allgemeine Regeln}
\section{Zeiteinheiten}
\begin{itemize}
\item eine Kampfrunde(KR) dauert 3 Sekunden.
\item ein Tag dauert c.a. 24 Stunden.
\item eine Woche hat 6 Tage(5 Arbeitstage und 1 Feiertag).
\end{itemize}

\section{Größenkategorien}
Alle Größenwerte werden in Kategorien eingeteilt. Jede Kategorie ist dabei $\approx1,5$ mal größer als de vorherige. Diese Kategorien werden bei bedarf auch benutzt um andere Zahlensteigerungen zu beschreiben(z.b. Masse).
\begin{tabular}{|c|c|c|}
\hline 
Kategorie & Maximalgröße & Bemerkung \\ 
\hline 
0 & 0,225m & Gilt in Rechnungen als $\frac{1}{4}$ \\ 
\hline 
1 & 0,35m & Gilt in Rechnungen als $\frac{1}{3}$\\ 
\hline 
2 & 0,5m &\\ 
\hline 
3 & 0,7m & Gilt in Rechnungen als $\frac{2}{3}$\\ 
\hline 
4 & 1m &\\ 
\hline 
5 & 1,5m& \\ 
\hline 
6 & 2,25m& \\ 
\hline 
7 & 3,5m& \\ 
\hline 
8 & 5m &\\ 
\hline 
9 & 7m& \\ 
\hline 
10 & 10m &\\ 
\hline 
+6 & *10& \\ 
\hline 
17/18 &  & Größe eines \textbf{H-Hohn}\\ 
\hline 
\end{tabular} 
\section{Proben}
\label{AllgProben}
\subsection{Würfel}

Es wird gewürfelt: 1w20 pro Attributspunkt, mit Spezialisierungspunkten kann 1 Würfel um 1ne Stufe verbessert werden, dies kostet 1 Punkt pro Würfelstufe.

\begin{itemize}
\item 1 Attribustpunkt$\xrightarrow{1}w20$
\item Spezialisierungspunkte verbessern Würfel $\xrightarrow{0}w20\xrightarrow{1}w12\xrightarrow{1}w8\xrightarrow{0,5}w6\xrightarrow{0,5}w4(\xrightarrow{1}w3\xrightarrow{1}w2\xrightarrow{3}w1)$
\item Spezialisierungspunkte können auch für weitere Würfel benutzt werden $\xrightarrow{3}w12$(allerdings nur 1x pro w4 Spezialisierung).
\end{itemize}

Sollten die Spezialisierungspunkte negativ sein(z.b. durch Mali) werden die obigen Regeln invers angewendet, beziehungsweise 1 w20 pro negativen Punkt entfernt, sollte hierdurch der Würfelpool auf unter 1 sinken wird mit einem einzelnen w100 gewürfelt.

\subsection{Basis Probe}

Jeder Würfel wird jeweils getrennt ausgewertet und liefert einen (Kritischen) Teilerfolg($\leq$Schwierigkeit) oder Misserfolg($>$Schwierigkeit). Übliche Probenschwierigkeiten:
\begin{description}
\item[Primitiv(20)] Natürliche Fähigkeit ohne Risiko(über eine Straße gehen).
\item[Trivial(16)] Triviale Probe(über eine Straße rennen).
\item[Einfach(12)] Standardprobe(über ein Feld rennen).
\item[Mittel(8)] Eine etwas schwierigere Probe(In der Dämmerung arbeiten).
\item[Schwierig(6)] Schwierigere Probe, die nicht jeder schafft(Einen Felsen hochklettern/ eine versteckte Falle finden).
\item[Schwer(4)] Schwierige Probe, die einen Profi erfordert(Falle entschärfen/ausbauen).
\item[Meisterhaft(3)] Probe bei der selbst Experten Fehler machen können(halb aktivierte Falle entschärfen).
\item[Legendär(2)] 
\item[Göttlich(1)] 
\end{description}

\textit{\textbf{Beispiel:} 
Schlucki der Zwerg hat Konstitution 3 und Zähigkeit 7. Er will in einer Taverne Zwergenschnaps trinken und muss eine einfache Zähigkeitsprobe ablegen um nicht umzukippen. Er würfelt insgesamt mit 3w12 und 1w8 und hat somit garantiert 4 Erfolge, hat also keinerlei Probleme mit dem Schnaps. }

\subsection{Gegenprobe}

Aktionator und Verteidiger würfeln beide eine 'Basis Probe'(siehe oben) und sortieren ihre Teilerfolge(aufsteigend für den Angreifer bzw. absteigend für den Verteidiger).
Falls einer der beiden mehr Teilerfolge als der andere hat werden die höchsten Teilerfolge als 'Kritische Teilerfolge' zur Seite gelegt.
Die sortierten Erfolge beider Parteien werden paarweise verglichen und generieren:
\begin{description}
\item[+1 Erfolg] pro 'kritischem Teilerfolg' des Aktionators.
\item[+1 Erfolg] wenn Aktionator Teilerfolg$<$Verteidiger Teilerfolg.
\item[$\pm0$ Erfolge] wenn Aktionator Teilerfolg$=$Verteidiger Teilerfolg.
\item[-1 Erfolg] wenn Aktionator Teilerfolg$>$Verteidiger Teilerfolg.
\item[11 Erfolg] pro 'kritischem Teilerfolg' des Verteidigers.
\end{description}
Falls anschließend:
\begin{itemize}
\item Erfolge $>$ 0: Die Aktion gelingt mit min('kritische Teilerfolge' des Aktionators;Erfolge) kritischen Teilerfolgen und Erfolge-min('kritische Teilerfolge' des Aktionators;Erfolge) normalen Teilerfolgen.
\item Erfolge $\leq$ 0: Die Aktion schlägt fehl mit min('kritische Teilerfolge' des Verteidigers;-Erfolge) kritischen Teilerfolgen und -Erfolge-min('kritische Teilerfolge' des Verteidigers;-Erfolge) normalen Teilerfolgen für den Verteidiger.
\end{itemize}

\textit{\textbf{Beispiel:}Ein anderer Gast der Taverne sieht das und kommt zu Schlucki um ihn zu einem Saufduell herauszufordern. Der Gegner hat Konstitution 3 und Zähigkeit 2 also 1w20 und 2w12, beide machen ihre Proben Schlucki würfelt 3,7,9,12 der Gegner würfelt 15,7,7. Damit ergibt sich 1 Misserfolg für den Gegner(15), 2 Kritische Teilerfolge für Schlucki(12,9), 1 Erfolg für den Gegner(5 vs 7) und 1 unentschieden(7 vs 7). Somit bleiben für Schlucki 1 Kritische Teilerfolge(der andere muss genutzt werden um den Erfolg des Gegners zu neutralisieren) und für den Gegner 1 Misserfolg, somit gewinnt Schlucki das Duell und der Gegner hat sich überschätzt und kippt ohnmächtig vom Stuhl.}

\subsection{Ohne Gegner}
\label{ohneGegner}
Ohne Gegner gilt:
\begin{description}
\item[$\leq Schwierigkeit$] Erfolg.
\item[$\leq \frac{Schwierigkeit}{2}$] Kritischer Erfolg.
\item[$=1$] Erfolg + Kritischer Erfolg.
\end{description}

Der Spielleiter kann dann Minimalmengen von (kritischen) Erfolgen festlegen damit die Probe Erfolgreich ist:
\begin{description}
\item[1-3 Erfolge] Einfache Komplexität(Eine Falle in einem Haus finden).
\item[2 Erfolge\& 1 Krit] Standard Komplexität(Eine Falle im Unterholz finden).
\item[4 Erfolge\& 2 Krit] erhöhte Komplexität(Der richtigen Spur folgen wenn diese einer Straße folgt.).
\item[...] ....
\end{description}

\textit{\textbf{Beispiel:}Anschließend wird Schlucki übermütig und bestellt sich das spezial des Hauses muss jetzt eine mittlere Probe(Schwierigkeit 8) schaffen und da er mittlerweile schon einiges getrunken hat erhöht sich auch die Komplexität auf 'Standard'. Er würfelt 5,7,8,10, das reicht leider nicht für einen Kritischen Teilerfolg(höchstens 4). Somit endet Schlucki's Abend mit diesem Drink und auch er sackt zusammen.}

Vor allem für Zauberproben oder wenn der Spielleiter es zulässt:
Um mehr Erfolge zu generieren kann man sich die Probe freiwillig schwerer machen. Falls man dies tut bekommt man für je 2 Erfolge einen Extraerfolg pro Stufe der Erschwernis.

\textbf{Beispiel:} Für eine einfache Aufgabe(Schwierigkeit 12) wird eine schwierige Probe(Schwierigkeit 6) gemacht. Es wird dabei 4,1,6 und 3 gewürfelt, somit 3 normale und 2 kritische Erfolge also in summe 5 Erfolge. Hierfür bekommt man für die Erschwernis $2_{durch Extrastufen}*2_{\frac{5}{2}}=4$ Extraerfolge(normale) zum Ergebnis dazu und landet bei insgesamt 7 normale und 2 kritischen Erfolgen.

\subsection{Deutung der Probenergebnisse}
\begin{description}
\item[Kritischer Teilerfolg:] Deutliche Überlegenheit über den Gegner(kann auch als normaler Teilerfolg benutzt werden). Kann benutzt werden um außergewöhnliche Nebeneffekte auszulösen(z.b. Kritische Treffer, verstärkter Effekt, schnellere Ausführung, ...).
\textbf{Im Standardfall werden übrige Kritische Erfolge genutzt um die Gesamtprobe aufzuwerten('leicht erfolgreich'/'geradeso geschafft'$\rightarrow$'erfolgreich'$\rightarrow$'kritischer Erfolg'/besonders gut$\rightarrow$...}
\item[Teilerfolg:] Erfolgreiche Aktion/Reaktion. Kann benutzt werden um normale Nebeneffekte auszulösen.\textbf{Falls keine Verwendung für normale Teilerfolge gefunden wird dürfen sie 3:1 in Kritische Teilerfolge umgetauscht werden.}
\item[Misserfolg:] Patzer/Missgeschick/Komplikation. Kann benutzt werden um negative Nebeneffekte auszulösen(z.b. unkritischere/leichte Treffer, schlechterer Effekt, langsamere Ausführung, Werkzeug geht kaputt, Charakter fällt hin, ...).\textbf{Im Standardfall(falls keine explizite Verwendung für Misserfolge definiert ist/wird) wird ein Misserfolg benutzt um einen Kritischer Teilerfolg auf einen normalen herunterzustufen.(oder auch um normale Erfolge komplett zu entwerten(optional)) Misserfolge des Gegners bewirken entsprechend das Gegenteil.}
\end{description}

\subsection{Unterstützung/Gruppenprobe}
(TODO)

Bei einer Unterstützung erhält der Unterstützte die Hälfte(aufgerundet) der Spezialisierungspunkte des Unterstützers als Bonus für seine Probe.

\section{Regeneration}
\label{vital}
Es gibt 2 Möglichkeiten Leben wiederherzustellen. Entweder mit Unterstützung durch Magie (oder Medizin), oder indem man einfach wartet. Für jede Stunde ohne große Aktivität(kein Kampf oder schwere Arbeit also z.b. rumsitzen, schlafen, durch die Stadt bummeln, ...) regeneriert man $Konstitution+Ausdauer$ LP.

Vitale Charaktere regenerieren doppelt so viel. Außerdem haben sie auch mehr Leben(12+(6 o. 1d12) pro Konstitutionspunkt).

\section{Erschöpfung}

Charaktere werden durch Zauber, Kämpfe, (körperliche) Arbeit und langem wach bleiben erschöpft. Bei solchen Aktionen können/werden 'Erschöpfungspunkte' generiert.

Erschöpfungspunkte pro Aktion:
\begin{itemize}
\item[1] pro Kampfrunde(zeigt erst nach dem Kampf Wirkung)
\item[1] pro (Zauber)probe(außerhalb von Kämpfen) um diese zu 'beschleunigen'
\item[1] pro 'unbeschleunigtem' Zauber
\item[1(0)] pro rennen Aktion(während eines Kampfs)
\item[2(1)] pro sprinten Aktion(während eines Kampfs)
\item[1] pro 10min Harte Arbeit(z.b. Schmieden)
\item[1] pro 30min mittlere Arbeit
\item[1] pro 1h leichtere Arbeit(z.b. Marschieren)
\end{itemize}

Wenn ein Charakter bestimmte Erschöpfungsschranken überschreitet bekommt er Mali:
\begin{description}
\item[$>(Ausdauer+1)*5$:] -1 auf körperliche Attribute
\item[$>(Ausdauer+1)*10$:] -2 auf körperliche und geistige Attribute
\item[$>(Ausdauer+1)*15$:] -3 auf alle Attribute
\item[$>(Ausdauer+1)*20$:] -4 auf alle Attribute
\item[...:] -... auf alle Attribute
\end{description}
Hinweis: Wenn ein Körperliches Attribut auf 0 fällt sind damit verbunden Aktionen nicht mehr möglich.

Zum abbauen von Erschöpfungspunkten gibt es folgende Möglichkeiten:
\begin{itemize}
\item[1] pro 1h Ruhepause
\item[+1] pro 1h ohne schlechtes Bedingungen(Regen,Wind,...)
\item[+1] pro 1h beim Schlafen
\item[+1] pro 1h beim Schlafen in Sicherheit/mit Wachen
\item[+1] pro 1h beim Schlafen pro 'Luxusstufe'(Schlafsack, gute Matratze, Bett, ...).
\item[+1] pro 1h bei reichhaltiger Verpflegung.
\item[X] durch Aufputschmittel(magisch oder chemisch/pflanzlich)
\end{itemize}

\section{Erfahrungspunkte}
Im verlauf eines Abenteuers werden Erfahrungspunkte(EP) verdient. Mit EP können verschiedene Charakterverbesserungen erlernt werden.
\begin{enumerate}
\item[1+EP] einen Zauber erlernen
\item[3 EP] 1 Spezialisierungspunkt(SpezP)
\item[12 EP] 1 Attributspunkt(AttrP)
\item[21 EP] 1 Spezialspezialisierung(1 freies Spezialisierungsfeld des Charakterblattes befüllen, erlaubt entsprechende Proben, Start mit Spezialisierung 0)
\item[42 EP] 1 Spezialtalent
\end{enumerate}
	
\section{Bewegung}
%TODO Hex feld bewegung!
Es gibt 5 Arten der Bewegung:
\begin{enumerate}
\item Wandern: $(Konstitution + Ausdauer) * 3 * $Größe km pro Tag(~8-10h).
\item Gewaltmarsch: $Wandern * 1,5$, stark erschöpft am Abend.
\item Gehen: $(\frac{Fitness + 5}{3}) * (Größe)$ pro Kampfrunde.
\item Schleichen: $\frac{Gehen}{2}$.
\item Rennen: $Gehen * 2 + \frac{Akrobatik}{2}$ pro Kampfrunde.
\item Sprinten: $Gehen * 3 + Akrobatik$ pro Kampfrunde.
\end{enumerate}
Alle Bewegungsgeschwindigkeiten sind abhängig von der Körpergröße, diese werden aufgerundet auf eins der folgenden Werte: $0,5m; 0,66m; 1m; 1,5m; 2m; 3m; 4,5m; 7m$. Für Geschwindigkeitsproben können diese Größen als Modifikation der Probenerfolge betrachtet werden.

Je nachdem wie sich ein Charakter über was für Gelände bewegen will kann eventuell eine Probe gefordert werden um die Bewegung auszuführen.(beim Schleichen,Rennen\& Sprinten erhöht sich die Schwierigkeit)

Beispielschwierigkeiten für verschiedene Gelände:
\begin{tabular}{rc}
\multicolumn{1}{c}{Gelände} & Minimal notwendige Proben(Basis)\\
\hline
Wald: & 1 Trivial\\
Sandwüste: & 2 Trivial\\
lose Felsen: & 1 Einfach\\
Sumpf: & 1 Mittel\\
Treibsand: & 1 Schwer\\
\end{tabular}

\section{Schleichen}
Um dich unerkannt zu bewegen ist eine Heimlichkeitsprobe erforderlich, die Schwierigkeit dieser Probe wird Maßgeblich vom Gelände beeinflusst(z.b. Hausboden:einfach, quietschende Dielen: mittel, ...).

Ein Gegner hat die Optionen den Schleichenden wahrzunehmen(z.b. Wachposten oder generell Leute die in der nähe sind), dazu wird eine Wahrnehmungsprobe ausgeführt. Je nachdem was die entsprechenden Personen gerade machen wird die Probenschwierigkeit gewählt(Wachen auf Patrouille: einfach, Leute ohne Wachabsichten: mittel, aktiv abgelenkte/betrunkene: schwer)

Sollte der Gegner sich des Schleichenden schon bewusst sein(zumindest, dass er da ist) wird stattdessen eine Suchenprobe durchgeführt(Schwierigkeit -1).

Alternativregel für große Gruppen: pro Person einer Gruppe, vor der man sich verstecken will muss ein Erfolg generiert werden, kritische Erfolge für wachsame/suchende Personen einer Gruppe.

\section{Glücksproben}
Bei Glücksproben wird ein w20 geworfen, besonders glückliche Menschen(Glück>1, oder Glücksbonus) dürfen mehrere w20 werfen und das beste Ergebnis nutzen.

Anders als bei anderen Proben können auch Zwischenwerte(3,5,7,...) zu einem leicht anderen Ergebnis führen, je besser die Probe ausfällt desto besser ist das Ergebnis für den Charakter.
\chapter{Charisma Regeln}

Es gibt verschiedene Möglichkeiten andere Charaktere zu beeinflussen, diese haben jedoch unterschiedliche Nebeneffekte, von denen einige Dauerhafter Natur sind.

\section{Unerkannt}
\label{Unerkannt}
Mit dem Spezialtalent Unerkannt können bist du ein sehr unauffälliger Typ und andere werden Probleme sich an dein genaues Aussehen zu erinnern(halt so'n Durchschnittstyp). Dadurch sind alle gesellschaftlichen Effekte nur temporär(solange sie dich noch wahrnehmen können), dies betrifft sowohl durch Beeinflussung hervorgerufene Effekte(Einschüchtern/Verführen) als auch andere Effekte(Schlechtes Benehmen bis hin zu Diebstahl). Effekte die auf dich wirken bleiben unverändert.

\section{Rhetorik}
\begin{description}
\item[Anwendungen:] Überzeugen/Überreden/Lügen/Verhandeln($\pm 10\%$ pro kritisch(inklusive 0-fach kritisch))/...

Allgemein eben alles was mit Worten zu tun hat.

\item[Gegenprobe:] Menschenkenntnis

\item[Erfolg:] Der Gegner ist überzeugt.

\item[(Kritischer) Fehlschlag:] Der Gegner ist definitiv vom Gegenteil überzeugt.
\end{description}

\section{Einschüchtern}
\begin{description}
\item[Probe:] $Einschüchtern(Charisma) + \frac{Fitness}{2} w20s$

\item[Gegenprobe bei Humanoiden:] $Menschenkenntnis(Charisma) + \frac{Fitness}{2} w20s$

\item[Gegenprobe allgemein:] $Willenskraft(Konstitution) + \frac{Fitness}{2} w20s$

\item[Erfolg:] Der Gegner ist verängstigt und versucht möglichst schnell und dauerhaft von dir wegzukommen.

\item[(Kritischer) Fehlschlag:] Du hast angst vor deinem Gegner und wirst dich ihm nicht mehr widersetzen.
\end{description}

\section{Verführen}
\begin{description}
\item[Beschränkung:] nur gegen anders geschlechtliche(m$\rightarrow$w oder w$\rightarrow$m)

\item[Gegenprobe:] Verführen

\item[Dekowert:]Die Probe wird um X erschwert wenn gilt:$Dekowert<2^X*gegnerischer\ Dekowert$.

Basis Dekowert=(Verführen+1)*2

\item[Homosexualität:] Verführen nur gegen gleichgeschlechtliche, Verführung von Homosexuellen gegen Heterosexuelle oder umgekehrt: -2 Schwierigkeit bei der Verteidigung.

\item[Erfolg:] Der Gegner ist in dich verliebt und wird versuchen dir nah zu bleiben (potentielle Anhänglichkeit).

\item[(Kritischer) Fehlschlag:] Du bist in deinen Gegner verliebt und machst dir (e.v. falsche) Hoffnungen auf eine Beziehung.
\end{description}

\section{Täuschung}
\begin{description}
\item[Anwendungen:] Für optische/Akustische Täuschungen(vorgeben jemand anders zu sein). Beschreibt nicht wie gut man mit Worten täuschen kann!

\item[Probe:] Heimlichkeit mit dem Würfelpool von Charisma(Das Talent Unerkannt gibt einen Schwierigkeitsbonus)

\item[Gegenprobe:] Menschenkenntnis mit dem Würfelpool von Wahrnehmung

\item[Erfolg:] Gegner nimmt dir deine Imitation ab.

\item[(Kritischer) Fehlschlag:] Gegner durchschaut die Täuschung und betrachtet dich als Feind.
\end{description}

\chapter{Kampf Regeln}
In jeder Kampfrunde hat jeder Charakter 3 Aktionen(1 Haupt- bzw. Angriffsaktion(Haupt), 1 Defensivaktion(Def) und 1 andere/unterstützende Aktion(Sup)) zur Verfügung.

Jede dieser 3 Aktionen kann unabhängig von den anderen Aktionen über die gesamte Kampfrunde hinweg ausgeführt werden. Aktionen können auch hintereinander passieren, in einander übergehen oder fusioniert werden(siehe \ref{SupportKampfaktionen}).

Eine Kampfrunde besteht aus 4 Phasen:
\begin{description}
\item[1. Manöverphase:] Support-/Bewegungsaktionen nutzen.
\item[2. Hauptphase der Spieler:] Angriffe der Spieler auswürfeln.
\item[3. Hauptphase der Gegner:] Angriffe der Gegner auswürfeln.
\item[4. Endphase:] Kampfrunde beenden \& Wirkung der Angriffe auswerten.
\end{description}
\section{Manöverphase}
Für jeden Charakter wird gesammelt/ausgewertet was er mit seiner Supportaktion machen will.
Die Supportaktion der Kampfrunde kann z.b. folgendes sein:
\begin{itemize}
\item Bewegen(gehen).
\item Spezielles Kampfmanöver ausführen.
\item Spezielles Kampfmanöver unterstützen(2. Aktion).
\item Nachladen ausführen(2. Aktion).
\item Zaubern(2. Aktion).
\item (einfache) Interaktionen ausführen(z.b. unverschlossene Tür öffnen).
\item Reden.
\end{itemize}
Falls die Supportaktion für andere Aktionen mitbenutzt wird gibt es entsprechend keine Auswertung in dieser Phase.

Falls der Charakter sich für eine Bewegung entscheidet(auch wenn andere Aktionen mit betroffen sind(rennen/sprinten)):
\begin{description}
\item[Nahkampfdistanz eines Gegners betreten:] Standarddistanz=max(eigene Waffenreichweite, gegnerische Waffenreichweite).
\item[Nahkampfdistanz eines Gegners verlassen:] Verteidigung nur über Ausweichen(nicht Abwehr) in dieser Runde.
\end{description}

\section{Hauptphase}
Die Hauptaktion/Primärfokus der Kampfrunde kann z.b. folgendes sein:
\begin{itemize}
\item Angriff durchführen.
\item Spezielles Kampfmanöver ausführen.
\item Nachladen ausführen(kann weitere Aktionen benutzen).
\item Zaubern(benutzt weitere Aktionen).
\item (Komplexe) Interaktionen ausführen.
\item ...
\end{itemize}
Bei einem Angriff wird generell eine Angriffsprobe(Nahkampf) gegen eine Verteidigungsprobe ausgeführt.

\subsection{Nahkampfdistanzen}
Jede Waffe hat eine andere optimale Nahkampfdistanz, diese leitet sich aus der Größenkategorie der Waffe ab(eine 1,5m Waffe hat eine um 1 höhere optimale Nahkampfdistanz verglichen mit einer 1m Waffe).

Für Verteidigungsproben(Abwehr \& Schildabwehr) gibt es einen Malus in Höhe von $max(0,Waffendistanz-Kampfdistanz)*2$.

Für Angriffsproben gibt es einen Malus in Höhe von $|Kampfdistanz-Waffendistanz|*2$.

Beispiel 1: Kampf von einer 0,33m Waffe(A) gegen eine 0,66m Waffe(B). Die beiden Kämpfer halten eine Kampfdistanz die einer 0,5m Waffe entspricht.
Greift A B an bekommt A er -2 Spezialisierungspunkte auf seinen Angriff während B -2 Spezialisierungspunkte auf seine Verteidigung erhält.
Greift B A an bekommt B er -2 Spezialisierungspunkte auf seinen Angriff während A seine normale Verteidigung nutzen darf.

Wenn ein Schild getragen wird gilt bei Verteidigungsproben immer die Waffendistanz des Schildes statt der der eigenen Waffe.

Um die Nahkampfdistanz(für die nächste Runde) zu ändern gibt es verschiedene Möglichkeiten:
\begin{itemize}
\item Spezielle (bewegungs-)Manöver.
\item Bei einem Angriffswurf kann jeder Erfolg genutzt werden um die Distanz um 1 zu verringern.
\item Bei einem Verteidigungswurf(Ausweichen)kann jeder Erfolg genutzt werden um die Distanz um 1 zu erhöhen.
\end{itemize}

\subsection{Abwehraktion}
Es gibt verschiedene Verteidigungsmöglichkeiten, eine Verteidigung erfordert jedoch immer die Defensivaktion, weshalb jede Verteidigungsprobe immer gegen alle Angriffe der Kampfrunde gilt(bei jedem Angriff gelten aber alle für den Angriff relevanten Würfe):

\begin{tabulary}{\textwidth}{|C|C|C|C|}
\hline 
Aktion & Ausweichen & Abwehr & Schildabwehr\\ 
\hline 
\hline 
Probe & Ausweichen & Abwehr & Ausweichen \textbf{oder} Abwehr \\ 
\hline 
Schwierigkeit gegen Nahkampf & $+ 1$ & $\pm 0$ & $\pm 0$ \\ 
\hline 
Schwierigkeit gegen Fernkampf & $\pm 0$ & $+ 1$ & $\pm 0$ \\ 
\hline 
Schwierigkeit gegen größere Gegner(nur bei 'gezielten Angriffen') & \multicolumn{3}{c|}{$- 1$ pro Größenkategorie Differenz.} \\ 
\hline
Gegen Tech. Nahkampf & \multicolumn{2}{c|}{Todo} & ? \\ 
\hline 
Konteraktion Gegenschlag(nur mit Nahkampfwaffe) & (leichter) Treffer & Gegner entwaffnen & Schildschlag(1 Runde betäubt) \\ 
\hline 
Konteraktion Angriff umlenken(Verfälschung der Angriffsrichtung um Ziel auszuwählen) & Ja($\pm 0^\circ$ Verfälschung) & Ja($\pm 30^\circ$ Verfälschung). & Ja($\pm 90^\circ$ Verfälschung) \\ 
\hline 
Tech. Nahkampf Konter & - & Umlenkung(Tech. trifft den Gegner) & Nahkampf Konter? \\ 
\hline 
Voraussetzung & - & Waffe o. Metallarmschienen & Schild \\ 
\hline 
\end{tabulary}

Bei Konteraktionen gilt: ein normaler kritischer Erfolg generiert einen leichten Treffer auf das Ziel der Konteraktion/löst den beschriebenen Effekt aus, ein 2-fach kritischer Erfolg generiert einen normalen Treffer oder verdoppelt den Effekt, ein X-fach kritischer Erfolg generiert einen X-3-fach kritischen Treffer oder ver-x-facht den Effekt.

\subsubsection{Kleinerer Gegner}
Bei einem Angriff durch einen größerer Angreifer wird die Verteidigungsprobe für Fernkampf um 1 Stufe(4 Punkte) erleichtert pro Größenkategorie unterschied.

\subsubsection{Mehrere Gegner}
Wenn mehrere Angreifer ein Ziel angreifen wird die Verteidigungsprobe um 1 Stufe(4 Punkte) erschwert.
Wenn die Angreifer das Ziel umzingelt(Angriffswinkel $>120^\circ$, mindestens 3 Angreifer) haben wird die Verteidigungsprobe nochmals um 1 Stufe erschwert.

Die Verteidigungsprobe erhält außerdem einen Malus von 2 pro zusätzlichen Angreifer gegen den Verteidigt werden soll.

Man muss sich nicht gegen alle Gegner verteidigen(wodurch z.b. der Malus durch umzingeln vermieden, bzw. die Angriffsboni reduziert werden können), alle gegen die man sich nicht verteidigt können einen allerdings angreifen als ob man sich überhaupt nicht verteidigt(Probe ohne Gegenprobe \ref{ohneGegner}).

Wenn man sich gegen '0' Angreifer(z.b. durch Verteidigungsboni) wird die Schwierigkeit um 1 Stufe reduziert.

\paragraph{Gruppenangriff}
TODO
\paragraph{Gruppenverteidigung}
TODO


\subsection{Schaden}
Der Schadensmultiplikator des Angriffs ist $Anzahl\ kritischer\ Erfolge-1$. Wenn kein kritischer Teilerfolg erzielt wurde gilt es als leichten Treffer, mit einem Schadensmultiplikator von 0,5.
Der Schaden(mindestens 0 Punkte) berechnet sich aus:

$Grundschaden*Multiplikator(abgerundet)/Rüstungswert(aufgerundet) - Zähigkeit$
\subsubsection{Nahkampfschaden}
Bei Nahkampfwaffen ergibt sich der Grundschaden aus:

$(4 + Qualität) * Waffengröße * Fitness + Kraft$
\subsubsection{Faustkampf}
Faustkampfwaffen gelten als Größenkategorie eigene Größe-5. Für unbewaffneten Kampf als 'Waffenqualität' standardmäßig -2 genutzt, sodass als unbewaffneter Schaden: $(4-2) * (eigene Größe/1,5^5) * Fitness + Kraft$ rauskommt(auf die passende Größenkategorie gerundet).

\subsubsection{Fernkampfschaden}
Bei üblichen Fernkampfwaffen ergibt sich der Grundschaden aus:

$(4 + Qualität) * Waffengröße * Waffenmultiplikator + w12(beim 'Kauf')$

\section{Endphase}
Effekte(Schaden, Magie, ...) die in der Kampfrunde erzeugt werden gelten normalerweise erst für die nächste Kampfrunde. Schaden \& Co können/sollten zwar schon während der Kampfrunde berechnet und aufgeschrieben werden die Konsequenzen davon(Tod, Bewusstlosigkeit, ...) gibt es jedoch erst in der Endphase.

Alle Teilnehmer bekommen außerdem 1 Erschöpfungspunkt(zusätzlich zu eventuellen Erschöpfungspunkten durch spezielle Aktionen).

\section{Manöver}
\label{SupportKampfaktionen}
Manöver werden unterteilt in Standardmanöver und Spezialmanöver.

Standardmanöver sind immer/für jeden verfügbar:
\begin{description}
\item[Gehen:] ($Sup$)
\item[Rennen:] ($Atta + Sup$) +1 auf Abwehrschwierigkeit.
\item[Sprinten:] ($Atta + Sup + Def$)

\item[Stürmen:] ($Sup$) Reduziert die Nahkampfdistanz(für die nächste Runde) auf die eigene Waffenreichweite. Alle Distanzmali auf Angriffsproben werden auf die Verteidigungsproben übertragen.
\item[Wegspringen:] ($Sup$) Erhöht die Nahkampfdistanz(für die nächste Runde) auf die eigene Waffenreichweite. Alle Distanzmali auf Angriffsproben werden auf die Verteidigungsproben übertragen.

\item[Normaler Angriff:] ($Atta$) Kritische Erfolge werden genutzt um kritische Treffer zu erzeugen.
\item[Gezielter Angriff:] ($Atta$) Gegenprobe wird um die Rüstungsschwierigkeit verbessert dafür wird der Rüstungswert ignoriert(sonst wie Normaler Angriff).
\item[Schneller angriff:] ($Atta + Sup$) Jeder kritische Erfolg erzeugt 1 normalen Treffer. Jeder Erfolg erzeugt 1 leichten Treffer. Gegnerische Misserfolge werten leichte Treffer zu normalen Treffern auf. Eigene Misserfolge werten Treffer ab oder entfernen 1 leichten Treffer.
\item[Heftiger Angriff:] ($Atta + Sup$) Normale Erfolge werden genutzt um kritische Treffer zu erzeugen. Misserfolge gleichen sich aus und wandeln Kritische Erfolge in normale Erfolge bzw. neutralisieren normale Erfolge. Verteidigungsproben der Runde sind erschwert um 1 Stufe.

\item[Konter:] ($Sup??$) Ausführung einer Konteraktion(siehe Tabelle) bei kritischem Defensiverfolg.

\item[Nahschuss:] ($Atta$) ??
\item[Hastiges Nachladen:] ($Atta + Sup + Def$) Keine anderen Aktionen(inklusive Verteidigungen) während des Nachladens möglich. Probe notwendig, Erfolge reduzieren die Nachladezeit, Misserfolge erhöhen die Nachladezeit.
\item[Zielen:] ($Sup$) Bonus auf den nächsten gezielter Angriff(+2 Spez).

\item[Angriff mit 2 Waffen(Erfordert Spezialtalent Beidhändigkeit):]($Atta + Sup + Def(optional)$) Ermöglicht einen Angriff mit beiden Waffen, entweder auf 2 Gegner oder als '2 Angreifer' gegen einen Gegner. Die Schwierigkeit des Angriffs erhöht sich um 1.
Abwehraktionen werden ausgeführt als gebe es 1 Angreifer weniger.
Schaden ist mit 'normalen' Waffen um 3 statt 1 Kraft reduziert, mit 'großen' Waffen um 6 statt 3.
Große Fernkampfwaffen sind nicht nutzbar.

\item[Schnelle Reaktion(Spezialtalent):] ($Speziell$) Keine Überraschungsangriffe/Schleichangriffe auf dich möglich, Waffe ziehen als Unterstützungsaktion statt Hauptaktion.
\end{description}

Spezialmanöver benötigen spezielles Training(ähnlich wie Zauber) deshalb:
\begin{itemize}
\item Der Spieler beschreibt was er vorhat bzw. was die Aktion seines Charakters bewirken soll und wie sie aussehen soll(z.b. dreifacher Salto rückwärts auf den Kopf des Gegners mit Schwert voraus und ich nenne es den 'Todesdreher').
\item Falls die Aktion und Wirkung sinnvoll sind wird eine Manöverprobe festgelegt(meist eine passende Kampfprobe, z.b. Nahkampf, mittel\& komplexer) um das Manöver richtig auszuführen.
\item Je nach Aktion und Wirkung können weitere Proben notwendig sein(bzw. kann das Ergebnis entsprechend weiter/wiederverwendet werden).
\item Falls die Manöverprobe erfolgreich ist kann der Charakter die Aktion ausführen.
\item Falls die Manöverprobe kritisch fehlschlägt ...
\item Falls die Manöverprobe kritisch erfolgreich ist(2 extra Krits) hat der Charakter das Manöver sogar direkt gelernt(ohne EP-Kosten).
\item Spezialmanöver können für EP gelernt werden(siehe Zauber).
\item Ein einmal erfolgreich durchgeführtes Manöver kann der Charakter einfacher lernen(siehe 'lernen mit Buch' bei Zaubern).
\item Erlernte Manöver benötigen keine Manöverprobe mehr.
\item Proben für das Manöver sind für gelernte Manöver leichter als für ungelernte Manöver.
\end{itemize}
Beispiele für Spezialmanöver(Zu lernen wie Zauber Stufe 0):
\begin{description}
\item[Rundumschlag:] ($Atta + Sup$) Trifft alle die sich innerhalb der eigenen Waffenreichweite befinden.
\item[Finte:] ($Atta$) Normaler oder gezielter Angriff um 1 Stufe erschwert für beide Seiten.
\item[Umhauen:] ($Atta + Sup$) Kein schaden aber ein Erfolg haut den Gegner um(muss erst wieder aufstehen). Jeder kritische Erfolge zwingen den Gegner min. 1 Runde liegenzubleiben.
\item[Entwaffnen:] ($Atta + Sup$) Je nach Waffe gibts eine Erschwernis(+1 für normale Waffen, +3 für kleine Waffen, +2 für zweihändig geführte Waffen). Kritische Erfolge schleudern die Waffe weiter weg.

\item[2 Pfeile gleichzeitig schießen:] TODO
\end{description}

\chapter{Magie Regeln}
\section{Regeln}
\label{Kampfmagier}
Eine Zauberaktion benötigt eine volle Runde(Atta. + Sup. + Def.).
Manche Magier, sogenannte Kampfmagier, haben gelernt sich beim Zaubern auf das notwendigste zu beschränken. Sie sind so geübt im Zaubern, dass dies nicht ihre komplette Aufmerksamkeit benötigt, somit können sie parallel zu ihrem Zauber eine weitere Aktion ausführen(Zauberaktionen benötigt -1 Aktion).

Durch langes Training mit auf sie gewirkte Zauber haben Kampfmagier eine bessere Kontrolle über ihren Magieschild und können Spezialisierungspunkte darin investieren.

Jeder Zauber hat eine Stufe und eine Zauberschule, alle Proben für den Zauber werden mit der Spezialisierung der Schule(entweder als Verstands oder als Macht Proben) und der Probenschwierigkeit (Primitiv für Stufe -1,) Trivial für Stufe 0 Zauber, Einfach für Stufe 1, ... ausgeführt.

Um einen Zauber zu wirken muss man mindestens den entsprechenden Basiszauber erlernt haben, am besten sollte man den Zauber selbst auch studiert haben.

Ein Basiszauber bildet die Grundlage aller Zauber einer Spezialisierung und stellt eine sehr allgemeine Zaubermatrix dar(z.b. haben entsprechend alle Zauber der Angriffsspezialisierung eine ähnliche Zaubermatrix und damit ähnliche Bewegungen/Beschwörungsformeln).
Die Kenntnis des Basiszaubers einer Spezialisierung wird durch die Magische Macht Spezialisierung dargestellt, womit bei dieser für den 1. Spezialisierungspunkt zusätzlich zu den Spezialisierungspunktkosten noch die Kosten für den Basiszauber hinzukommen.
Ein Basiszauber selbst ist jedoch direkt anwendbarer Zauber sondern mehr eine Grundlage, die man verändert um einen 'nützlichen' Zauber zu kreieren.

Die Verbesserungseffekte, die für den Basiszauber gelten, gelten für alle Zauber der entsprechenden Spezialisierung(alle davon abgeleiteten Zauber).

\subsection{Zaubern}
Um Zauber zu wirken müssen 2 Proben durchgeführt werden eine mit Magieverstand um zu sehen ob der Zauber richtig gewirkt wurde und eine mit Magischer Macht um die Stärke zu bestimmen.

Jeder von Zauberproben generierte Erfolg kann benutzt werden um den Zauber stärker zu machen indem damit die (Sonder)effekte eines Zaubers ausgelöst werden(siehe Zauberliste(\ref{Zauberliste})/Zaubergenerierung(\ref{Zaubergenerierung})).

\subsection{Zauber kombinieren TODO?!}
Es ist möglich Zauber zu kombinieren, dafür werden beide Zauber gleichzeitig ausgeführt.

Dafür wird:\begin{itemize}
\item der Attributspool wird auf die zu wirkenden Zauber verteilt(für alle Proben), hierbei muss für jeden Zauber mindestens 1 Würfel benutzt werden, die Spezialisierungen sind allerdings unabhängig von einander.
\item alle Proben aller Zauber müssen gleichlang ausgeführt werden(wenn ein Zauber mit 2 Zauberwebenproben durchgeführt wird müssen alle anderen auch mit 2 Zauberwebenproben durchgeführt werden).
\item die Erschöpfungspunkte der Zauberproben werden aufsummiert(bei 2 Zaubern 2 temporären Erschöpfungspunkte pro Runde).
\item es müssen zusätzliche Kombinationserfolge generiert werden, damit die Zauber nicht scheitern.
Je Probe müssen (Zauberanzahl - 1) Erfolge mit Schwierigkeit=(Schwierigkeit des Schwersten Zaubers + 1) Kombinationserfolge erzielt werden.
Diese müssen von einem(oder mehreren) der beteiligten Proben gezahlt werden(z.b. 1. Zauber=mittlerer und schwerer Erfolg, 2. Zauber=einfacher Erfolg, 1 schwere Kombinationsprobe ist nötig => Ergebnis der Probe ist 1. Zauber=mittlerer Erfolg, 2. Zauber=einfacher Erfolg).
\end{itemize}
\subsection{Anhaltende Zauber}
Für anhaltende Zauber muss in jeder Runde eine zusätzliche Probe gemacht werden, ob der Zauber aufrecht erhalten werden konnte.
Dafür gilt:\begin{itemize}
\item Es wird für den Zauber in jeder Runde, die er aufrecht erhalten werden soll eine Zauberwebenprobe und eine Zaubermachtprobe ausgeführt.
\item Sollten mehrere Zauber gleichzeitig aufrecht erhalten werden oder zusätzlich neue Zauber gewirkt werden gelten die selben regeln wie bei mehreren Zaubern, der Attributspool muss also auf alle gewirkten und erhaltenen Zauber aufgeteilt werden mit min. 1 Punkt pro Probe.
\item Die Zauberwebenprobe muss mindestens einen(2 bei ungelernten Zaubern) Erfolg(e) in der Schwierigkeit des Zaubers generieren oder sie gilt als gescheitert.
\item Die Zaubermachtprobe benötigt nur 1 trivialer Erfolg, da temporäre Zauber kaum Energie abgeben.
\item Sollten eine Probe nicht ausgeführt werden können oder scheitern erlischt die Wirkung des Zaubers(e.v. gibt es einen Abklingeffekt).
\end{itemize}

\subsection{Zauberwebenprobe(Magie Verstand)}

Bei einer Zauberwebenprobe wird standardmäßig 1 kritischer Teilerfolg verbraucht um den Zauber erfolgreich/kontrolliert zu erzeugen.
Ohne diesen kritischen Teilerfolg ist der Zauber Ungezielt, die Effekte werden zwar normal bestimmt, jedoch wird bei der Aktivierung des Zaubers ein zusätzlicher Zielwürfel genutzt.

Für ungelernte Zauber sind alle Kosten verdoppelt und es werden entsprechend auch 2 kritische Teilerfolge für die Erzeugung verbraucht.

%Um einen Zauber besonders gut auszuführen kann der Zauberer den Umgebungswürfelpool um 1 w20 erweitern um bei einem Zaubererfolg 1 extra Erfolg für Verbesserungen zu erhalten.
%Der Zauber kann auch den Umgebungswürfelpool um 1 w12 erweitern um bei einem Zaubererfolg 1 extra Kritischen Erfolg für Verbesserungen zu erhalten.

%Der ein Zauber bei dem die 'Umgebungswürfel' gewinnen gilt als Fehlschlag ein Zauber bei dem sich alle erfolge ausgleichen wird ungezielt(gekaufte extra Erfolge werden dennoch eingesetzt, da diese die ursprüngliche Absicht des Zauberers darstellen).

Bei einem Fehlschlag($<1 Erfolg$ bzw. $<2 Erfolge$ bei ungelernten Zaubern) wird eine Glücksprobe ausgeführt und ein dem Ergebnis entsprechend ein zufälliges magisches Ereignis generiert. Die folgenden Tabellen sind Beispiele für mögliche Ereignisse.

\begin{TODO}
seltene Änderung des Magiebereichs -> Hälfte im gleichen bereich hälfte speziell/2 Würfel/...
\end{TODO}
\begin{tabulary}{\textwidth}{|c|C|}
\hline 
Fehlschläge (1w20) & Effekt (immer ungezielt wenn nicht näher definiert) \\ 
\hline 
\hline 
1 & Dein Zauber. \\ 
\hline 
2 & Jede Mag. Tech wird aufgeladen \\ 
\hline 
3 & • \\ 
\hline 
4 & zufällige Heilung \\ 
\hline 
5 & • \\ 
\hline 
6 & Zufällige Explosionen \\ 
\hline 
7 & • \\ 
\hline 
8 & Alles wird sauber \\ 
\hline 
9 & • \\ 
\hline 
10 & Drachenillusion(alle) \\ 
\hline 
11 & Gegenstände fliegen durch die Luft. \\ 
\hline 
12 & • \\ 
\hline 
13 & (Elementar)Beschwörung \\ 
\hline 
14 & Feuerball auf dich \\ 
\hline 
15 & zufälliger Schutz \\ 
\hline 
16 & Aktiven Zauber werden aufgelöst. \\ 
\hline 
17 & Licht \\ 
\hline 
18 & • \\ 
\hline 
19 & Du(+zufällige andere) verwandelst dich(nutzlose Erscheinung)\\ 
\hline 
20 & 'Angriff' auf dich. \\ 
\hline 
\end{tabulary} 

\begin{tabulary}{\textwidth}{|c|C|}
\hline 
Ungezielt(1w20) & Effekt \\ 
\hline 
\hline 
1 & Zauber wird auf alle Verbündeten des Ziels gewirkt. \\ 
\hline 
2 & Zufällig triffst du das richtige Ziel \\ 
\hline 
3 & \multirow{3}{*}{Ein Verbündeter des Ziels.} \\ 
4 & \\ 
5 & \\ 
\hline 
6 & \multirow{12}{10cm}{\center Umgebung/Neutral:\linebreak Niedrigere Zahlen haben positivere Nebenwirkungen.\linebreak Höhere Zahlen haben negativere Nebenwirkungen.} \\ 
7 & \\ 
8 & \\ 
9 & \\ 
10 & \\ 
11 & \\ 
12 & \\ 
13 & \\ 
14 & \\ 
15 & \\ 
16 & \\ 
\hline 
17 & \multirow{3}{*}{Ein Gegner des Ziels} \\ 
18 & \\ 
19 & \\ 
\hline 
20 & Zauber wird auf alle Gegner des Ziels gewirkt. \\ 
\hline 
\end{tabulary}

\subsection{Zaubermachtprobe(Magische Macht)}

%Wie bei der Verstandprobe kann auch für die Macht probe der Umgebungswürfelpool erweitert werden um extra erfolge zu erhalten. Für die Machtprobe ist der Standardpool allerdings leer.

Bei 0(oder weniger) Erfolgen verpufft der Zauber wirkungslos.

Für jeden Erfolg können wie auch beim Zauberweben bestimmte Effekte gekauft werden, diese sind vom jeweiligen Zauber abhängig. Anders als die Effekte der Zauberweben probe, die hauptsächlich die Kontrolle über den Zauber darstellen, sind die Effekte der Zaubermachtprobe eine Darstellung der Kraft/des Effekts eines Zaubers.

\subsection{Magie Verteidigung}

Jedes Geschöpf, dass mit der Magie verbunden ist hat einen natürlichen Schutzschild, der die Einwirkung schädlicher bzw. ungewollter Magie versucht zu verhindern. Dabei wird mit magischer Macht(Schild Spezialisierung) gegen den Zaubermachtwurf gewürfelt.

Dies betrifft ausschließlich Zauber, die das entsprechende Geschöpf direkt betreffen und nicht von diesem gewirkt wurden:
\begin{enumerate}
\item Zauber die direkten Schaden verursachen(betrifft nur direkten magischen Schaden(keine Elementarmagie)).
\item Verwandlungsmagie(außer Rückverwandlung).
\item Illusionen, die die körperliche Erscheinung verändern(Unsichtbarkeit, Illusionäre Verwandlung, ...).
\item Beschwörungsmagie.
\end{enumerate}

\subsubsection{TODO: effekte...}

Bei einem 'kritischen Treffer' werden die Erfolge des Ziels 'konsumiert' und stärken den Zauber(Gesamterfolge=Zaubererfolge + Verteidigungserfolge).

Bei einem 'leichten Treffer' wird der Effekt halbiert(2 Erfolge für einen Verbesserungspunkt).

Bei einem Fehlschlag kann der Zauber die Verteidigung nicht überwinden. 

Bei einem kritischen Fehlschlag prallt der Zauber ab und trifft ein zufälliges nahes Ziel.

\subsection{Lebensdauer der Magie}
Normalerweise ist ein Zauber eine einmalige Angelegenheit und erzeugt einen bestimmten Effekt, dies ist spontane Magie, bei der die Zaubermatrix für das Wirken des Zaubers 'verbraucht' wird. Die Zaubermatrix destabilisiert sich dabei durch das zuführen Magischer Kraft und der gewünschte Effekt tritt normalerweise auf wenn die Matrix zerbricht und die enthaltene Energie gezielt freigibt.

Einige Zauber haben jedoch eine stabilere Zaubermatrix, diese werden temporäre Zauber genannt, bei diesen tritt der Effekt auf solange die Zaubermatrix intakt und befüllt ist. Solche Zauber benötigen konstante Aufmerksamkeit um die Zaubermatrix über längere Zeit aufrechtzuerhalten. Eine Zuführung Magischer Energie nach der Aktivierung können diese Zauber allerdings nur sehr selten verkraften. Sobald der Zauberer sich nicht mehr um die Zaubermatrix kümmert(sei es durch Unfähigkeit oder weil er sie einfach nicht mehr braucht), zerfällt sie relativ schnell(innerhalb der nächsten Runde).

Eine dritte Kategorie sind sogenannte permanente Zauber, diese sind in vielen Punkten den temporären Zaubern sehr ähnlich, mit dem unterschied, das sich nach einer gewissen Zeit (bestimmten Anzahl an Runden) die Zaubermatrix stabilisiert, sodass sie auch ohne das Zutun von irgendjemandem erhalten bleibt, solche stabilisierten Matrizen lassen sich nur durch Spezialzauber auflösen.
Während eine permanente Magie noch in der Anfangsphase ist gelten jedoch die gleichen Regeln wie für temporäre Magie.

\chapter{Technik Regeln - Anwendung TODO}

Jeder durch Technik verursachte Schaden ist tödlich wenn es bei der Waffe nicht anders definiert ist, anders als bei Magischem und Nahkampfschaden kann man sich bei technischen Angriffen nicht zurückhalten, um den Gegner nicht zu töten....

\section{Gegenstände allgemein}
\subsection{Eigenschaften von Gegenständen}
Jeder Gegenstand kann bestimmte Eigenschaften besitzen:

\begin{itemize}
\item Stufe angegeben als Technikstufe-Qualitätsstufe bestimmt den Schwierigkeitsgrad für Konstruktion/Analyse/Verbesserung des Gegenstands.
\item Spezial listet Spezialfunktionen des Gegenstands auf, diese können reguläre Aktionen(Angriff/Nachladen/...) ersetzen, erweitern oder dauerhaft ändern.
\item Supportaktionen listet Spezialfunktionen des Gegenstands auf, diese benötigen mindestens eine Supportaktion um aktiviert zu werden.
\item Magazin/Verbrauch wenn im Magazin weniger ist als Verbraucht werden müsste kann das Gerät/Spezialfunktion nicht benutzt werden. Wenn nichts angegeben ist sind beide Werte 1.
\item Nachladen Zeit, die investiert werden muss bis das Magazin erneuert wurde einsetzbar ist. Ist für Technische Geräte relevant. Wenn nichts angegeben ist kann das Gerät nicht(normal) Nachgeladen werden.
\end{itemize}

\subsection{Haltbarkeit}
Jeder Gegenstand(Schwert, Tür, Mauer, ...) hat 10 Haltbarkeitspunkte(\textbf{HP}) wenn mit dem Gegenstand eine Aktion ausgeführt wird, für die er nicht geeignet ist(Schwert als Brechstange, Gewehr als Keule, Schaden, ...) wird eine Haltbarkeitsprobe durchgeführt. Eine Haltbarkeitsprobe hat triviale Schwierigkeit und nutzt die Haltbarkeitsspezialisierung des Gegenstands. Standardmäßig hat jeder Gegenstand 2 Haltbarkeitswürfel und 0 Spezialisierungspunkte.

Je nachdem wie unpassend die Aktion war/wie leicht sie zu Beschädigungen führt wird eine Erfolgszahl festgelegt (z.b. 3 für leicht unpassende Aktionen, 15 für mutwillige Zerstörung(Taschenuhr aus 20 m Höhe auf den Boden werfen)). Wenn die Probe scheitert verliert der Gegenstand HP in Höhe der Differenz von benötigten zu erreichten Erfolgen.

Wenn ein Gegenstand die Hälfte seiner HP verloren hat können keine Spezialaktionen mehr ausgeführt werden, außerdem werden wichtige Werte halbiert(halber Schaden bei Waffen). Durch eine Reparatur kann jedoch die volle Funktionsfähigkeit wiederhergestellt werden. Für Schusswaffen gilt jeder weitere Schuss als unpassende Aktion(3-5 je nach Beschädigung).

Wenn ein Gegenstand auf 0 HP sinkt zerbricht er völlig/wird komplett unbrauchbar, eine Reparatur ist dann kaum mehr möglich.

\subsection{Reparatur}
Es gibt 3 Varianten einen Gegenstand zu reparieren. Für die ersten beiden gilt: wenn die Ausführung scheitert muss eine erneute Haltbarkeitsprobe(7) gemacht werden.\begin{enumerate}
\item durch Magie
\item \label{Reperatur} durch eine normale Reparatur: Mechanik + Spezialerfolge(je nach Typ z.b. Chemie für Explosivwaffen) Schwierigkeit je nach Schaden und Technikstufe(1h Basiszeit, passendes Werkzeug) um den Gegenstand zu reparieren in der folgenden Tabelle wird angegeben, wie schwer es ist den Gegenstand um je 1 HP zu reparieren. Für eine vollständige Reparatur eines beschädigten Gegenstands muss eine Probe mit der Erfolgssumme aller Schadenswerte von 1 HP bis X HP geschafft werden.

\begin{description}
\item[1 HP Schaden:] 1 Technikstufe-2 Mechanikerfolge \& 2 Technikstufe-2 Spezialerfolge
\item[2 HP Schaden:] 2 Technikstufe-2 Mechanikerfolge \& 1 Technikstufe-1 Spezialerfolge
\item[3 HP Schaden:] 1 Technikstufe-1 Mechanikerfolge \& 2 Technikstufe-1 Spezialerfolge
\item[4 HP Schaden:] 2 Technikstufe-1 Mechanikerfolge \& 1 Technikstufe Spezialerfolge
\item[5 HP Schaden:] 1 Technikstufe Mechanikerfolge \& 2 Technikstufe Spezialerfolge
\item[6 HP Schaden:] 2 Technikstufe Mechanikerfolge \& 3 Technikstufe Spezialerfolge
\item[7 HP Schaden:] 3 Technikstufe Mechanikerfolge \& 1 Technikstufe+1 Spezialerfolge
\item[8 HP Schaden:] 1 Technikstufe+1 Mechanikerfolge \& 2 Technikstufe+1 Spezialerfolge
\item[9 HP Schaden:] 2 Technikstufe+1 Mechanikerfolge \& 3 Technikstufe+1 Spezialerfolge
\item[10 HP Schaden:] 3 Technikstufe+1 Mechanikerfolge \& 4 Technikstufe+1 Spezialerfolge
\end{description}
\item Einen Handwerker bezahlen(nur möglich bei weniger als 10HP Schaden): TODO!?
\begin{description}
\item[Kosten:] $Schaden*10\%$ des Grundpreises(abzüglich der Materialkosten(bis zu 50\%))
\item[Zeit:] $Schaden*10\%$ der Basisherstellungszeit
\end{description}
\end{enumerate}

\subsection{Wartung}
Nach Spielleiterentscheid können für Gegenstände Wartungskosten anfallen(bis zu 1\% pro Monat), wenn diese nicht (rechtzeitig) bezahlt werden verlieren die entsprechenden Gegenstände 1HP. Die Wartungskosten können u.a. auch Munitionskosten ect. mit abdecken.

\section{Fernkampf}

Siehe Kampfregeln, Schusswaffen als Angriffsprobe. Ein Schuss abzugeben benötigt eine Haupt oder Supportaktion.

Beidhändigkeit: Auch mit Fernkampfwaffen/kombiniert möglich. Bei 2 Fernkampfwaffen darf für jede bis zu 2 mal geschossen werden. Bei Nutzung 'Normaler' Waffen gibt es einen Malus von 1 Würfel aufs Zielen. Große Fernkampfwaffen benötigen mindestens Stärke 3 und geben einen Malus von 3 Würfeln. Das Nachladen muss für jede Waffe getrennt durchgeführt werden.

Eine bestimmte Körperstelle anzuvisieren(so es mit der Waffe überhaupt möglich ist) erhöht die Schwierigkeit des Angriffs um 1, dadurch kann z.b. eine tödliche Wunde in eine dauerhafte Verkrüppelung verwandelt werden(wenn Medizinischer Support vorhanden ist), es kann auch zu Schadensänderungen/Nebeneffekten führen.

Jede Waffe verfügt über eine Magazingröße, diese beschreibt wie oft geschossen werden kann bevor nachgeladen werden muss.

Jede Waffe hat eine Nachtladedauer, die angibt wie viele Aktionen zum Nachladen benötigt werden, pro Nachladerunde muss mindestens die Hauptaktion benutzt werden, es kann optional auch die Supportaktion oder die Defensivaktion mit benutzt werden(um bis zu 3 Nachladeaktionen pro Runde auszuführen).

Jede Waffe hat einen Schaden, dieser ist normalerweise fix(pro Kugel).

Sollte sich das Ziel im Nahkampf befinden besteht die Möglichkeit andere Kämpfer zu treffen, so auch wenn das Ziel durch Personen/Gegenständen gedeckt ist, in beiden Fällen kann der Spielleiter eine Erschwernis(z.b. mindestens X Erfolge um das Ziel zu treffen) festlegen, bei einem 'Fehlschuss' wird dann potentiell die Person/der Gegenstand getroffen.

Jede Waffe hat eine Reichweite für jeden Gegner, der näher dran ist wird die Schwierigkeit für einen Schuss um 1 erhöht. Ist das Ziel weiter als $2^X*Reichweite$ entfernt wird die Schwierigkeit um X+1 erhöht.

\section{Nahkampf}

Spezielle Nahkampfwaffen(z.b. ein Explosivhammer o. Federspeer) haben die Möglichkeit Spezialfunktionen zu aktivieren um den Angriff zu verbessern. Das aktivieren muss allerdings noch vor dem ausführen der Angriffsprobe angekündigt werden(außer mit dem Spezialtalent Schnelle Reaktion, mit dem keine Ankündigung nötig ist). Sollte der der Angriff fehlschlagen schlägt auch der Technische Angriff fehl.

Für den Technischen Angriff wird eine Probe auf sonstige Waffen durchgeführt gegen die Verteidigungsprobe des Gegners. Manche Waffen erfordern eine Mindesterfolge, die für eine erfolgreiche Aktivierung notwendig sind.

Jede Waffe verfügt über eine begrenzte Anzahl von Aktivierungen, und einer Nachtladezeit(üblicherweise nur außerhalb eines Kampfes möglich).

Auch Fallenstellen/-entschärfen wird mit dieser Spezialisierung geregelt, je nachdem wie geschickt der Fallensteller beim aufstellen der Falle vorgegangen ist ist es schwerer sie zu entdecken, sie zu entschärfen oder auf sie zu reagieren(Ausweichen als Def.Aktion wenn man sie bemerkt hat).

\section{Rüstungen}

Das Rüstungsgeschick wird benutzt um den Umgang mit Technologien zu beschreiben, die der Verteidigung dienen(Rüstungen, Schilde, Burgtore, ...).

Eine normale Rüstung hat folgende Werte:

Rüstungsklasse: Wenn ein Treffer erzielt wurde wird die Rüstungsklasse zu den Verteidigungserfolgen hinzugefügt und diese Summe anschließend verwendet um die Schwere des Treffers(leichter/normaler/X-Fach-kritisch) zu bestimmen. Rüstung kann einen Treffer jedoch nicht verhindern also selbst mit RK+Verteidigungserfolge=2*Angriffserfolge würde noch ein leichter Treffer generiert werden.

Minimale Anzahl von Basiserfolgen, welche bei jeder Verteidigungsaktion(auch bei unterlassener Verteidigung) erzeugt wird, sollte ein Probe weniger als die angegebenen Erfolge erzielen werden stattdessen diese Basiserfolge benutzt. 
Diese Basiserfolge können niemals einen kritischen Verteidigungserfolg generieren.

Rüstungsschutz: Schadensreduktion, diese besteht meist aus Relativwerten($\frac{1}{10}$, ...) es können jedoch auch zusätzlich Absolutwerte(als $+X$ mit $X\in\mathbb N_{> 0}$) angegeben werden, diese werden unabhängig vom Grundschaden bei Schaden subtrahiert. Absolutwerte sind meistens bei magisch verbesserter Ausrüstung zu finden.

Primäre Voraussetzungen/Sekundäre Voraussetzung/Tertiäre Voraussetzung: min. Wert in Gegenstände + Tech. Geschick zum tragen/verwenden der Rüstung bzw. um Mali zu reduzieren.

Primäre Mali: Negative Effekte durch Tragen der Rüstung. Dieser Malus ist aktiv solange die 3. Voraussetzung nicht erfüllt ist.

Sekundäre Mali: Negative Effekte durch Tragen der Rüstung. Dieser Malus ist aktiv solange die 2. Voraussetzung nicht erfüllt ist.

\section{Sonstiges}

Technik kann für vieles verwendet werden, alles außer den oben beschriebenen wird aktuell unter sonstiges zusammengefasst(z.b. der Umgang mit Lampen/Werkzeug/...).

Auch der Umgang mit medizinischem Gerät(Verbandskasten, ...) fällt in diese Kategorie. Heilung mit regulärer Medizin dauert länger als eine Magische Heilung wird aber ähnlich eruiert(z.b. Magie verstand Probe fällt weg, Technische Geschick(Sonstiges) Probe ersetzt die Zaubermachtprobe als Zeitprobe(1h, Verbände, mit Misserfolgen)).

\section{Technik Regeln - Erzeugung und Analyse}
Für die Herstellung(/Analyse) sind folgende Sachen nötig:
\begin{itemize}
\item Werkzeug
\item Material(oder ein zu analysierender Gegenstand)
\item Zeit(eine Einheit ist bei Analyse weniger($\sim \frac{1}{2}$h) als bei Herstellung($\sim 8$h))
\end{itemize}

Als nächstes wird die Technikstufe und Komplexität des Gegenstands bestimmt: TODO(abhängig von AP Kosten?)

Anschließend wird die eigentliche Probe durchgeführt, dabei gilt:
\begin{itemize}
\item Erfolg: wird benutzt um Komplexität zu erreichen
\item zusätzliche Erfolge: Reduzieren die Dauer um 1 Basiszeiteinheit oder halbiert die Zeit(je nach dem was weniger bringt)(normalerweise gilt Komplexität * Basiszeiteinheit)
\item Kritischer Erfolg: wie Erfolg + zusätzlicher Erfolg
\item Misserfolg: Erhöht die Dauer um 1 Basiszeiteinheit(siehe zusätzlicher Erfolg)
\end{itemize}

TODO: MAC Guyver/Kosten

Mit dem Spezialtalent Mac Guyver werden:
\begin{itemize}
\item Zeitproben um 50\% reduziert(oder auf Minutenbasis statt Stundenbasis berechnet wenn Probenschwierigkeit+1 in Kauf genommen wird)
\item oder Materialkosten um 50\% reduziert(oder auf nahezu 0 gesetzt wenn Probenschwierigkeit+1 in Kauf genommen wird)
\item oder Werkzeugerfordernisse um 50\% reduziert(oder auf nahezu 0 gesetzt(funktioniert mit irgendwelchem Werkzeug) wenn Probenschwierigkeit+1 in Kauf genommen wird)
\end{itemize}

\subsection{Alt}
Erzeugung und Analyse kann/wird über einen längeren Zeitraum durchgeführt, es werden also unterbrechbare Zeitproben, mit Bestrafung der Misserfolge gemacht.

Normalerweise benötigt jede Probe mindestens 1h Zeit und passendes Werkzeug(z.b. Werkstatt/Labor) als Katalysator.

Zusätzlich zu den pro Ressourcen pro Probe können zusätzlich Gesamtressourcen benötigt werden(ein X-Kg Klumpen Arglardor um eine Arglardorwaffe zu schmieden, ...).

Mit dem Spezialtalent Mac Gyver werden Erfordernisse der einzelnen Zeitproben etwas reduziert(entweder alle etwas(um $\approx25\%$) oder eine stärker($\approx50\%$)). 
Zusätzlich darf durch Erhöhung der Probenschwierigkeit um 1 je eine Voraussetzung stark reduziert werden($\approx90\%$).
Es wird dann z.b. nur noch eine Basisausstattung an Werkzeug(nur noch Hammer\& Zange statt Werkstatt/Schmiede) bzw. Ressourcen benötigt.
Die Gesamtressourcen können sich entweder zusammen mit den Probenressourcen reduzieren oder als Konstante erhalten bleiben, diese Entscheidung obliegt dem Spielleiter.

Die Proben werden immer auf Technikwissen durchgeführt die Spezialisierungen richten sich dabei nach den im technischen Gerät benutzten Grundprinzipien. Magische und elektrische Geräte können nur von entsprechenden Meistertechnikern verstanden werden.
Ein technisches Gerät, dass auf mehreren Grundprinzipien aufbaut benötigt auch entsprechend aufgeteilte Erfolge(z.b. Revolver Kombination aus Mechanik und Chemie benötigt von beidem je 10 Erfolge, also auch mindestens 1 Mechanik- und 1 Chemieprobe).

Die Probenschwierigkeit entspricht der Technikstufe des entsprechenden Gegenstands.
Die Menge der Erfolge für einen Erfolg ist von dem Projekt und der Qualitätsstufe des Projekts abhängig.

Üblicherweise benötigte Erfolge um ein Projekt erfolgreich abzuschließen(z.t. können weniger ausreichen um Zwischenergebnisse zu erhalten bzw. mehr Erfolge das Ergebnis verbessern. Bsp.: nach der Analyse einer Falle werden weitere Erfolge gesammelt mit dem Ergebnis, dass ein weiterer versteckter Auslöser entdeckt wird.):
\begin{itemize}
\item Analyse: $\frac{Qualitätsstufe}{2}$
\item Reparieren: Je nach Beschädigung siehe \ref{Reperatur}
\item Bau: 2x Qualitätsstufe.
\item Verbesserung: (Analyse, falls nicht schon gemacht) + Bau
\end{itemize}

\section{Magietechnik}
Mittles Magietechnik lassen sich Zauber in Gegenständen 'speichern'. Dies kann auf 2 verschiedene Weisen passieren: entweder wird nur eine Zaubermatrix im Gegenstand gespeichert(z.b. Zauberstab), dadurch entfällt beim Anwenden des Zaubers die Verstandprobe, es wird immer die im Gegenstand gespeicherte Matrix verwendet.
Die Machtprobe ist allerdings weiterhin notwendig(siehe Magieregeln) und Machtverbesserungen können normal ausgewählt werden.

Die 2. Option ist zusätzlich einen Magiekristall einzusetzen, der letztlich die Machtprobe übernimmt(falls genug Energie gespeichert ist).

Regeltechnisch wird beim erschaffen von Magietechnik 1 bzw. 2 Magietechnik Proben durchgeführt.

Die 1. Probe bestimmt die Zaubermatrix, e.v. kann eine zusätzliche Verstandprobe(leichter als normalerweise) gefordert sein, falls komplexere Zauber gespeichert werden sollen. Erfolge der Magietechnikprobe können genutzt werden um die Zaubermatrix zu 'verbessern'.

Die optionale 2. Probe bestimmt wie die Zaubermatrix mit Energie versorgt wird. Erfolge der Magietechnikprobe können genutzt werden um 'Machtverbesserungen' des Zaubers zu kaufen. Die menge der benutzten Erfolge beeinflusst auch wieviel Energie die Aktivierung des Gegenstands braucht.

Für die 2. Probe gilt:
\begin{description}
\item[1 benutzter Erfolg:] +4 Kapazitätseinheiten verbrauch
\item[1 benutzter kritischer Erfolg:] +6 Kapazitätseinheiten verbrauch
\item[1 unbenutzter Erfolg:] -2 Kapazitätseinheiten verbrauch
\item[1 unbenutzter kritischer Erfolg:] -3 Kapazitätseinheiten verbrauch
\item[1 Misserfolg:] +2 Kapazitätseinheiten verbrauch
\end{description}

Die Energiekristalle können jederzeit von jedem wieder nachgeladen werden. Dafür wird eine Magische Macht(Angriff) Probe gegen 12 ausgeführt:
\begin{description}
\item[1 Erfolg:] +5 Kapazitätseinheiten wiederhergestellt.
\item[1 Misserfolg:] 1 Kapazitätseinheit wiederhergestellt.
\end{description}

\chapter{Regelanhang: Sonderregeln}
\section{Trolle TODO?}
Trolle sind keine Fleischlichen sondern Mineralische Geschöpfe. Sie werden bis zu 7000 Jahre alt sind sehr selten. Die meisten Trolle bestehen aus Gesteinen, einige wenige bestehen aber auch aus Metallen oder Kristallen, diese sind aber auch deutlich kleiner als ihre gesteinischen Brüder. Trolle essen üblicherweise(zumindest zum teil) was sie sind(Ein Granittroll frisst Granit um Material für sein Wachstum zu erhalten).
\begin{description}
	\item [0-700 Jahre] Entstehungsphase
	\item [700-1500 Jahre] Wachstumsphase(0,5m zu 2m), erste selbstwahrnehmung/Bewegungen
	\item [1500-2500 Jahre] Jugendphase(2m-3m), Welt erkunden, Abenteuer erleben. 
	\item [2500-6000 Jahre] Weisheitsphase(<10m), Rückzug zu anderen Trollen, Philosophieren, ...
	\item [6000-7000 Jahre] Endphase(bis zu 20m), Troll findet seinen 'Endfelsen' und wird langsam eins mit ihm.
	\item [Geschlechtslos]
	\item [Mineralisch:] Körper funktioniert anders als bei anderen. Essen Mineralien/Erze/Steine/... können sich nur sehr schwer in 'Fleischsäcke' hineinversetzen.(gegenseitige Verständnislosigkeit)
	\item[Temperierter Geist:]\ 
	\begin{description}
	\item [$\bm{<\text{-}30^\circ C}$:] Geistige Attribute +1 je 1 vollen Punkten
	\item [$\bm{<\text{-}15^\circ C}$:] Geistige Attribute +1 je 2 vollen Punkten
	\item [$\bm{<\ \ 0^\circ C}$:] Geistige Attribute +1 je 3 vollen Punkten
	\item [$\bm{>\ 10^\circ C}$:] Geistige Attribute -1 je 3 vollen Punkten
	\item [$\bm{>\ 25^\circ C}$:] Geistige Attribute -1 je 2 vollen Punkten
	\item [$\bm{>\ 40^\circ C}$:] Geistige Attribute auf 0
	\end{description}
	\item [Haut] Die Haut eines Trolls ist härter als die eines Fleischsacks, daher ist die bei der Charaktererstellung ausgewählte Rüstung für den Troll seine Haut(je nach Rüstung ist dies dann eben Granit/Sandstein/...).
	\item[Größe:] 2-3m
\end{description}



\part{Ausrüstung}
\chapter{Gegenstandstände}
\section{Allgemeine Erklärungen}
Es gibt Inventarbeschränkungen(je nach Rüstung):
\begin{itemize}
\item Riesige Ausrüstung: Größere Größenkategorie als der Charakter(Stationär oder Transport mir Karren/..)
\item Große Ausrüstung: bis zu selbe Größenkategorie wie der Charakter(bzw. auf dem Rücken tragbar)
\item Normale Ausrüstung: bis zu Größenkategorie des Charakters -2(bzw. an Hüfte/Gürtel tragbar)
\item Kleine Ausrüstung: bis zu Größenkategorie des Charakters -5(In Taschen verstaubar) oder explizit als klein deklariert(Messer, ...).
\item Kleine Ausrüstung: Größenkategorie 0,25m ist die kleinste und gilt immer als kleiner Gegenstand.
\end{itemize}
Zusätzlich können Gegenstände jederzeit auch aktiv getragen werden, wodurch sie keinen Stauraum verbrauchen(Rüstung, Ringe, Amulette in sinnvoller Menge)(Mann kann also auch die ganze Zeit einfach immer mit seiner Großaxt in der Hand rumrennen....könnte aber Konsequenzen haben...).

Man kann Gegenstände auch in größere Inventurslots packen, dabei wird jedoch der ganze Slot verbraucht.

Normale und Große Slots sind normalerweise immer 'Außenlasten'(Schwertscheide am Gürtel/auf dem Rücken/...).

In einem zusätzlichen Behältnis(Rucksack, Karren, ...) können weitere nicht direkt benutzbare Gegenstände untergebracht werden.
\section{Generische Gegenstände}
\section{Waffen}
\begin{description}
\item[Größe:] Größenstufe * 3 AP(bzw. * 3 Gold)
\end{description}
\
subsection{Waffengröße}
Die maximal verwendbare Waffengröße(abgesehen von semi-stationären Waffen) ist die eigene Größenkategorie-1. Es gelten des weiteren folgende Regeln:
\begin{tabular}{|r|l|}
\hline
Größenkategorie relativ zum Benutzer & Einschränkungen\\
\hline
-1 & Fitness-1 für Schaden\\
 & Nur zweihändig\\
\hline
-2 & Fitness-1 für Schaden\\
 & (bei einhändiger Benutzung)\\
 & Für Fernkampf: nur zweihändig\\
\hline
-3 & Für Fernkampf: Malus bei einhändiger Benutzung.\\
\hline
$\leq-4$ & Keine Einschränkungen\\
\hline
0,25m & minimale Waffengröße\\
\hline
\end{tabular}

\subsection{Nahkampfwaffen}
Qualität ist bei Erstellung immer 'normal'(Basisschaden=4)

\begin{tabulary}{\textwidth}{|C|C|C|}
\hline 
Kosten AP &Kosten Geld &Bonus\\ 
\hline
1 AP	&	-	&		Schwer/Massiv(Axt): -1 Größenkategorie Platz\& Reichweite		\\
\hline 
1 AP	&	-	&		Langwaffe: +1 Größenkategorie Nahkampfdistanz		\\
\hline 
1 AP	&	-	&		Stumpf/Hammer: -1 Stufe gegnerischer Rüstungsschutz		\\
\hline 
1 AP	&	-	&		Klingenwaffe: +2 Nahkampfspezialisierung		\\
\hline 
1 AP	&	-	&		Mehrere Griffe: für eine Supportaktion kann die Waffenreichweite um bis zu 2 reduziert werden.(-1 Größenkategorie Schaden)\\
\hline 
1 AP	&	-	&		...\\
\hline 
\end{tabulary}
\subsection{Fernkampfwaffen TODO}
Qualität ist bei Erstellung immer 'normal'(Basisschaden=4)

TODO: Bonuskosten

\begin{tabulary}{\textwidth}{|C|C|C|}
\hline 
Kosten AP &Kosten Geld &Reichweite\\ 
\hline
4 AP	&*5	&200m	\\
\hline 
3 AP	&*4	&80m	\\
\hline 
2 AP	&*3	&30m	\\
\hline 
1 AP	&*2	&15m\\
\hline 
0 AP	&*1	&3m\\
\hline 
\end{tabulary}

\begin{tabulary}{\textwidth}{|C|C|C|}
\hline 
Kosten AP &Kosten Geld &Nachladezeit\\ 
\hline
4 AP	&*6	&3 Aktionen	\\
\hline 
3 AP	&*4	&6 Aktionen	\\
\hline 
2 AP	&*2	&9 Aktionen	\\
\hline 
1 AP	&*1,5	&20 Aktionen	\\
\hline 
0 AP	&*1	&10 min\\
\hline 
\end{tabulary}

Magazin: 1+AP

\begin{tabulary}{\textwidth}{|C|C|C|}
\hline 
Kosten AP &Kosten Geld &Schadenfaktor\\ 
\hline
-	&*15	&20x	\\
\hline 
-	&*10	&15x	\\
\hline 
4 AP	&*6	&10x	\\
\hline 
2 AP	&*4	&7x	\\
\hline 
1 AP	&*2	&4x	\\
\hline 
0 AP	&*1	&2x\\
\hline 
\end{tabulary}
\section{Kleidung/Rüstung}
\subsection{Schutzwirkung}
Rüstung wird durch das Rüstungsmaterial und die Bauweise bestimmt:
\begin{tabulary}{\textwidth}{|C|C|C|C|}
\hline 
Material & Rüstungsschutz(Schadensdivisor) & Malistufe & Kosten\\ 
\hline
Textil & 1,5 & 2 & 2 AP (1,5 Gold*Größe) \\
\hline 
Leder & 2 & 4 & 4 AP (3 Gold*Größe) \\
\hline 
Schweres Leder & 3 & 6 & 6 AP (6 Gold*Größe)\\
\hline 
'Kettenhemd' & 5 & 8 & 8 AP (10 Gold*Größe)\\
\hline 
Schuppen/Lamellen & 7 & 10 & 10 AP (14 Gold*Größe)\\
\hline 
Stahlplatte & 10 & 12 & 12 AP (18 Gold*Größe)\\
\hline 
Spezial(z.b. sehr dicker Stahl) & 15 & 14 & 14 AP\\
\hline 
Spezial(z.b. Argladorplatte) & 20 & - & -\\
\hline 
\end{tabulary}

\begin{tabulary}{\textwidth}{|C|C|C|C|C|C|}
\hline 
Bauweise & Verteidigungsbonus(bei gezielten Angriffen) & Verteidigungsbonus(bei gezielten Angriffen)(Fernkampf) & Malistufe & Kosten\\ 
\hline
Verstärkte Kleidung & 2 & 3 & 2 & 1 AP (Geldpreis *1)\\
\hline 
Brustpanzer & 4 & 6 & 4 & 2 AP (Geldpreis *2)\\
\hline 
+ Helm & 6 & 9 & 6 & 3 AP (Geldpreis *3)\\
\hline 
+ Arm/Beinschienen & 8 & 12 & 8 & 4 AP (Geldpreis *4)\\
\hline 
Vollrüstung & 10 & 15 & 10 & 5 AP (Geldpreis *5)\\
\hline 
...(z.b. magisch verbessert) & ... & ... & ... & ?\\
\hline 
\end{tabulary}

Die Malistufen der beiden Tabellen werden zusammengezählt und ergeben die folgenden Mali. Dabei gilt: 
\begin{description}
\item[Mali stufe 1:] werden benutzt falls der Nutzer $< Malistufensumme$ Punkte in Technisches Geschick + Gegenstände hat.
\item[Mali stufe 2:] werden benutzt falls der Nutzer $< \frac{Malistufensumme}{2}$ Punkte in Technisches Geschick + Gegenstände hat.
\item[Mali stufe 3:] werden benutzt falls der Nutzer $< \frac{Malistufensumme}{3}$ Punkte in Technisches Geschick + Gegenstände hat.
\end{description}

\begin{tabulary}{\textwidth}{|C|C|C|C|}
\hline 
Malistufensumme & Mali stufe 1 & Mali stufe 2 & Mali stufe 3\\ 
\hline
1-5 & Primär - 1 & Primär - 2 & Primär - 3\\ 
\hline 
6-10 & Primär - 1 & Primär - 3 & Primär - 3, Sekundär -1\\ 
\hline 
11-15 & Primär - 2 & Primär - 3, Sekundär -1 & Primär - 3, Sekundär -2\\ 
\hline 
16-20 & Primär - 2 & Primär - 3, Sekundär - 2 & Primär - 4, Sekundär -2, Tertiär -1\\ 
\hline 
21+ & Primär - 3, Sekundär -1 & Primär - 4, Sekundär -2, Tertiär -1 & Bewegung=0\\ 
\hline 
\end{tabulary}

'Primär':Malus auf Ausweichen, Heimlichkeit, Akrobatik, Zauberweben \& ähnliche besondere Spezialisierungen

'Sekundär':Malus auf Kraft, Ausdauer, Basteln \& Reparieren, Gegenstände(außer für die Rüstungsvoraussetzung) \& ähnliche besondere Spezialisierungen

'Tertiär':Malus auf Abwehr, Nahkampf, Wachsamkeit, Suchen, Schusswaffen, andere Waffen \& alle anderen Körper bezogenen Proben.

\subsection{Taschen}

\begin{tabulary}{\textwidth}{|C|C|}
\hline 
Kosten &Gegenstandlimit (kann jeweils mehrfach gewählt werden) \\ 
\hline
-&(+1 gigantischer Gegenstand)	\\
\hline 
0,75 AP(1 Gold)&+1 großer Gegenstand	\\
\hline 
0,5 AP(5 Silber)&+1 normaler Gegenstand	\\
\hline 
0,25 AP(3 Silber)&+1 kleiner Gegenstand	\\
\hline 
\end{tabulary}
\subsection{Sonstiges}
Magische Verzauberungen oder andere Extras können an Rüstungen angebaut werden sind jedoch nicht bei Charaktererstellung verfügbar. Die einzige Ausnahme sind Kühlungen für Trolle.

\begin{tabulary}{\textwidth}{|C|C|}
\hline 
Kosten &Wirkung \\ 
\hline
0,25 AP&Isolierung(Temperaturschwankungen werden ausgeglichen.)\\
\hline 
\end{tabulary}

\section{Besondere Materialien}
Es ist Möglich spezielle Materialien zu nutzen um die Ausrüstung zu verbessern, dies ändert nicht zwingend die 'Generatormatrix' des Gegenstands, es kann jedoch die Technikstufe(durch eine 'Minimalstufe' des Materials) und die Qualitätsstufe(Abgeleitet von den Extrakosten) eines Gegenstands erhöhen.
\subsection{Arglardor}
\vbox{\begin{description}
	\item Arglardor ein extrem seltenes, schweres, teures, robustes Metall, dass die Wirkung vom Magie negiert(sowohl eigene als auch fremde), jedoch nur direkte Magie(z.b. keine Elementarmagie), selbst kleine Mengen(etwa ein Amulett) tragen dazu bei, das Magie den Träger schwerer beeinflussen kann. Arglardor wird ausschließlich von Zwergen produziert, die das Geheimnis des Metalls mit keinem anderen Volk teilen, selbst unter den Zwergen kennen es nur wenige.;
	Als Rüstung oder Amulett dient es vor allem dem Schutz des Trägers vor Magie.
	Als Waffe kann damit Magie aufgelöst werden(Schaden an der Zaubermatrix und Kraftquelle) kann jedoch den Träger nicht mehr so gut abschirmen.
	\item[Als Vollmaterial:] 1.: Hohe Magieresistenz(siehe \ref{Arglardormamulett}) bis -immunität(nur bei den direkt berührten Körperteilen);
							2.: besonders haltbar(+3 auf Haltbarkeitsspezialisierung);
							3.: Rüstungen reduzieren Schaden mehr(+2 Rüstungsstufen für Schutz, allerdings auch +2 Malistufe);
							4.: Schilde wirken gegen Magie?;
							5.: Waffen lösen Magie auf(feste Matrix hat etwa 10HP + e.v. passiven RK und RS. Der Zauberer kann die Erhaltungsprobe verstärken um dem entgegenzuwirken jeder (angesagte) extra Erfolg negiert 1 Schadenspunkt.)
	\item[Als Veredelung:] 	1.: leichte Magieresistenz(Effekt siehe \ref{Arglardormamulett}, jedoch nur bei direkt berührten Körperteilen);
							2.: Rüstungen/Schilde können gegen Magische Angriffe benutzt werden (Unterstützung des passiven Magieschilds);
							3.: Waffen verursachen erhöhten schaden gegen Magie(Magische Gegenstände, ...)(+X), können jedoch die Zaubermatrix nicht direkt angreifen.
	\item[Kosten Vollmaterial, Groß:] +35AP/7 Kupferkristalle
	\item[Kosten Vollmaterial, Mittel:] +20AP/3 Kupferkristalle
	\item[Kosten Vollmaterial, Klein:] +10AP/1 Kupferkristall
	\item[Kosten Veredelung, Groß:] +8AP/70 Gold
	\item[Kosten Veredelung, Mittel:] +5AP/30 Gold
	\item[Kosten Veredelung, Klein:] +3AP/10 Gold
\end{description}}
	
\subsection{Mithril}
\vbox{\begin{description}
	\item Mithril ein extrem leichtes und doch stabiles Metall. Gegenstände aus Mithril sind leichter als normal und werden meist besonders elegant gestaltet.
	\item[Als Vollmaterial:] Rüstungen haben -2 Malistufe, Waffen sind leichter zu führen(+1 auf Fitness für Angriffe).
	\item[Kosten Vollmaterial, Groß:] +30AP/3 Kupferkristalle
	\item[Kosten Vollmaterial, Mittel:] +15AP/1,5 Kupferkristalle
	\item[Kosten Vollmaterial, Klein:] +7AP/50 Gold
\end{description}}

\subsection{Adamantium}
\vbox{\begin{description}
	\item Adamantium ein extrem seltenes, schweres, teures, robustes Metall. Anders als Arglardor ist Adamantium eine bekannte natürliche wenn auch sehr seltene Ressource. Adamantium ist ein magisches Metall und kann gegen Magie schützen oder Magie und nicht durch normale Waffen verletzbaren Wesen schaden zufügen.
	\item[Als Vollmaterial:] besonders haltbar(Bonus auf Haltbarkeitsspezialisierung), Rüstungen/Schilde reduzieren Schaden mehr(+1 Rüstungsstufe), Waffen fügen Bonusschaden zu, ....
	\item[Kosten Vollmaterial, Groß:] +55AP/2 Silberkristalle
	\item[Kosten Vollmaterial, Mittel:] +35AP/1 Silberkristall
	\item[Kosten Vollmaterial, Klein:] +20AP/3 Kupferkristalle
\end{description}}

\section{Magiekristalle}


\subsection{Kapazitäten}
\begin{tabular}{|c||c|c|c|c|c|}
\hline
\diagbox{Größe}{Qualität} & einfach & normal & besser & ausgezeichnet & legendär\\
\hline
\hline
winzig & 1d20 & ... & ... & ... & ...\\
\hline
klein & 1d100 & 50+1d100 & 150+2d100 & 300+3d100 & 1000+\\
\hline
mittel & 2d100 & 100+2d100 & ... & ... & ...\\
\hline
größer & 4d100 & ... & ... & ... & ...\\
\hline
Groß & 10d100 & ... & ... & ... & ...\\
\hline
Gigantisch & ... & ... & ... & ... & ...\\
\hline
\end{tabular}
\begin{tabular}{|c||c|c|c|c|c|}
\hline
\diagbox{Größe}{Qualität} & einfach & normal & besser & ausgezeichnet & legendär\\
\hline
\hline
winzig & 1 Gold & ... & ... & ... & nicht käuflich\\
\hline
klein & ... & ... & ... & ... & nicht käuflich\\
\hline
mittel & ... & ... & ... & ... & nicht käuflich\\
\hline
größer & ... & ... & ... & ... & nicht käuflich\\
\hline
Groß & ... & ... & ... & ... & nicht käuflich\\
\hline
Gigantisch & ... & ... & ... & ... & nicht käuflich\\
\hline
\end{tabular}

\chapter{Zauber TODO}
\newcommand\Verbesserung[1][1]{\item[$\xrightarrow{#1}$] }
Beschreibung der Beschreibungen: Verbesserungseffekte: '$\xrightarrow{1}$' bedeutet: durch Einsatz eines Verbesserungspunktes(= Ein (zusätzlicher) Erfolg) kannst du diese spezielle Fähigkeit freischalten. Eine solche Verbesserung kann üblicherweise beliebig oft gewählt werden, solange genügend Verbesserungspunkte beim Zaubern erzielt wurden.

\section{Überblick}
\label{Zaubergenerierung}
Zauber werden in verschiedene Stufen eingeteilt genau wie die Technik. Zauber mit höheren Stufen haben zunehmend ungewöhnliche Effekte, die auch zunehmend weniger mit der Beschreibung des Basiszaubers gemein haben müssen. Höhere Zauber können außerdem stärkere Effekte haben oder eine größere Bandbreite von Effekten gleichzeitig abbilden.
\begin{tabulary}{\textwidth}{|C|C|C|}
\hline 
Stufe & Beschreibung & Startkosten\\ 
\hline
Stufe -1 (primitive Zauber)&Aufladen von Magiespeichern u.ä.&Kann jeder\\
\hline
Stufe 0 (triviale Zauber)&Abgeschwächte Basiseffekte/Haushaltszauber&1 AP\\
\hline 
Stufe 1 (einfache Zauber)&Standarteffekte&4 AP\\
\hline 
Stufe 2 (normale Zauber)&Erweiterte Standarteffekte&8 AP\\
\hline 
Stufe 3 (Schwierige Zauber)&beliebige 'normale' Effekte&15 AP\\
\hline 
Stufe 4 (schwere Zauber)&Ungewöhnliche Effekte&-\\
\hline 
Stufe 5 (meisterhafte Zauber)&Hochrangig/(sehr) seltene Effekte&-\\
\hline 
Stufe 6 (legendäre Zauber)&Grenzeffekte der physikalischen Machbarkeit.&-\\
\hline 
Stufe 7 (göttliche Zauber)&Realitätsverändernde Effekte&-\\
\hline 
Stufe 8 (unmögliche Zauber)&Einzig Morodis konnte solche Zauber wirken.&-\\
\hline 
\end{tabulary}

\section{Zaubergenerierung}
Bei der Zaubererstellung wird folgendes für den Zauber bestimmt:
\begin{itemize}
\item genau 1 Zauberweben Basiseffekt(Verbesserungskosten: initial), durch den der Zauber zwar verwendbar aber nur begrenzt nützlich ist.
\item maximal 1 Zaubermacht Basiseffekt(Verbesserungskosten: initial, nur in Ausnahmefällen)
\item mindestens 1 Zaubermacht Verbesserungseffekt mit Kosten 1, der beliebig oft gewählt werden kann.
\item weitere Verbesserungseffekte(Menge ist je nach Zauberstufe begrenzt).
\begin{itemize}
\item Zauberweben Verbesserung: Form, Art, Genauigkeit u.ä.
\item Zaubermacht Verbesserung: Radius, Stärke, Dauer u.ä.
\end{itemize}
\item Die Zauberstufe ergibt sich dann aus der Art/Menge der Verbesserungen.
\end{itemize}
Ein generischer Zauber kann Verbesserungseffektgruppen haben(z.b. Zielgröße), die als 1 Verbesserungseffekt mit funktionaler Kostensteigerung gilt(Verbesserungspunkte=Maximalkosten(mit X=1 wenn der letzte Eintrag eine Formel hat z.b. 3+X: $Größe=Körpergröße*X$)).

Ein generischer Zauber der Stufe 0 kann normalerweise keinen der Standarteffekte, der Zauberschule, nutzen.
Zauber höherer Stufen sollten die Standarteffekte der Schulen mit nutzen(müssen für diese aber entsprechende Punkte 'ausgeben').

Für Zauber gelten folgende Richtwerte für die Anzahl und Güte der Verbesserungseffekte, es sind jedoch nicht zwingend festen Grenzen, durch die sich die Zauberstufe erhöht, da die Effekte vor allem sinnvoll zusammenpassen sollten.
\begin{tabulary}{\textwidth}{|C|C|C|}
\hline 
Stufe & max. Verbesserungseffekte & max. Verbesserungspunkte\\ 
\hline
Stufe 0 (triviale Zauber)&3&7\\
\hline 
Stufe 1 (einfache Zauber)&5&12\\
\hline 
Stufe 2 (normale Zauber)&7&18\\
\hline 
Stufe 3 (schwierige Zauber)&10&24\\
\hline 
Stufe 4 (schwere Zauber)&15&30\\
\hline 
Stufe 5 (meisterhafte Zauber)&20&35\\
\hline 
Stufe 6 (legendäre Zauber)&Alle&Alle\\
\hline 
Stufe 7 (göttliche Zauber)&Alle&Alle\\
\hline 
\end{tabulary}

\subsection{Verbesserungseffekte}
\begin{tabulary}{\textwidth}{|C|C|C|C|C|}
\hline 
Name & Beschreibung & Kosten Verbesserungspunkte & Basiszauberstufe & Levelpunkte Kosten \\ 
\hline 
Schaden & Einzelziel schaden(5 pro Erfolg) & 1 & 1 & 1 \\ 
 & Einzelziel schaden(7 pro Erfolg) & 1 & 2 & 1 \\ 
+AOE & +1 Zauberstufe & - & +1 & 1 \\ 
+Zeitschaden & +1 Zauberstufe & - & +1 & 1 \\ 
\hline 
Heilung & Einzelziel heilen(7 pro Erfolg) & 1 & 1 & 1 \\ 
+AOE & +1 Zauberstufe & - & +1 & 1 \\ 
+über Zeit & +1 Zauberstufe & - & +1 & 1 \\ 
\hline 
Schutz & +/-1 & 1 & 1 & 1 \\ 
\hline 
Wertänderung & +/-1 & 1 & 1 & 1 \\ 
Attribut & +/-1 & 1 & 1 & 1 \\ 
Spezialisierung & +/-1 & 1 & 1 & 1 \\ 
Anderes & +/-1 & 1 & 1 & 1 \\ 
(Situationsabhängig) & +/-1 & 1 & 1 & 1 \\ 
(Situationsunabhängig) & +/-1 & 1 & 1 & 1 \\ 
(mehrere Werte) & +/-1 & 1 & 1 & 1 \\ 
\hline 
Erschaffung & +/-1 & 1 & 1 & 1 \\ 
\hline 
Änderung & +/-1 & 1 & 1 & 1 \\ 
\hline
Bewegung & +/-1 & 1 & 1 & 1 \\ 
\hline 
Wissen & +/-1 & 1 & 1 & 1 \\ 
\hline 
\end{tabulary} 

\subsection{Beispiele}
Einige Beispiele beliebter und verbreiteter Zauber:
\begin{tabulary}{\textwidth}{|C||C|C|C|C|}
\hline 
Schule & Stufe 0 (Trivial, 1 AP) & Stufe 1 (Einfach, 4 AP) & Stufe 2 (Mittel, 8 AP) & Stufe 3 (Schwierig, 15 AP) \\ 
\hline 
\hline 
Ilusion & Licht & Ilusion/ WobelBobelBoden & Verändernde Illusion & ... \\ 
\hline 
Angriff & Aufladen Mag. Technik & Angriff & Direkter Schmerz /Explosion & ... \\ 
\hline 
Verstärkung & Reinigung & Heilung/ Schlid(Magie) & Schild(Elementar o. Physisch) & Schutzschild \\ 
\hline 
Verwandlung & Telekinese & Rückver-wandlung & Verwandlung selbst & Verwandlung andere \\ 
\hline 
Elementar (jeweils pro Element ein Zauber) & einfache Elementarmanipulation & Erzeugung von Elementen & Elementarangriff St. 2 & Elementarangriff St. 3; Elementarexplosion St. 2 \\ 
\hline 
Beschwörung & ??? & ??? & ??? & ???? \\ 
\hline 
\end{tabulary} 

\subsection{Zauberbücher}
Um einen Zauber zu meistern wird ein Zauberbuch benötigt. Dies kann(ähnlich wie bei normalen Gegenständen) je nach Angebot verschiedene Kosten haben(insbesondere Bücher des Typs Elementar/Beschwörung sind häufig teurer). Ab Magie Stufe 3 sind die Bücher ebenfalls extrem selten(und damit potentiell teurer). Listenpreise:
\begin{tabular}{|c||c|c|c|c|c|c|c|}
\hline 
Kosten & Basiszauber & Stufe 0 & Stufe 1 & Stufe 2 & Stufe 3 & Stufe 4 & Stufe 5\\ 
\hline 
\hline 
Lernen mit Buch & 1 SpezP & 1 EP & 2 EP & 4 EP & 7 EP & 11 EP & 15 EP\\ 
\hline 
Buch & 10 Gold & 1 Gold & 5 Gold & 10 Gold & 25 Gold & speziell & -\\ 
\hline 
\hline 
Lernen ohne Buch & über Spezialtalent & 4 EP & 8 EP & 16 EP & 30 EP & 50 EP & 75 EP\\ 
\hline 
\end{tabular} 

\addAndPrintSpell{Magie allgemein}{
	Beschreibung={Diese generalisierten Effekte sind für alle Zauber benutzbar.;
		Standard Reichweite Magie = Berührung.},
	MagVerstand={
		;$\xrightarrow{1}$: Einfacher Split: +1 zusätzliches Ziel(Magische Macht Erfolge werden auf beide Ziele aufgeteilt(beliebige Aufteilung nach dem Wurf möglich)).
		;$\xrightarrow{1 Krit}$: Perfekter Split: +1 zusätzliches Ziel(Magische Macht Erfolge - 1 für beide Ziele).
		;Hinweis: Bei einem Split gelten Verbesserungen für beide Zauber, es wird nur eine Erhaltungsprobe benötigt, jedoch auch nur eine Macht Probe ausgeführt.
		;$\xrightarrow{1}$: Reichweite 10m
		;$\xrightarrow{1}$: Reichweite + 1 Stufe(*1,5)
		;$\xrightarrow{1}$: + 1 Mag. Macht Spezialisierung.},
	MagMacht={Je nach Kategorie.}
}
\section{Illusion}
\addAndPrintSpell{Illusion allgemein}{
	Beschreibung={Es wird eine Täuschung erzeugt, die verschiedene Sensorische Eigenschaften haben kann(Licht, Ton, ...). Hochrangige Magiern ist es möglich damit auch direkt die Nerven eines Ziels zu beeinflussen, sodass nur dieses eine Ziel diese Illusion 'wahrnimmt'.
	}
	MagDauer={Temporär},
	MagVerstand={
		;$\xrightarrow{3}$: Illusion ist nur für ein einzelnes Ziel wahrnehmbar(Berührungsreichweite).},
	MagMacht={Gegenprobe mit Mag. Verstand Illusion(gleiche Schwierigkeit wie der Zauber) oder Wahrnehmung($Zauberschwierigkeit + 1$), ob die Illusion durchschaubar ist. Bei einem Kritischen Erfolg nimmt der Gegner die Illusion als wahrhaft echt an, bei einem normalen Erfolg erkennt der Gegner zumindest, das Magie im Spiel ist, bei einem Patzer kann die Illusion durchschaut werden, bei einem Kritischen Patzer ist die Illusion zu schwach um überhaupt Einflüsse zu haben.;
	Je nach Qualität der Illusionen können diese auch im laufe der Zeit durchschaut werden, bei einer 'unpassenden' Aktion(durch gleiten durch etwas, ...) werden Gegenproben um 1 erleichtert und neu ausgeführt(Gegenprobe leichter als Trivial ist automatisch bestanden).
		;$\xrightarrow{1}$: Gegenprobe benötigt +1 Erfolg.}
}
\section{Angriff}
\addAndPrintSpell{Angriff allgemein}{
	Beschreibung={Der Magier nutzt die Elementare Magie um meist zerstörerische Effekte zu erzielen, diese Magie kann aber auch als Energiequelle benutzt werden, da sie die magischen Energien nur minimal umwandelt. Mit höherer Kontrolle können auch Schockwellen u.ä. ausgelöst werden.;
	Die Energie kann auch genutzt werden um die Zaubermatrix anderer Zauber zu zerstreuen.;
	Bei einem Magieangriff gelten(soweit nicht anders spezifiziert) die selben Verteidigungsregeln wie für normale Angriffe(Fernkampf).},
	MagDauer={Spontan},
	MagVerstand={
		;$\xrightarrow{X}$: Angriff hat Verteidigungsschwierigkeit trivial und $X+3$ Angriffserfolge.
		;$\xrightarrow{X}$: Angriff hat Verteidigungsschwierigkeit einfach und $X$ Angriffserfolge.
		;$\xrightarrow{1,5*X+2}$: Angriff hat Verteidigungsschwierigkeit mittel und $X$ Angriffserfolge.
		;$\xrightarrow{2*X+4}$: Angriff hat Verteidigungsschwierigkeit schwer und $X$ Angriffserfolge.
		},
	MagMacht={
		;$\xrightarrow{1}$: +5 Schadenswürfel(d1) für den Zauber.
		},
}
\section{Verstärkung}
\addAndPrintSpell{Verstärkung allgemein}{
	Beschreibung={Die Magie wird genutzt um in einer positiven Weise auf Wesen/Objekte einzuwirken und schon existierende Eigenschaften zu verstärken. Dies beinhaltet u.a. die Stärkung der normalen Regenerativen Eigenschaften von Lebewesen, um Wunden innerhalb kürzester Zeit zu heilen, mitunter sogar solch, die auf normalem Wege nicht heilbar wären. Hochrangigen Magiern ist es sogar möglich nicht existierende Eigenschaften zu verstärken/erzeugen und somit Wesen bzw. Objekten neue Eigenschaften zu geben.},
}
\section{Verwandlung}
\addAndPrintSpell{Verwandlung allgemein}{
	Beschreibung={Die Magie wird genutzt um eine Permanente Veränderung zu erzeugen, dabei wird meist 'nur' das äußere(Körperliche) eines Objektes/Wesens verändert. Hochrangige Magier können aber sogar die inneren(Geistigen) Eigenschaften verändern, oder Veränderungen bewirken, die äußerlich kaum bis gar nicht sichtbar sind.;
	Sollte die Magische Macht nicht für eine vollständige Verwandlung reichen bleibt man in einer Mischform stecken.;
	Verwandlungsmagie Manifestiert sich nach 24h vollständig sollte in dieser Zeit ein weiterer Verwandlungszauber auf das Ziel angewendet werden ohne es vorher in seinen Ursprungszustand zurückzuversetzen kann es zu unschönen Komplikationen kommen, die Wahrscheinlichkeit dafür steigt mit jeder weiteren Verwandlung.;
	Braucht man für einen Verwandlungszauber länger als 1 KR verzögern sich auch alle anderen Effekte entsprechend der Zauberzeit(Vollständige Manifestation($24h*Zauberzeit$), Dauer der temporären Phase, ...).},
	MagDauer={Permanent(5 * Zauberzeit(in Kampfrunden))},
	MagMacht={
		;$\xrightarrow{1}$: Zielgröße $\pm 1$Größenstufe.
		;$\xrightarrow{2}$: Zielgröße $*2$ oder $\div2$ der Ursprungsgröße.
		;$\xrightarrow{1}$: 1 Attribut + 1 \& 1 Attribut - 1 (entsprechend dem Verwandlungsziel).
		;$\xrightarrow{4}$: 1 Attribut $\pm 1$.
		;$\xrightarrow{4}$: 1 Spezialisierung $\pm 1$.
		;$\xrightarrow{Automatisch}$: Vollständige Verwandlung wenn alle Attribute denen des Ziels entsprechen.
		;$\xrightarrow{8}$: 1 Spezialtalent/Spezialeigenschaft.
		}
}
\section{Elementar}
TODO
\addAndPrintSpell{Elementar allgemein}{
	Beschreibung={Manche Magier sind dazu in der Lage mithilfe der Magie direkt die Elemente zu beeinflussen um z.b. Feuer, Blitzte, o.ä. zu erschaffen und zu kontrollieren. Jeder Zauber, außer dem Basiszauber, ist nur für ein einzelnes Element gültig, ungelernt kann man nur Zauber anwenden die zu einem Element gehören für das man mindestens 1 Zauber gelernt hat.},
	MagDauer={Spontan},
}
\section{Beschwörung}
\addAndPrintSpell{Beschwörung allgemein}{
	Beschreibung={Wenige Magier haben die Kunst der Beschwörung gemeistert. Eine erfolgreiche Beschwörung besteht an sich aus 2 Zaubern, dem eigentlichen Beschwörungszauber, bei dem ein Tor geöffnet wird und das zu beschwörende Subjekt hindurch gezogen wird und einem Kontrollzauber, damit die beschworene Kreatur auch das tut, was sie soll.;
		Das Tor führt dabei normalerweise in eine andere Dimension und hält nur für den Bruchteil einer Sekunde, Meistermagier können jedoch die nötige Kontrolle aufbringen um ein solches Tor auch länger offen zu halten und beliege orte zu verbinden.;
		Der Kontrollzauber lässt sich normalerweise auch nur auf die Beschworenen Kreaturen anwenden, und auch dann nur in dem Moment der Beschwörung und muss aufrechterhalten werden um die Kontrolle zu behalten. Meistermagier können jedoch diesen Zauber jedoch auch für andere Kreaturen/Wesen verwenden.;
		Um ein Tor zu öffnen wird meistens ein Fokus verwendet, durch den die Toröffnung erleichtert wird. Auf einem Friedhof zum Beispiel sind die Grenzen zum Totenreich schwächer und Totenbeschwörungen somit einfacher.},
	MagDauer={Spontan(Tor) und Temporär(beherrschen)},
	MagVerstand={
		;$\xrightarrow{TODO}$: Beschwören von X
		;$\xrightarrow{TODO}$: Beschwören ohne passenden Fokus.
		;$\xrightarrow{TODO}$: Beschwören ohne Ritualkreis.
		},
	MagMacht={
		;$\xrightarrow{TODO}$: Kreatur Beschwörung
		;$\xrightarrow{TODO}$: Kontrolle (über Beschwörung)
},
}


\part{Anhang}
\begin{Beispiele}
\part{Beispiele}
\chapter{Gegenstandstände}
\section{Nahkampfwaffen}
\begin{multicols}{2}
\raggedcolumns
\addAndPrintItem{Dolch}{
	Schaden={$\frac{Stärke+Geschick}{2}d4$},
	Kosten=1AP, Preis={1+ Gold}, Stufe=0-5, Size=Klein}%
%
\addAndPrintItem{Spritze}{
	Schaden={1},
	Kosten=1AP, Preis={?}, Stufe=1, Size=Klein,
	Spez={Je nach Inhalt}}%
%
\addAndPrintItem{Einhänder}{
	Schaden={$(Stärke)d6$},
	Kosten=4AP, Preis={4+ Gold}, Stufe=1-7+, Size={Normal},
	Spez={Je nach Waffe;
	Die andere Seite(Schwert): nicht tödlicher Schaden;
	Einschläger(Hammer): Angriff gegen Türen/Schlösser/...(ohne Haltbarkeitsverlust)}}%
%
\addAndPrintItem{Zweihänder}{
	Schaden={$(Stärke-1)d10$},
	Kosten={5AP}, Preis={5+ Gold}, Stufe=1-9+, Size={Groß},
	Spez={Je nach Waffe siehe auch Einhänder}}%
%
\addAndPrintItem{Explosivhammer}{
	Schaden={$(Stärke-1)d12$},
	Kosten={15AP}, Preis={45 Gold}, Stufe=2-18, Size={Groß}, Mag={3d4 Einheiten(baubedingt)},
	Nachladen={90 Aktionen}, Verbrauch={$Xd4$ ($X\in\mathbb N_{> 0}$)},
	Spez={Explosive Verstärkung: Min(Xd4, Restkapazität) Einheiten werden verbraucht(X beliebig wählbar). Jede Einheit generiert 1 zusätzlichen Stärkewürfel.},
	Bild=Waffenbilder/Hammer.jpg}%
%
\addAndPrintItem{Federspeer}{
	Schaden={;Kurz:$(Stärke)d8$;Lang:$(Stärke-1)d12$},
	Kosten={11AP}, Preis={22 Gold}, Stufe=2-11+, Size={Normal}, Nachladen={7 Aktionen},
	Supp={Verlängern: Auslösen der Feder um die volle Länge zu erreichen. Verursacht 20 Schaden und schaltet die Waffe vom Kurzspeer Modus in den Langspeer Modus.}}%
%
\addAndPrintItem{Feuerschwert}{
	Schaden={$(Stärke)d6$},
	Kosten=18AP, Preis={75 Gold}, Stufe=2, Size={Normal}, Mag={20+4d12 Einheiten(baubedingt)},
	Nachladen={15 Aktionen}, Verbrauch={2..5d8},
	Spez={Feuerschlag: Jede verbrauchte Einheit generiert 1 zusätzlichen Feuerschaden(ein genügend großer Feuerschaden kann Sachen in Brand setzen).},
	Bild=Waffenbilder/flameSword.jpg}%
%
\addAndPrintItem{Revolverschwert}{
	Beschreibung={Durch erstklassige zwergische Handwerkskunst konnte Spezialmunition geschaffen werden, die eine eigene Minidruckluftkapsel enthält, wodurch die Munition wesentlich teurer ist, die Abschussvorrichtung ist damit jedoch deutlich einfacher.},
	Schaden={Schlag:$(Stärke)d6$;Schuss:$9(+1d6, baubedingt)$},
	Kosten=20AP, Preis={1 Kupferkristall}, Stufe=3, Size={Normal},
	Supp={Nahschuss: Nakampfangriff mit anschließendem Schuss(ohne Nahkampfmalus)},
	Bild=Waffenbilder/BladeGun.jpg,
	Nachladen={4 Aktionen + 1 Aktion pro Patrone}, Reichweite={3-30m}, Mag={6 Schuss}}%
%
\end{multicols}

\newpage
\section{Fernkampfwaffen}
\begin{multicols}{2}
\raggedcolumns
\addAndPrintItem{Donnerbüchse}{
	Beschreibung={Spezielles Schießpulvermodel der Schrotflinte von menschlichen Waffenschmieden.},
	Schaden={$1d6$(bei jedem Schuss, pro Kugel},
	Kosten={4AP}, Preis={8 Gold}, Stufe=1-8, Size={Groß},
	Nachladen=18 Aktionen, 
	Reichweite={4-10m;bis 1m: 20 Kugeln.;bis 2m: 16+d4 Kugeln.;bis 3m: 12+d8 Kugeln.;bis 4m: 8+d12 Kugeln.;bis 7m: d20 Kugeln.;bis 10m: d12 Kugeln.;bis 12m: d8 Kugeln.;bis 15m: d6 Kugeln.;ab 15m: d4 Kugeln.},
	Spez={Potthässlich: -5 Deko und niemand will so was offen Tragen, vor allem nicht in Siedlungen/Städten;Powerboost: 8 Schaden pro Kugel, jedoch wird das Gewehr beschädigt: 1d4 'Kugeln' treffen den Schützen, jede dieser Kugel fügt der Waffe 2HP Schaden zu.}}%
%
\addAndPrintItem{Bogen}{
	Beschreibung={Jeder Bogen ist für einen bestimmten Stärkebereich(normalerweise X..X*2) konstruiert(ein zu geringer Stärke Einsatz kann keinen Schuss erzeugen, ein zu hoher beschädigt den Bogen)},
	Schaden={;Kurz:$Schaden wie Einhänder(d6)$;Lang:$Schaden wie Zweihänder(d10)$},
	Kosten=4AP, Preis={4-10 Gold}, Stufe=1-8/10/12, Size={Groß},
	Mag={Extern als Köcher.},
	Nachladen={Entfällt},
	Reichweite={4-30m},
	Spez={Vor jedem Schuss muss ein neuer Pfeil aus dem Köcher genommen werden(Supportaktion);Elfenbogen: Kann mit Geschick statt Technik benutzt werden, erhöht die Kosten um 1AP.}}%
%
\addAndPrintItem{Armbrust}{
	Beschreibung={Modelle mit Schnellspanner sind teurer als Modelle mit Kurbelspanner.},
	Schaden={$10(+1d10, baubedingt)$},
	Kosten=4/5AP, Preis={10/12+ Gold}, Stufe=1-12, Size={Groß},
	Nachladen={7/18 Aktionen}, Reichweite={4-30m}}%
%
\addAndPrintItem{Gewehr}{
	Schaden={$20(+1d10, baubedingt)$},
	Kosten=8AP, Preis={28 Gold}, Stufe=2-14, Size={Groß},
	Nachladen={5 Aktionen}, Reichweite={4-60m}}%
%
\addAndPrintItem{Schrotflinte}{
	Schaden={$4(+1d4, baubedingt, pro Kugel)$},
	Kosten=7(9)AP, Preis={24/32 Gold}, Stufe=2-12/13, Size={Normal},
	Nachladen={5 Aktionen + 1 Aktion pro Patrone}, Mag={1-2 Schuss},
	Reichweite={2-10m;bis 2m: 12 Kugeln.;bis 3m: 8+d4 Kugeln.;bis 4m: 6+d6 Kugeln.;bis 5m: 4+d8 Kugeln.;bis 7m: 2+d10 Kugeln.;bis 10m: d12 Kugeln.;bis 13m: d10 Kugeln.;bis 16m: d8 Kugeln.;bis 20m: d6 Kugeln.;ab 20m: d4 Kugeln.}
	}%
%
\addAndPrintItem{Repetier}{
	Schaden={$20(+1d10, baubedingt)$},
	Kosten=15AP, Preis={54 Gold}, Stufe=2-18, Size={Groß},
	Nachladen={5 Aktionen}, Reichweite={5-60m}, Mag={6 Schuss},
	Spez={Durchladen(Supportaktion) und Feuern(Hauptaktion)}}%
%
\addAndPrintItem{Revolver}{
	Schaden={$9(+1d6, baubedingt)$},
	Kosten=15AP, Preis={45 Gold}, Stufe=2-18, Size={Klein},
	Nachladen={2 Aktionen + 1 Aktion pro Patrone}, Reichweite={1-30m}, Mag={6 Schuss}}%

\addAndPrintItem{Magieblaster}{
	Schaden={$10+2d4$},
	Kosten=15AP, Preis={50 Gold(ohne Kristall)}, Stufe=3, Size={Klein},
	Nachladen={20 Aktionen(Kristallwechsel)}, Reichweite={1-15m},
	Mag={Kleiner Magiekristall(normal)}, Verbrauch={Wie Schaden},
	Supp={Powerboost: Der Gesamte Speicher wird vollständig entleert und als Schaden verursacht.}}%

\addAndPrintItem{Magiesniper}{
	Schaden={$15+2d8$},%6*0,66[Gnom]=4*6=24
	Kosten=22AP, Preis={60 Gold(ohne Kristall)}, Stufe=3, Size={Groß},
	Nachladen={25 Aktionen(Kristallwechsel)}, Reichweite={5-60m},
	Mag={Mittlerer Magiekristall(normal)}, Verbrauch={15}}%
\end{multicols}
\newpage
\section{Rüstungen \& Defensives}
Jede Rüstung hat eine bestimmte Kapazität um Inventargegenstände zu verstauen(auch im Kampf zugreifbar). Jede Standard Kleidung hat eine Standardausstattung und eine Maximalausstattung pro Kategorie. Normalerweise kann die Standardausstattung durch Modifikationen verbessert werden(Anpassung der Kleidung entweder selbstständig oder durch geschultes Personal(+1AP/Kategorie um zu Maximieren)).
TODO
\begin{multicols}{2}
\raggedcolumns
\addAndPrintItem{Kleidung}{
	RusRS=0, RusRK={1}, RusVor=0, RusMali={-},
	InvGrAnz=0,
	InvNorAnz=0,	InvNorAnzMax=1,
	InvKleinAnz=0, InvKleinAnzMax=3,
	Kosten=0AP, Stufe=0}%
%
\addAndPrintItem{Leder}{
	RusRS={$\frac{1}{10}$}, RusRK=1, RusVor=0, RusMali={-},
	InvGrAnz=0, InvGrAnzMax=1,
	InvNorAnz=1, InvNorAnzMax=2,
	InvKleinAnz=2, InvKleinAnzMax=4,
	Kosten=4AP, Stufe=1}%
%
\addAndPrintItem{Schweres Leder}{
	RusRS={$\frac{1}{5}$}, RusRK=1, RusVor=1,
	RusMali={Akrobatik,Zauberweben - 1},
	InvGrAnz=1, InvNorAnz=1, InvNorAnzMax=2,
	InvKleinAnz=3, InvKleinAnzMax=6,
	Kosten=5AP, Stufe=1}%
%
\addAndPrintItem{Beschlagenes Leder}{
	RusRS={$\frac{1}{4}$}, RusRK={$1,5$}, RusVor=2,
	RusMali={Ausweichen, Heimlichkeit, Akrobatik, Zauberweben - 1},
	InvGrAnz=1, InvNorAnz=2, InvKleinAnz=3, InvKleinAnzMax=7,
	Kosten=6AP, Stufe=2}%
%
\addAndPrintItem{Kettenhemd}{
	RusRS={$\frac{1}{3}$}, RusRK={$2$}, RusVor=3,
	RusMali={Ausweichen, Heimlichkeit, Akrobatik, Zauberweben - 2},
	InvGrAnz=1, InvGrAnzMax=2, InvNorAnz=2, InvKleinAnz=3, InvKleinAnzMax=7,
	Kosten=8AP, Stufe=2}%
%
\addAndPrintItem{Vollrüstung}{
	RusRS={$\frac{1}{2}$}, RusRK={$2,5$}, RusVor=4,
	RusMali={Geschick - 1, Ausweichen, Heimlichkeit, Akrobatik, Zauberweben - 2},
	InvGrAnz=2, InvNorAnz=2, InvKleinAnz=4, InvKleinAnzMax=9,
	Kosten=15AP, Stufe=2}%
%
\addAndPrintItem{Arglardor-Rüstung}{
	RusRS={$\frac{2}{3}$}, RusRK={$3$}, RusVor=6,
	RusMali={Geschick - 1, Ausweichen, Heimlichkeit, Akrobatik, Zauberweben - 2, Zaubern=0, Mag.Macht=0},
	InvGrAnz=2, InvGrAnzMax=3,
	InvNorAnz=2, InvNorAnzMax=3,
	InvKleinAnz=5, InvKleinAnzMax=10,
	Kosten=50AP, Stufe=3,
	Spez= {Magieimmunität: Träger kann nicht direkt von Magie beeinflusst werden und auch selbst keine Magie wirken (inklusive sämtlicher Magietech).}	}%
%
\end{multicols}

\chapter{Gegenstandstände}
\section{Schilde}
\begin{tabulary}{\textwidth}{|C|C|C|C|C|}
\hline 
&Wirkung&Größe&Kosten AP &Kosten Geld\\ 
\hline
Kleiner Schild& Abwehrspezialisierung+0 &
eigene Größe-3 Stufen	&3 AP&?\\
\hline 
Mittlerer Schild& Abwehrspezialisierung+4\& Angriffsspezialisierung-1 &	
eigene Größe	-2&4 AP&?\\
\hline 
Großer Schild& Abwehrspezialisierung+8\& Angriffsspezialisierung-2 &	
eigene Größe	-1&5 AP&?\\
\hline 
Riesen Schild& Abwehrspezialisierung+12\& Angriffsspezialisierung-4 &
eigene Größe	&
6 AP&?\\
\hline 
\end{tabulary}
\section{Gebrauchsgegenstände}
\begin{tabulary}{\textwidth}{|C|C|C|C|C|}
\hline 
Name&Wirkung&Größe&Kosten AP &Kosten Geld\\ 
\hline
Arglardormamulett& 
Mag.Macht \underline{Wert} wird um X reduziert, bei Magie von dir oder die auf dich wirkt. &
Kleidung (klein)	&? AP&?\\
\hline
Magie Unterdrücker(Handschellen oder Armband)& 
Die Magie des Trägers wird durch den eingebauten 'Magie auflösen' Zauber komplett unterdrückt. &
Kleidung (klein)	&? AP&?\\
\hline
Kleiner Rucksack& 
+5 Plätze für kleine Gegenstände &
Kleidung	&1 AP&?\\
\hline
Rucksack& 
+10 Plätze für kleine Gegenstände &
Kleidung	&2 AP&?\\
\hline
Wanderrucksack& 
+16 Plätze für kleine Gegenstände \& +2 Plätze für normale Gegenstände &
Kleidung	&4 AP&?\\
\hline
Kleiner Gegenstand& 
z.b. Uhr, Topfpflanze, ...&
klein?	&0-2 AP&?\\
\hline 
Basis Werkzeugset& 
Für Basisbedarf(nur Reparatur begrenzte Anwendbarkeit) z.b. Nadel und Faden&
klein	&1 AP&?\\
\hline 
Standard Werkzeugset& 
Für die meisten Aufgaben nutzbar(begrenzte Anwendbarkeit) z.b. Hammer, Meißel, Zange\& Co., bzw. Dietrichset&
klein (1-3x)	&2 AP&?\\
\hline 
Erweitertes Werkzeugset& 
Für spezialisierte Aufgaben nutzbar(Herstellung o. universelle Anwendbarkeit) z.b. ?&
klein (1-3x) + Normal (0-1x)	&5 AP&?\\
\hline 
Reisender Paket& 
Standard Reiseausrüstung, beinhaltet Zelt/Schlafutensilien, einfaches Besteck/Geschirr, Feuerzeug(bzw. Streichhölzer o.ä.)\& Co., Wasserschlauch\& Proviantbeutel für je 3 Tage.&
klein (3x) + Normal (1x)	&3 AP&?\\
\hline 
Abenteurer Paket& 
Erweitert Reiseausrüstung um für Abenteurer relevante Ausrüstung dies beinhaltet Kletterseil mit Wurfhaken(10m), Fackel, Messer\& Beil, +2 Tage Nahrung\& Wasser, erste Hilfe Set(Verbände,..).&
klein (5x) + Normal (1x)	&5 AP&?\\
\hline 
nimmerleerer Munitionsbeutel& 
Beutel mit einem endlosen Vorrat an Munition(Patronen, Gasflaschen, ...) zum nachladen(Verbindung zu einem Munitionslager über Beschwörungsmagie(Dauerportal), Munitionskosten werden dem Besitzer in Rechnung gestellt.).
 Gilt nur für 1 Munitionsart(z.b. Revolverkugeln).
 Munitionskosten werden über Erhaltungskosten simuliert.&
Je nach Munition&In der Waffe enthalten&50 Gold\\
\hline 
\end{tabulary}
\chapter{Zauber TODO}
\label{Zauberliste}
\section{Illusion}
\addAndPrintSpell{Licht}{
	Stufe=0,
	Beschreibung={Verzaubert einen Gegenstand, sodass er leuchtet.},
	MagDauer={Spontan},
	MagVerstand={
		;$\xrightarrow{2}$: Licht wird als eigenständiges Kugelförmiges Objekt geformt.},
	MagMacht={
		;$\xrightarrow{1}$: Licht hält für (+) 1 min.
		;$\xrightarrow{1}$: Licht erzeugt (+) 5m Leuchtradius.
		;$\xrightarrow{3}$: Startet mit einem Lichtblitz, der alle blendet, die hinsehen(2 Runden lang Treffer um 1 erschwert).}
}

\addAndPrintSpell{WobelBobelBoden}{
	Stufe=1,
	Beschreibung={Der Boden scheint sich ständig zu bewegen(wie Wackelpudding).},
	MagVerstand={
		;$\xrightarrow{Initial}$: 5m radius mit dem Magier als Zentrum, Geländeschwierigkeit +1
		;$\xrightarrow{1}$: +5m radius
		;$\xrightarrow{1}$: Zentrum kann bis zu 5m vom Magier entfernt sein(gilt als Zauberreichweite)
		;$\xrightarrow{3}$: Geländeschwierigkeit nochmals +1.
		},
}

\addAndPrintSpell{Illusion}{
	Stufe=1,
	Beschreibung={Erzeugt eine einzelne Illusion, eines Objekts, in bis zu 5m Abstand zum Zauberer, die alle Anwesenden beeinflusst.;
	Für Kampfsituationen gilt ein Kritischer Treffer sind 2 unpassende Handlungen.},
	MagVerstand={
		;$\xrightarrow{0}$: Illusion hat sehr schlechte Qualität, aussehen der Illusion zufällig(Elf, Mensch, Monster, Truhe, Schwert, ...), keine Anpassung an die Umgebung(jede Aktion gilt als unpassende Handlung).
		;$\xrightarrow{1}$: Illusion hat schlechte Qualität, nur die Basisart der Illusion ist bestimmbar(Humanoid, Monster, Gegenstand, Pflanze, ...), Anpassung immer noch schlecht, jedoch gut genug, dass die Illusion aktiv beobachtet werden muss um die Abnormalitäten zu erkennen.
		;$\xrightarrow{2}$: Illusion hat normale Qualität(wählbare Rassen (Mensch, Elf, Goblin, Troll, ...), Anpassung an die statische Umgebung(nicht an bewegte Objekte/Personen).
		;$\xrightarrow{3}$: Illusion hat gute Qualität(die eigenen Vorstellungen werden weitestgehend umgesetzt), Anpassung an alles außer Kampfsituationen(wenn Illusion 'getroffen' würde(1 Angriffserfolg)).
		;$\xrightarrow{4}$: Illusion hat gute Qualität \& kann sich an Duelle anpassen( mindestens 2 Angreifer o. Treffer(Ausweicherfolge = 4) gelten als unpassende Handlung).
		;$\xrightarrow{7}$: Illusion hat ausgezeichnete Qualität, komplexe Kampfsituationen möglich(nur ein Treffer(Ausweicherfolge = 11 (-2 pro Angreifer)) gilt als unpassende Handlung).
		;$\xrightarrow{1}$: Ausweicherfolge + 1.
		;$\xrightarrow{2}$: Illusion führt Ausweichkonteraktionen aus wenn möglich.
		},
}

\addAndPrintSpell{Verändernde Illusion}{
	Stufe=2,
	Beschreibung={Erzeugt eine Illusion, die ein Objekt verändert(Unsichtbarkeit, Versetzter Arm, scheinbare Verwandlung).},
	MagVerstand={
		;$\xrightarrow{0}$: 'Verpixelungseffekte'.
		;$\xrightarrow{1}$: Extra Bein/Ohr/Kopf/...(kein Arm).
		;$\xrightarrow{2}$: Extra Arm.
		;$\xrightarrow{3}$: Teilweise Unsichtbarkeit/Teilverwandlung.
		;$\xrightarrow{4}$: Schein Körperteil(Illusion eines Körperteils und Tarnung des echten).
		;$\xrightarrow{5}$: Schein Verwandlung.
		;$\xrightarrow{6}$: Vollständige Unsichtbarkeit.
		},
}

\addAndPrintSpell{Imitation}{
	Stufe=2,
	Beschreibung={Ändert Körperliche Eigenschaften(Aussehen, Stimme, ...) in die der Zielperson(du musst diese schon einmal gesehen/gehört haben).},
	MagVerstand={
		;$\xrightarrow{0}$: Unter einem Umhang wirkst du genau wie die Zielperson ... wenn du nichts sagen musst ... und schon eine ähnliche Größe hattest.
		;$\xrightarrow{1}$: Gesichtsform passt.
		;$\xrightarrow{1}$: Stimme passt.
		;$\xrightarrow{2}$: Körperform passt.
		;$\xrightarrow{3}$: Perfekte Imitation.
		;$\xrightarrow{2}$: Größenanpassung des Körpers um 1 Größenstufe.
		},
}

\section{Angriff}
\addAndPrintSpell{Angriff}{
	Stufe={1},
	Beschreibung={Mag Geschoss o.ä..},
}
\addAndPrintSpell{Magie auflösen}{
	Stufe={2},
	Beschreibung={Vernichtet die Zaubermatrix eines temporären Zaubers.;
	Wenn die Ziel Zaubermatrix vor weniger als 10 Minuten entstanden ist muss außerdem die Magieverteidigung überwunden werden.},
	MagVerstand={
		;$\xrightarrow{0}$: löst einen Stufe 0 Zauber auf.
		;$\xrightarrow{1}$: löst einen Stufe 1 Zauber auf.
		;$\xrightarrow{2}$: löst einen Stufe 2 Zauber auf.
		;$\xrightarrow{4}$: löst einen Stufe 3 Zauber auf.
		;$\xrightarrow{8}$: löst einen Stufe 4 Zauber auf.
		;$\xrightarrow{15}$: löst einen Stufe 5 Zauber auf.
		},
	MagMacht={Jeder Schadenspunkt erhöht die benötigte Erhaltungsprobe um 1 Erfolg.},
}
\addAndPrintSpell{Explosion}{
	Stufe={2},
	MagVerstand={
		;$\xrightarrow{Initial}$: Ausweichen erzeugt immer mindestens einen leichten Treffer wenn der Bereich(2m Radius) nicht komplett verlassen wird.},
	MagMacht={
		;$\xrightarrow{Initial}$: Durchmesser 1 m.
		;$\xrightarrow{1}$: Radius + 1 m.},
}
\addAndPrintSpell{Direkter Schmerz}{
	Stufe={2},
	Beschreibung={Angriffsmagie, umgeht Rüstung u. Verteidigung(Ausweichen/Schild), nicht tödlicher Schaden. Die Angriffswert Verbesserungen können für diesen Zauber nicht verwendet werden.},
	MagVerstand={
		;$\xrightarrow{Initial}$: Angriff umgeht Rüstung und normale Verteidigungsmaßnahmen(immer Basisschaden).},
}

\section{Verstärkung}
\addAndPrintSpell{Reinigung}{
	Stufe=0,
	Beschreibung={Säubert einen Gegenstand, indem die Schmutzabweisende Eigenschaft verstärkt wird. Der Dreck verschwindet nicht einfach sondern wird einfach ab/herausfallen.},
	MagDauer={Spontan},
	MagVerstand={
		;$\xrightarrow{0}$:Größe: Kleiner Gegenstand (oder einfache Kleidung)
		;$\xrightarrow{1}$:Größe: Normal Gegenstand (oder Lederrüstung)
		;$\xrightarrow{2}$:Größe: Großer Gegenstand (oder Metallrüstung)
		;$\xrightarrow{3+X}$:Größe: Riesiger Gegenstand($\leq(250*2^X)\%$ der Körpergröße)
		},
	MagMacht={
		;$\xrightarrow{1}$: Schmutz wird um +25\% reduziert.
		;$\xrightarrow{1}$: wenn vollständig gereinigt: +1 Deko für 8h.
		}
}
\addAndPrintSpell{Schild(Magie)}{
	Stufe=1,
	Beschreibung={Verstärkt den natürlichen magischen Schild eines Wesens/Objekts.},
	MagDauer={Spontan},
	MagVerstand={
		;$\xrightarrow{0}$: Schild wirkt nur gegen Zauber der Stufe 0-1.
		;$\xrightarrow{2}$: Schild wirkt nur gegen Zauber der Stufe 0-2.
		;$\xrightarrow{4}$: Schild wirkt nur gegen Zauber der Stufe 0-3.
		;$\xrightarrow{6}$: Schild wirkt gegen alle Zauber.
		;$\xrightarrow{2}$: Auch wenn das Objekt/Wesen keinen natürlichen Schild besitzt(magische Macht=0)},
	MagMacht={Der Schild verstärkt den magischen Schild soweit, dass entweder der Zauber gerade so neutralisiert wird(Schilderfolge>Zaubererfolge) oder der Schild verbraucht ist. Wenn noch Erfolge übrigbleiben werden sie auf die nächste Schildprobe übertragen.
		;$\xrightarrow{1}$: +1 magischer Erfolg gegen Zauber der Stufe 0-1.
		;$\xrightarrow{2}$: +1 magischer Erfolg gegen beliebige Zauber.
		},
}
\addAndPrintSpell{Heilung}{
	Stufe=1,
	Beschreibung={Verstärkt die natürliche Regeneration eines Lebewesens, sodass sich Wunden innerhalb von Sekunden schließen und heilen.},
	MagDauer={Spontan},
	MagMacht={
		;$\xrightarrow{1}$: +5 Heilungswürfel(d1) für den Zauber.
		},
}
\addAndPrintSpell{Schild(Schaden)}{
	Stufe=2,
	Beschreibung={Verstärkt die Haut/Oberfläche eines Wesens/Objekts, damit diese eine gewisse Schadensmenge (zusätzlich, insgesamt) absorbiert.},
	MagDauer={Spontan},
	MagVerstand={
		;$\xrightarrow{Initial}$: Schild wirkt gegen eine Schadensquelle(Physisch, Feuer, Eis, ...).
		;$\xrightarrow{2}$: Schild wirkt gegen eine weitere Schadensquelle.
		;$\xrightarrow{6}$: Schild wirkt gegen jede Schadensquelle.
		},
	MagMacht={
		;$\xrightarrow{1}$: +5 Schutzwürfel(d1) für den Zauber.
		},
}
\addAndPrintSpell{Schutzschild}{
	Stufe=2,
	Beschreibung={Erzeugt einen unsichtbaren Schutzschild um den Zauberer(oder einen nahen Verbündeten(max. 5 Meter entfernt am ende des Zuges des Zauberers)). Der Schild bricht zusammen wenn sich das Ziel aus der Zauberreichweite entfernt oder der Schild auf 0 Schutzpunkte fällt. Der Schild startet mit seiner vollen Kapazität und regeneriert bei jeder erfolgreichen Erhaltungsprobe Punkte.},
	MagDauer={Temporär},
	MagVerstand={
		;$\xrightarrow{Initial}$: Schild wirkt gegen eine Schadensquelle(Physisch, Feuer, Eis, ...).
		;$\xrightarrow{2}$: Schild wirkt gegen eine weitere Schadensquelle.
		;$\xrightarrow{6}$: Schild wirkt gegen jede Schadensquelle.
		},
	MagMacht={
		;$\xrightarrow{1}$: Schild regeneriert +1 Punkt pro Runde, Schildkapazität +2.
		;$\xrightarrow{1}$: Schildkapazität +5.
		;$\xrightarrow{1}$: Schild regeneriert +2 Punkt pro Runde.
		},
}
\addAndPrintSpell{Reparatur}{
	Stufe=3,
	Beschreibung={Erzeugt regenerative Eigenschaften bei einem Gegenstand. Jeder Zaubermisserfolg generiert einen Gegenstandschadenseffekt, der eine Haltbarkeitsprobe gegen $3+Zaubermachterfolge$ nach sich zieht.},
	MagDauer={Spontan},
	MagVerstand={
		;$\xrightarrow{0}$:Größe: Kleiner Gegenstand (oder einfache Kleidung)
		;$\xrightarrow{1}$:Größe: Normal Gegenstand (oder Lederrüstung)
		;$\xrightarrow{2}$:Größe: Großer Gegenstand (oder Metallrüstung)
		;$\xrightarrow{3+X}$:Größe: Riesiger Gegenstand($\leq(250*2^X)\%$ der Körpergröße)
		;$\xrightarrow{0}$:Beschädigung: leicht(bis zu 4 HP Schaden)
		;$\xrightarrow{2}$:Beschädigung: mittel(bis zu 8 HP Schaden)
		;$\xrightarrow{4}$:Beschädigung: katastrophal(bis zu 10 HP Schaden)
		},
	MagMacht={
		;$\xrightarrow{1}$: 1 HP wird regeneriert.
		;$\xrightarrow{2}$: wenn vollständig repariert(10 HP): +1 HP.
		}
}

\section{Verwandlung}
\addAndPrintSpell{Telekinese}{
	Stufe=0,
	Beschreibung={Verändert die Umgebung eines Objekts, um dieses zu bewegen(über den Boden oder andere feste Oberflächen gleiten zu lassen).}
	MagDauer={Spontan},
	MagVerstand={
		;$\xrightarrow{Initial}$: Max. 1 Pfund Gewicht, kann sich nur entlang von festen Gegenständen bewegen.
		;$\xrightarrow{1}$: +1 Pfund maximales Gewicht
		;$\xrightarrow{1}$: Objekt kann für maximal (+) 1 KR auch innerhalb von Flüssigkeiten beliebig bewegt werden(Bewegungsradius ist halbiert).
		;$\xrightarrow{4}$: Objekt kann für maximal (+) 1 KR beliebig bewegt werden(z.b. schweben)(Bewegungsradius ist halbiert).
		},
	MagMacht={
		;$\xrightarrow{1}$: +1 Meter Bewegungsradius in der ersten KR.
		;$\xrightarrow{3}$: +1 Meter Bewegungsradius in der ersten KR ohne Bodenkontakt(als Sprung/Wurf).
		;$\xrightarrow{1}$: +2 Meter Bewegungsradius (0,5m pro KR).
		}
}

\addAndPrintSpell{Rückverwandlung}{
	Stufe=selbst:1;andere:2,
	Beschreibung={Erlaubt es Sachen in ihre Ursprüngliche Form zurückzuverwandeln, dabei müssen alle  Magische Macht Effekte der Verwandlung rückgängig gemacht werden(selbe Erfolgsmenge wie bei der Verwandlung)},
	MagDauer={Spontan},
	MagVerstand={
		;$\xrightarrow{Initial}$: $<\frac{1}{2}h$ Verwandlungsdauer.
		;$\xrightarrow{1}$: $<1h$ Verwandlungsdauer.
		;$\xrightarrow{2}$: $<2h$ Verwandlungsdauer.
		;$\xrightarrow{3}$: $<4h$ Verwandlungsdauer.
		;$\xrightarrow{1}$: $+4h$ Verwandlungsdauer.
		;$\xrightarrow{9}$: $\geq24h$ Verwandlungsdauer.
		},
}

\addAndPrintSpell{Verwandlung Personen und Kreaturen}{
	Stufe={selbst:2;andere:3},
	Beschreibung={Du kannst dich(bzw. andere) verwandeln, allerdings hast du kaum Kontrolle darüber was dabei passiert...;
Bei Verwandlungen in Tiere ändern sich nur körperliche Attribute, die Verwandlung in andere Reiche, zum oder vom Unbelebten, oder in magische Wesen ändert jedoch auch die Seele des Verwandelten.;
Geistige Attribute(und mit ihnen auch die Persönlichkeit, ...) werden in jedem Fall nach 4, 12 und 24 Stunden angepasst. Nach 24h hat man sich vollständig in das andere Wesen verwandelt und ist sich nicht mehr bewusst darüber mal was anderes gewesen zu sein.},
	MagVerstand={Gezieltere Verwandlung:
		;$\xrightarrow{1}$: Klasse(z.b. Säugetier/Vogel/...)
		;$\xrightarrow{2}$: Ordnung(z.b. Raubtier/Primat/...)
		;$\xrightarrow{3}$: Familie(z.b. Katzen/Hunde/Humanoide/...)
		;$\xrightarrow{4}$: Gattung(Spezies wie Mensch/Tiger/...)
		;$\xrightarrow{5}$: Ein spezielles Wesen(z.b. Eronso Mümmel von nebenan oder seine Katze) 
		;$\xrightarrow{5}$: Verwandlung in ein anders Reich(z.b. Dämon/Pflanze/Geist/Untoter/...)
		;$\xrightarrow{6}$: Verwandlung in ein spezifisches Reich/Familie
		;$\xrightarrow{7}$: $\rightarrow$Unbelebtes(weiteres siehe \ref{unlebendVerwandlung})
		;$\xrightarrow{7}$: Verwandlung in ein Magische Wesen(Elementare/Greifen/Drachen/Schleim/...)
		;$\xrightarrow{8}$: Verwandlung in ein bestimmtes Magisches Wesen(Familie z.b. Elementar)
		;$\xrightarrow{3}$: Gezielte Teilverwandlung(z.b. Die Flügel eines Vogels wachsen lassen.)(+ Basiskosten des ursprünglichen Ziels z.b. Vogel)
		;$\xrightarrow{??}$: Spezialisierte Verwandlungen??},
}\label{lebendVerwandlung}

\addAndPrintSpell{Verwandlung Gegenstände}{
	Stufe=1,
	Beschreibung={Du kannst Gegenstände verwandeln, die du in den Händen hältst/berührst.},
	MagVerstand={
		;$\xrightarrow{0}$: Größe: Kleiner Gegenstand
		;$\xrightarrow{1}$: Größe: Normal Gegenstand
		;$\xrightarrow{3}$: Größe: Großer Gegenstand
		;$\xrightarrow{5+X}$: Größe: Riesiger Gegenstand($\leq(250*2^X)\%$ der Körpergröße)
		;$\xrightarrow{0}$: Ergebnis: Typ Äquivalent(Schwert$\rightarrow$Schwert)
		;$\xrightarrow{1}$: Ergebnis: Übertyp Äquivalent(Waffe$\rightarrow$Waffe)
		;$\xrightarrow{4}$: Ergebnis: beliebiger Typ (Waffe$\rightarrow$Stein)
		;$\xrightarrow{0}$: Ergebnis: gleiches Material
		;$\xrightarrow{6}$: Ergebnis: beliebiges Material
		;$\xrightarrow{4}$: Ergebnis: Material wird 'lebendig'(könnte etwas fühlen/tun)
		;$\xrightarrow{1}$: Ergebnis('lebendiges' Material): +1 Empfindungsart
		;$\xrightarrow{2}$: Ergebnis('lebendiges' Material): +1 Aktionsmöglichkeit
		;$\xrightarrow{5}$: Ergebnis: $\rightarrow$Lebendes(weiteres siehe \ref{lebendVerwandlung})},
	MagMacht={TS = Technikstufe des Zielgegenstands(Maximal Stufe 1).
		;$\xrightarrow{TS+1}$: 1 Qualitätsstufe umverteilen
		;$\xrightarrow{2*(TS+1)}$: $\pm$ 1 Qualitätsstufe
		;$\xrightarrow{?}$: Materialänderungskosten je nach Veränderung.
		;$\xrightarrow{1}$: 1 Empfindungsart wird sensitiver(z.b. taub$\rightarrow$schwerhörig$\rightarrow$...).
		;$\xrightarrow{2}$: 1 Aktionsmöglichkeit wird stärker(z.b. stumm$\rightarrow$flüstern$\rightarrow$...).
		;$\xrightarrow{??}$: TODO: ....}
}\label{unlebendVerwandlung}

\section{Elementar}
\addAndPrintSpell{Einfache Elementarmanipulation}{
	Stufe=0,
	Beschreibung={Du kannst das gewählte Element beeinflussen(in einem Kubus von 10cm Länge).},
	MagDauer={Spontan},
	MagVerstand={
		;$\xrightarrow{1}$: Kubusgröße * 2},
	MagMacht={
		;$\xrightarrow{1}$: Basiseffekt wird verstärkt oder geschwächt(Wärme bei Feuer,...)(Faktor 1,5)
		;$\xrightarrow{2}$: Element wird verstärkt oder geschwächt(Feuer wird größer,...)(Faktor 1,5)},
}\label{Einfache Elementarmanipulation}

\addAndPrintSpell{Erzeugung von Elementen}{
	Stufe=1,
	Beschreibung={Wie \ref{Einfache Elementarmanipulation} nur dass das Element nicht vorher dagewesen sein muss(wird einfach erschaffen).},
	MagDauer={Spontan bis Temporär},
	MagMacht={
		;$\xrightarrow{1}$: Element wird erzeugt(Streichholzflamme bei Feuer,...)
		;$\xrightarrow{2}$: Element wird erzeugt ohne Basismaterial(Feuer ohne brennbares, ...)
		;$\xrightarrow{1}$: Erzeugtes Element wird verstärkt(Faktor 1,5)
		},
}
\addAndPrintSpell{Elementarangriff}{
	Stufe=2,
	Beschreibung={Wie Angriffszauber, aber Schadenswürfel beginnen bei d4 statt d1 Schwierigkeit ist einfach bis meisterlich, außerdem können Zusatzeffekte gekauft werden(abhängig vom Element).},
}
\addAndPrintSpell{weitere Elementarzauber}{
	Stufe=0-5,
	Beschreibung={Kreativität des Spielers/Spielleiters.},
}

\section{Beschwörung}
\addAndPrintSpell{weitere Beschwörungen}{
	Stufe=0-5,
	Beschreibung={Kreativität des Spielers/Spielleiters.},
}
\begin{TODO}!!!TODO!!!
->Zufällige vs gezielte??
BSP:

->Verstand: ...
-> Macht: ...

St0:
Geisterbeschwörung/Gegenstandsbelebung

St1:
Beschwörung niederer(Diener/Underlinge)
Zurückschicken von Kreaturen unter Kontrolle
leichte Beeinflussung/Gedankenlesen

St2:
Beschwörung mittlerer(normal)


St3:
Beschwörung höherer('General' z.b. Lich/Zombigeneral/...)
->e.v. unkontroliert
Zurückschicken von unkontrollierten Beschwörungen
Gedankenkontrolle
Tor

-> Beschwörung/Verbannung von Menschen/...??
\end{TODO}

\chapter{Manöver}
\label{Manöverliste}
\section{Finte}
Schwierigkeit Angriff und Verteidigung erhöht sich um 1.

\section{Heftiger Schlag}
 Schwierigkeit für Angriff und die eigene Verteidigung +1, Stärke wird für die Schadensberechnung verdoppelt.

\section{Rundum Schlag} Schwierigkeit +1, trifft alle in Nahkampfreichweite der Reihe nach. Für jeden Treffer wird der Schaden neu ausgewürfelt und für alle weiteren Treffer die Stärke um 1 Punkt reduziert. Endet wenn einem Ziel eine (Schild)Abwehrprobe gelingt oder die Reststärke 0 beträgt.

\section{Überraschungsangriff} (nicht vorher wahrgenommen) Schwierigkeit + 1 für Verteidigungsaktionen.

\section{Schleichangriff} (nicht wahrgenommen\& nur mit geräuschlosen/leisen Waffen) keine Verteidigungsmöglichkeit für Verteidiger.

\section{(Angriff) vorbereiten:} (Haupt+Supportaktion) Kämpfer bereitet sich auf eine bestimmte Aktion des Gegners vor und hat - 1 auf seine nächste Aktionsschwierigkeit wenn sein Plan aufgeht.


\end{Beispiele}
\begin{characterSheets}
\makeatletter

\newcommand{\CharacterSheetDefault}{
\setkeys{CharacterSheet}{LPMax,MPMax,MPDesc={MP:\hfill}}
\setkeys{CharacterSheet}{St=2,Ge=2,Ko=2,Wa=2,Ch=2,AW=2,TW=2,TG=2,MV=2,MM=2}
\setkeys{CharacterSheet}{StKr=0,StAb=0,StAk=0,
							GeNa=0,GeAu=0,GeHe=0,
							KoZa=0,KoAu=0,KoWi=0,
							WaSu=0,WaWa=0,WaSc=0,
							ChRh=0,ChEi=0,ChVe=0,ChMe=0,
							AWMo=0,AWGe=0,AWAk=0,AWUn=0,AWFr=0,
							TWMe=0,TWCh=0,TWDr=0,TWMa=NA,TWEl=NA,
							TGSc=0,TGAn=0,TGRu=0,TGSo=0,
							MVAn=0,MVIl=0,MVVS=0,MVVW=0,MVEl=NA,MVBe=NA,
							MMAn=0,MMIl=0,MMVS=0,MMVW=0,MMEl=NA,MMBe=NA,MMSh=0}
\setkeys{CharacterSheet}{Titel, Name, Rasse, Alter, Gewicht, Size,
				Eigenschaften,
				Beschreibung,
				Talente,
				Gold=0, Silber=0, Kupfer=0,
				InvGrAnz=0, InvGr, InvNorAnz=0, InvNor, InvKleinAnz=0, InvKlein, InvSonst,
				Zauber,
				Rus=Kleidung, RusRS=0, RusRK=0,RusMali,RusVor=0}
}

\newcommand{\CharacterSheetUnknown}{
\setkeys{CharacterSheet}{LPMax,MPMax,MPDesc}
\setkeys{CharacterSheet}{St=??,Ge=??,Ko=??,Wa=??,Ch=??,AW=??,TW=??,TG=??,MV=??,MM=??}
\setkeys{CharacterSheet}{StKr=??,StAb=??,StAk=??,
							GeNa=??,GeAu=??,GeHe=??,
							KoZa=??,KoAu=??,KoWi=??,
							WaSu=??,WaWa=??,WaSc=??,
							ChRh=??,ChEi=??,ChVe=??,ChMe=??,
							AWMo=??,AWGe=??,AWAk=??,AWUn=??,AWFr=??,
							TWMe=??,TWCh=??,TWDr=??,TWMa=??,TWEl=??,
							TGSc=??,TGAn=??,TGRu=??,TGSo=??,
							MVAn=??,MVIl=??,MVVS=??,MVVW=??,MVEl=??,MVBe=??,
							MMAn=??,MMIl=??,MMVS=??,MMVW=??,MMEl=??,MMBe=??,MMSh=??}
\setkeys{CharacterSheet}{Titel, Name, Rasse, Alter, Gewicht, Size,
				Eigenschaften,
				Beschreibung,
				Talente,
				Gold, Silber, Kupfer,
				InvGrAnz, InvGr, InvNorAnz, InvNor, InvKleinAnz, InvKlein, InvSonst,
				Zauber,
				Rus, RusRS, RusRK,RusMali,RusVor}
}

\newcommandx{\CharSetup}[1]{\setkeys{CharacterSheet}{#1}}

\newcommandx{\CharSetupSt}[4][2=0, 3=0, 4=0]{\setkeys{CharacterSheet}{St=#1, StKr=#2, StAb=#3, StAk=#4}}
\newcommandx{\CharSetupGe}[4][2=0, 3=0, 4=0]{\setkeys{CharacterSheet}{Ge=#1, GeNa=#2, GeAu=#3, GeHe=#4}}
\newcommandx{\CharSetupKo}[4][2=0, 3=0, 4=0]{\setkeys{CharacterSheet}{Ko=#1, KoZa=#2, KoAu=#3, KoWi=#4}}
\newcommandx{\CharSetupWa}[4][2=0, 3=0, 4=0]{\setkeys{CharacterSheet}{Wa=#1, WaSu=#2, WaWa=#3, WaSc=#4}}
\newcommandx{\CharSetupCh}[5][2=0, 3=0, 4=0, 5=0]{\setkeys{CharacterSheet}{Ch=#1, ChRh=#2, ChEi=#3, ChVe=#4, ChMe=#5}}
\newcommandx{\CharSetupAW}[6][2=0, 3=0, 4=0, 5=0, 6=0]{\setkeys{CharacterSheet}{AW=#1, AWMo=#2, AWGe=#3, AWAk=#4, AWUn=#5, AWFr=#6}}
\newcommandx{\CharSetupTW}[6][2=0, 3=0, 4=0, 5=0, 6=0]{\setkeys{CharacterSheet}{TW=#1, TWMe=#2, TWCh=#3, TWDr=#4, TWMa=#5, TWEl=#6}}
\newcommandx{\CharSetupTG}[5][2=0, 3=0, 4=0, 5=0]{\setkeys{CharacterSheet}{TG=#1, TGSc=#2, TGAn=#3, TGRu=#4, TGSo=#5}}
\newcommandx{\CharSetupMV}[7][2=0, 3=0, 4=0, 5=0, 6=0, 7=0]{\setkeys{CharacterSheet}{MV=#1, MVAn=#2, MVIl=#3, MVVS=#4, MVVW=#5, MVEl=#6, MVBe=#7}}
\newcommandx{\CharSetupMM}[7][2=0, 3=0, 4=0, 5=0, 6=0, 7=0]{\setkeys{CharacterSheet}{MM=#1, MMAn=#2, MMIl=#3, MMVS=#4, MMVW=#5, MMEl=#6, MMBe=#7}}


\makeatother

\part{Monster}
\newcommand{\tablistcommand}{% <-- for eliminating vertical space
                             %     before and after itemize
            \leavevmode\par\vspace{-\baselineskip}
                            }
\CharacterSheetUnknown
\CharSetupSt{100}[100][600][100]
\CharSetupGe{30}[180]
\CharSetupKo{100}[100]
\CharSetupWa{30}
\CharSetupCh{3}[0][618]
\CharSetupMM{100}[??][??][??][??][??][??]
\CharacterSheet{Titel={Monster Blatt}, Rasse={H-Hohn}, Size={>100m}, Gewicht={unbekannt},
				Eigenschaften={frisst Drachen},
				Beschreibung={\begin{normalsize}\begin{itemize}[leftmargin=0cm, noitemsep, topsep=0pt, before = \tablistcommand, after = \tablistcommand]
					\item Riesiges (Killer) Huhn
					\item nahezu ausgerottet seit dem 20. Jahrhundert(große Drachen Rache Expedition)
				\end{itemize}\end{normalsize}},
				Talente={Mehrfachangriff, Flächenangriff, ??}
}

\CharacterSheetUnknown
\CharSetupSt{2}[4][4][2]
\CharSetupGe{3}[6][2][7]
\CharSetupKo{3}[5][3][1]
\CharSetupWa{2}[2][2][0]
\CharacterSheet{Titel={Junger Mantikor}, Rasse={Mantikor}, Size={2,5m}, Gewicht={?},
				Eigenschaften={Greift immer den an der in der letzten runde am meisten Schaden verursacht hat.},
				Beschreibung={},
				Talente={Doppelschlag(Klauen, 1h-waffe), Giftstachel alle 2 Runden},
				InvGr={Giftstachel(3d12, Lähmendes Gift(+1d4 Giftpunkte bei Treffer, mittlere Zähigkeitsprobe gegen alle Giftpunkte oder gelähmt für die Differenz in Runden))},
				InvNor={Klauen(3d6)}
}
\CharacterSheetUnknown
\CharSetupSt{4}[6][6][3]
\CharSetupGe{4}[9][3][5]
\CharSetupKo{4}[6][4][3]
\CharSetupWa{4}[4][4][1]
\CharacterSheet{Titel={Erwachsener Mantikor}, Rasse={Mantikor}, Size={4m}, Gewicht={?},
				Eigenschaften={},
				Beschreibung={},
				Talente={Doppelschlag(Klauen), Giftstachel alle 2 Runden},
				InvGr={Giftstachel(5d12, Lähmendes Gift(+1d8 Giftpunkte bei Treffer, mittlere Zähigkeitsprobe gegen alle Giftpunkte oder gelähmt für die Differenz in Runden))},
				InvNor={Klauen(5d6)}
}


\part{NPCs}
\CharacterSheetDefault
\GenerateCharacterRus{Vollrüstung}
\CharacterSheet{Ge=3, GeNa=6, St=3, StAb=2, Ko=3, KoAu=2, KoZa=2, KoWi=5, Ch=2, ChMe=3, Wa=2, WaWa=2, AWAk=2, TG=2, TGRu=4, MM=1, Name=Kommandant, Rasse=Mensch, Alter=50, Size={1,76m}, Gewicht=75Kg, InvNor={Schwert(Schaden 3d6)}, InvGr={Großer Schild}, LPMax=42}

\CharacterSheetDefault
\GenerateCharacterRus{Kettenhemd}
\CharacterSheet{Ge=2, GeNa=1, St=3, StAb=1, Ko=2, KoAu=2, KoZa=1, KoWi=2, Ch=1, ChMe=2, Wa=2, WaWa=3, AWAk=1, TG=3, TGSc=3, TGRu=2, MM=1, Name=Wache 1, Rasse=Mensch, Alter=23, Size={1,82m}, Gewicht=84Kg, InvGr={Langbogen(Schaden 2d12)}, InvNor={Schwert(Schaden 3d6)}, LPMax=33}

\CharacterSheetDefault
\GenerateCharacterRus{Kettenhemd}
\CharacterSheet{Ge=4, GeNa=4, St=4, StAb=2, Ko=4, KoAu=3, KoZa=3, KoWi=3, Ch=1, ChMe=2, Wa=2, WaWa=1, AWAk=1, TG=2, TGRu=2, MM=1, Name=Wache 2, Rasse=Mensch, Alter=30, Size={1,93m}, Gewicht=90Kg, InvNor={Schwert(Schaden 4d6)}, InvGr={Großer Schild}, LPMax=72}

\CharacterSheetDefault
\GenerateCharacterRus{Kettenhemd}
\CharacterSheet{Ge=2, GeNa=2, GeAu=4, St=2, StAb=1, Ko=2, KoAu=2, KoZa=1, KoWi=2, Ch=1, ChMe=1, Wa=2, WaWa=3, AW=1, AWAk=1, TG=4, TGSc=4, TGRu=2, MV=1, MM=1, Name=Wache 3, Rasse=Mensch, Alter=27, Size={1,68m}, Gewicht=69Kg, InvGr={Repetier(2-60m, Mag.: 7, Nachladen: 6 Aktionen, Schaden: 22)}, LPMax=29}

\CharacterSheetDefault
\GenerateCharacterRus{Kettenhemd}
\CharacterSheet{Ge=3, GeNa=6, GeHe=4, GeAu=2, St=3, StAb=2, StKr=5, Ko=2, KoAu=2, KoZa=4, KoWi=2, Wa=2, WaWa=3, Name=Bandit, Rasse=Mensch, Alter=27, Size={1,68m}, Gewicht=69Kg, InvNor={Schwert(Schaden 4d6)}, InvGr={Großer Schild}, LPMax=36}
\end{characterSheets}
\end{document}

\input{Abenteuer/DerTraum.tex}
\input{Abenteuer/VerlorenerJunge.tex}
\input{Abenteuer/00Kampanie01.tex}
\input{Abenteuer/Plots.tex}
Alatas(Elf/Lucas)
Lucy(Morodianer/Felix)
Karf(Zwerg/Tobias)
Kijo(Tiermensch/Eric)

TODOS:
1. Wertesystem(Geldwert Bier Essen Unterkunft Gegenstände ...)(Orientierung ~ DSA /2) ~3(1-10(20)) Gold pro AP -> Rüstung Waffen etc haben Werte von 30-1000 Gold...
2. Tränke&Spruchrollen
3. Abenteurerbedarf(Zelt, ...)
4. Reisemittel(Pferd, Kutsche, ...)
5. ...

-> Kosten: Trinken: 1-3 Kupfer pro Becher(-unendlich bei Spezialwünschen)
-> Kosten: Essen: ~3-5 Kupfer pro Mahlzeit(-unendlich bei Spezialwünschen)
-> Kosten: Unterkunft: ~7K(Absteige)/1 Silber(Günstig)/++ pro Nacht.

->Sprache/...?/

Historie?!

->Scharfschütze(Doppelte Reichweite...)

-> Magie: 
	einschränkungen?(spezielle eingeschränkte Zauber einfacher, 
	allgemeine überarbeitung Stufensystem,
	 Gedankenkontrolle/lesen(Beschwörung), 
	 ?!Verzauberungen/Beeinflussung(aktuell nur über Beschwörung)!?
	 ?!Nekromantie(->als elementare(Lebens)kraft über Elementar)!?
	 ?!Erkentnisszauber(Illusionen vernichten(e.v. über angriffsmagie))!?
	 ?!Hellsicht/etc. e.v. über vertärkung!?
	=> Rebalance Verbesserungseffekte
	=> Rebalance Schaden/Wirkung(Zauber brauchen 2KR!)

DONE Bewegung KR=5s(3s):
DONE 	Gehen: Weltrekord 20Km Gehen: 4,3m/s
DONE 	Min: 7(4)m pro KR(3,5)(2)
		ca 1,5m/s -> 4,5 pro KR
		(Schrittgeschwindigkeit ca. 1 m/s oder als 4Km/h angesetzt)
		max:11
		ca. 1,5m/s-3,5m/s

DONE 	Joggen/rennen: Weltrekord Marathon: 5,8m/s; Hürdenlauf(3Km): 6,25
DONE 	Min: 11(7)m pro KR(6)(3,5)
		
		Dauerlauf(Marathon)
		Max:<3-<5m/s (<11-~17,5 pro KR)

DONE 	Sprinten: Weltrekord Sprint: 10,5m/s
DONE 	Min: 20(12,5)m pro KR(10)(6)
DONE 	Max: 50(30)m pro KR(25)(15)
		ca. 4m/s-10m/s
Laufen für Lame typen überarbeiten
	-> Basistempo*Größe + Akrobatik
		=> 	Gehen: Größe + (1,5)
					4-7
					=> (Str/4+1,5)*Größe
			Rennen: Größe*1,5 + (2)
					7-17,5/20
					=> Gehen((Str/4+1,5)*Größe)*2+Akro/2
			Sprinten: Größe*2 + (6)Akro*3/4
					12-30
					=> Gehen((Str/4+1,5)*Größe)*3+Akro

Gehen,rennen,sprinten
11/17,5/30
1/*1,6/*3
gehen min/max:1/*2,5
Attr: 3-4 zu 1-2
Spez: ~7 zu ~0
=> (1+Attr+spez/2)*(Größe-2)=Gehen
=> Gehen*1,5=Rennen
=> Gehen*3=Sprinten

Bewegung:
DONE	Rennen/Sprinten=Gehen*X
DONE	Probe nur einmal!?
DONE	Wandern/Gewaltmarsch -> fixformel
DONE?	Erschöpfung für Rennen/sprinten
Erschöpfung:
DONE?	- einfacher
DONE?	- pro Kampfrunde?
DONE?	- wichtiger?
DONE?	- Probe entfernen?(Schranken -> Normal,müde,erschöpft,ende)
DONE?	- Erschöpfung vs Zeitverlängerung für Zeitproben
DONE?	- Erschöpfungspunkte auf Charakterblatt


Technlogie:=> generatortabelle => obsolet
	-> beschreiben
	-> neue Waffen/Gegenstände
	

DONE? Kampf/Schaden&HP überarbeiten
DONE? 	HP: Ko:(8+1d8) + Zä:+1d4
DONE? 	Schaden: ST:XdY + Kr:ZdQ
DONE? 	Zähigkeit: ...?

DONE Kampfmanöver?!
DONE 	-> Absicht Beschreiben(Rundumschlag/Klingenwirbel/...)
DONE 	-> Manöver testen/erlernen(Kampfprobe)
DONE 		-> Krit: Dauerhaft erlernt(kostet sonst EP wie Zauber)
DONE 		-> Erfolg: Durchführung(bei erlernung wie mit Buch?)
DONE 		-> Misserfolg: Keine Aktion
DONE 		-> Krit: Malus(Stolpern)

DONE 'Würfelorgie' reduzieren
DONE 	Attr in Spez umwandeln?

DONE Freie Spezialisierungen?

Verteidigung:
	Abwehr: primär gegen Nahkampf, malus auf Fernkampf
	Ausweichen: primär gegen Fernkampf, malus gegen Nahkampf
	Schild: erlaubt Proben mit ausweichen oder Abwehr, immer ohne malus
	Schild bonus durch Rüstungskenntnis? => canceled
	-Vorteil für kleine!(Schadensbonus =20%-50% pro Größenkategorie)
			-v1:Schwierigkeit Verteidigung-1 pro Großenkategorie kleiner als Angreifer bei gezielten Angriffen

Rüstung:
	- Gibt bonus auf Proben statt min. Probe
		-> hilft nur Treffer zu entwerten nicht treffer zu vermeiden(
			angriff>=Verteidigung=> Treffer
			angriff>=(Verteidigung+RK)*1,5 => normaler Treffer
			angriff>=(Verteidigung+RK)*2 => kritischer Treffer
			angriff>=(Verteidigung+RK)*4 => kritischer Treffer)
	- Einfacherere Proben?/Schranken?
		- S1: tragbar
		- S2: Mali reduziert
		- S3: Mali~0
	- Spezialisierung umbenennen(Gegenstände)?
	

\section{Zähigkeit}
Wenn ein Angriff erfolgreich durchgeführt wird und Schaden erzeugt muss der Geschädigte eine Zähigkeitsprobe bestehen oder er fällt in Ohnmacht.

Für diese Zähigkeitsprobe muss 1 einfacher Erfolg aller vollen 5 LP Gesamtschaden und aller 5 LP Angriffsschaden geschafft werden damit die Probe erfolgreich ist(bsp.: 19 Punkte Angriffsschaden \& 1 LP Schaden von vergangenen Aktionen: 3*5LP(15+4LP) + 4*5LP(20LP) = 3+4 = 7 einfache Erfolge).

Sollte die Probe misslingen fällt der Charakter für die Differenz Sollerfolge - geschaffte Erfolge in Ohnmacht.





		
		
		
		
		
		
		
		
		
		
		
Würfel rework:(https://topps.diku.dk/torbenm/troll.msp : function)
	Idee 1: Attribute: 0..10; Spezialisierung: 0..??
Function:
	Attr:=5;
	Spez:=0;
	Min:=11;
	Max:=10;
	W:=20;

	(Spez~0)+sum foreach x in (Attr~1)d(W~1) do (count (x>=(Min~1)) - count (x<=(Max~1)))
		- Attributspunkte geben Würfel
			- Würfel können +/- 1 Erfolg generieren
			- >Würfelgröße*0,5: Erfolg
			- 1: Misserfolg
		- Spezialisierungspunkte geben (feste) Boni
		- Schwierigkeit:
			- je Schwieriger desto:
				a) kleiner die Attributswürfel(größere Chance auf Misserfolg(/kleinere Chance auf Erfolg))
				oder b) kleiner der Erfolg-teil/größer der Misserfolg teil des Würfels
					- trivial:?
					- einfach:?
					- mittel:?
					- schwer:?
					- meisterhaft:?
					- legendär: 50\%+ 50\%- -> würfel bringt (fast) nix
				+ optional c) mehr Spezialisierungspunkte werden für 1 Erfolg benötigt
					- offene Spezpunkte?
						-> wie jetzt?
			- je Komplexer desto:
				a) mehr Erfolge sind notwendig um die Probe zu bestehen
	Idee 2: Attribute: 0..10; Spezialisierung: 0..15(>18 unnötig?!)
Function:
	Attr:=8;
	Spez:=0;
	Min:=2;
	Diff:=5;
	W:=20;

	sum foreach x in (Attr~1)d(W~1) do (count ((x+(Spez~0))>=((Min~1)+(Diff~1))) - count ((x+(Spez~0))<(Min~1)))
		- Attributspunkte geben Würfel
			- Würfel können +/- 1 Erfolg generieren
			- >=X: Erfolg
			- <Y: Misserfolg
		- Spezialisierungspunkte verringern X und Y
			- Con: Spezialisten erfordern auch Attributspunkte
			- ?bei negativem X? (gibt es |X| als Erfolgsbonus?)
!!!!			- Würfeln e.v. sinnlos/weg -> schwierigeres ist nicht mehr(sinnvoll) mit Attributen allein schaffbar!!
		- Schwierigkeit:
			- je Schwieriger desto:
				a) kleiner der Erfolg-teil/größer der Misserfolg teil des Würfels
					- X/Y werden größer
					- trivial:X=8,Y=3
					- einfach:X=11,Y=6
					- mittel:X=14,Y=9
					- schwer:X=17,Y=12
					- meisterhaft:X=20,Y=15
				b) ??kleiner ist die Differenz X-Y???
			- je Komplexer desto:
				a) mehr Erfolge sind notwendig um die Probe zu bestehen
				
	Idee 1,2: Attribute: 0..10; Spezialisierung: 0..15(>18 unnötig?!)
Function:
	Attr:=3;
	Spez:=0;
	Erfolg:=2;
	Diff:=5;
	W:=20;

	sum foreach x in (Attr~1)d(W~1) do (count ((x)<=(Erfolg~1)) - count ((x)>((Erfolg~1)+(Diff~5))))

		- Attributspunkte geben Würfel
			- Würfel können +/- 1 Erfolg generieren
			- <=X: Erfolg
			- >X+5: Misserfolg
		- Spezialisierungspunkte wie jetzt
			- 1 Würfel 1 Stufe kleiner machen
				w20->w12!!->w10->w8->w6
			- 5 Punkte sind in 1 freien Erfolg umtauschbar
			- verbesserung in fester reihenfolge?
		- Spezialisierungspunkte2:
			- jeder Punkt kann benutzt werden um 1 belibiges Würfelergebnis um 1 zu verändern
				-> kann stacken.
		- Schwierigkeit:
			- je Schwieriger desto:
				a) kleiner der Erfolg-teil/größer der Misserfolg teil des Würfels
					- X/Y werden größer
					- trivial:X~15
					- einfach:X~12
					- mittel:X~9
					- schwer:X~6
					- meisterhaft:X~3
					- legendär:X~1
				b) master w12:
					- trivial:X~10
					- einfach:X~8
					- mittel:X~6
					- schwer:X~4
					- meisterhaft:X~2
					- legendär:X~1

			- je Komplexer desto:
				a) mehr Erfolge sind notwendig um die Probe zu bestehen
				
				
				
				
								
DONE Würfel v3(Fusion+):
   - Würfelsystem anforderungen: Spez/Attribute erzeugen eine gewisse '2 dimensionalität'
   - Erfolge/Misserfolge lassen sich in 'leicht'/'normal' und 'kritisch' einteilen(mindestens)

	Wirkung Attr: mehr Würfel
	Wirkung Spez: bessere Würfel(wie jetzt)(neue Würfel regeln: wenn nix anderes mehr geht oder wenn der Gegner mehr Würfel hat('auffüllend'))	
	=> Gegenproben:
		- Angreifer/Aktionator und Verteidiger würfeln normal
			- Attr*d20, die durch Spez verbessert werden
				20->12->8->4;-(3/4)>12->...
				primitiv:20
				Trivial: 16
				Einfach: 12
				Mittel:   8
				Schwer:   4
				Meister:  2
				Legendär: 1
			- Jeder Wurf <= Schwierigkeit zählt als Erfolg
			- Jeder Wurf > Schwierigkeit zählt als Fehlschlag
		- Die Erfolge des Angreifers werden AUFSTEIGEND sortiert
		- Die Erfolge des Verteidigers werden ABSTEIGEND sortiert
		- Falls einer der beiden mehr Erfolge als der andere hat werden die SCHLECHTESTEN direkt als Erfolg gezählt(für die Qualitätsbetrachtung zählt es als ob der Gegner eine(erfolgreiche) 20 gewürfelt hätte).
		- Angreifer u. Verteidiger Erfolge werden paarweise verglichen
			- gleich: ungültig
			- Angreifer niedriger: Angriffserfolg
			- Verteidiger niedriger: Verteidigungserfolg
		=> Kritisch:
			??- Je nach Qualität der einzelnen Erfolge(Differenz zum Gegner)?
				- vs Qualität der Gegenerfolge
				- +Qualität: besonders elegant
					-> wertet 'Erfolge gegen Gegner' zu 'freien Erfolgen' auf
				- -Qualität: 'Grad so geschafft'
					-> 'leichtere' Erfolge/Treffer
			- Je nach menge der
				- 'freie' Erfolge(übrige Würfel).
					- 1 'freier' Erfolg = 1 Dominanz über den Gegner
					=> Krit/'normalere Erfolge'/positiver Nebeneffekt?/schneller?
				- Erfolge gegen Gegner.
					- 1 Erfolg = 1 verhinderte Reaktion
					=> 'normalere' Erfolge/Treffer /?positiver Nebeneffekt?/?schneller?
				- Gegenerfolge(ab 3 Verteidigererfolgen,...)
					- 1 Gegenerfolg = 1 (teilweise) erfolgreiche Reaktion/Komplikation/unvorhergesehene Hürde
					=> Erfolgsreduktion(erst 'Erfolge gegen Gegner' dann 'freie Erfolge')
					=> Antikrit
				- Fehlschlägen
					- 1 Fehlschlag = 1 Patzer/Missgeschick/Komplikation
					=> Wertet 'freie Erfolge' zu normalen Erfolgen ab.
					=> Antikrit/'leichte Erfolge'/negativer Nebeneffekt/langsamer
			- Je nach Erfolg/Gegenerfolg rate(3x so viele Erfolge wie Verteidiger,...).
			- Je nach Erfolg/Anzahl rate(9/10 Würfel haben einen Erfolg generiert).
			=> z.b. 
				- jeder eigene Erfolg +3 Verbesserungspunkte
				- jeder gegner Erfolg -3 Verbesserungspunkte
				- jeder eigene Fehlschlag -1 Verbesserungspunkte
		=> Situationsboni(Kampf gegen mehrere Gegner): Schwierigkeit wird erhöht.


	=> Normale Proben: Wie Gegenproben, aber:
		- es wird für die Probe e.v. ein Minimalergebnis festgelegt(max. X Gegenerfolge/Fehlschlägen oder min Y Erfolge)
		- (oder) Es wird zusätzlich zur Schwierigkeit ein 'Attribut'/'Spez'/'Schwierigkeit' für die Umgebung definiert.
		- um die Lösungsqualität zu verbessern können freiwillig mehr 'Umgebungswürfel' erzeugt werden.
			- +4 Verbesserungspunkte pro Extrawürfel
		- Lösungsqualität(Geschwindigkeit/Eleganz/Komplikationen/...) wird durch die Menge der:
			- 'Umgebungserfolge'(o. rate)
			- Fehlschläge(o. Fehlschlagsrate)
			- Erfolge
		und die 'Qualität' der einzelnen Erfolge(Differenz zur 'Umgebung'/Schwierigkeit/20) bestimmt


	=> Zeitproben: entfällt(bzw. ersetzt durch boni/Erleichterung)
	
Dices(Damage&Co):
	Nahkampf: 1dx
		-> dx ist bestimmt durch den "Waffenstäkewert"(min. Wert um Waffe zu benutzen)
		-> Waffenstäkewert hat je nach Gegenstandsgröße bestimmte Standardwerte(+min/max)
		-> Größere haben einen bonus auf ihre Stärke wenns um den Waffenstäkewert geht(Troll mit Stärke 1 kann Keule mit Waffenstäkewert 2 benutzen)
	Fernkampf: festwert
	Würfeln: Krits seltener machen -> Geg. Erfolg negiert zuerst die Krits dann die Erfolge
	Schaden: 
		-> Krit stufen ermittlen(anzahl Krits-Anzahl Misserfolge)
		-> Krit stufe > 0: würfele extra dx(siehe Schaden)/multipliziere Festwert mit der Krit stufe
		-> Krit stufe < 0: würfele d(x*2^Krit stufen)(-1 = d8 zu d4/d12 zu d6/d6 zu d4/d4 zu 1/1 zu 0)
	Rüstung:
		-> pro 2 Rüstungspunkte: entferne 1 positive Krit stufe
		-> pro 1 Rüstungspunkte: veringere Krit stufen um 1 unter 0


Nahkampf zu viele Proben(rankommen/wegkommen + Angriff)
	=> Distanzkram überarbeiten
		'Klauen bei Low Fantasy von Felix?'
			=> Runde: 
				1. Manövrieren der Nahkämpfer(Supportaktion)
					a)
						Distanzänderung/Bewegung zu/von einem anderen -> mind. supp aktion
						(relative) Distanz halten: supp aktion oder Nahkampfverteidigungsaktion
					b)Probe:
						alle beteiligten machen 1? entsprechende Probe(akrobatik/Ausweichen)
							- Akrobatik: nur gegen 1 Ziel(näher/nicht weiter weg)
							- Ausweichen: gegen beliebig viele Ziele(nicht zu nah)
						es wird festgelegt auf wen man sich fokusiert(Angriffsziel)?
					c)auswertung:Kritische Erfolge einzusetzen benötigt min. 1 anderen Erfolg
						Ausweichen Probe:
							- pro Erfolg: Erhöhung der Distanz um 1/halten der Distanz
							- Kritischer Erfolg: Kann die (anderen) erfolge gegen einen 2. Gegner einsetzen
							- 2x Kritischer Erfolg: Kann die (anderen) erfolge gegen 2 andere Gegner einsetzen
							- ...
 						Akrobatik Probe:
							- pro Erfolg: senkung der Distanz um 1/halten der distanz
							- Kritischer Erfolg: Kann die (anderen) erfolge (als ausweichen erfolge) gegen einen 2. Gegner einsetzen
verwenden(ohne diesen Kritischen)
												
				2. Angriffe/Sonstiges(Spieler)
				3. Angriffe/Sonstiges(NPC/Gegner)
				4. Leute sterben
		=> Alternative zu Distanzproben: mit in Angriff verrechnen: 
				1. spezielle Manöver um Distanzen schnell zu ändern(mit entsprechenden Mali)
				2. Erfolge bei Angriff-> näher ran, Erfolge bei Verteidigung(ausweichen) -> weiter weg
etwas viel Schaden(HP*1,5)?! => Basisschaden auf 4 reduzieren
	-> Rüstungszeugs(Mali ca. 1/3 weniger)




Alatas(Elf/Lucas)
Lucy(Morodianer/Felix)
Karf(Zwerg/Tobias)
Kijo(Tiermensch/Eric)

Hinweis: 1 Gold~300€

Schwierigkeitsmeter:
	-Trivial: keine besonderen Kenntnisse nötig(Gehen auf Staßen, nahes Gebirge finden, Tür öffnen(mit schlüssel), Leiter hochklettern, ...)
	-Einfach: Standartaufgabe(Schlagen, Zechen mit Bier/Wein, Orientierung, Tür aufbrechen, Seil hochklettern, ...)
	-Mittel: Semiprofessionelle Kenntnisse nötig(Orientierung im Wald/Höhlen, Klettern an Felswand, )
	-Schwer: professionelle Kenntnisse nötig(Orientierung in der Wüste, Klettern im Überhang, )
	-...
	-'+1': 1 Einschränkung(Klettern mit leichter Verletzung an der Hand, ...)
	-'+2': 1 starke oder mehrere Einschränkungen(Klettern mit leichter Verletzung an der Hand, ...)
	- '+...':...
	-'-1': mit Hilfe(Klettern mit Professioneller Ausrüstung, mit (guten) Anweisungen, ...)
	...
Komplexitäts:
	1: Trivial(1m Klettern, 1 Bier/... trinken, ...)
	2: Normal(<5m Klettern, 3 Bier/... trinken, )
	3:
	...
	12: extrem(doppelt versteckter Mechanismus)

		
Sonst:
	-Leichter/Schwerer als Trivial/Legendär optional+/- 1 Erfolg pro extra Schwierigkeit wenn Probe erlaubt ist auf Probenergebniss
Regeln e.v. vereinfachen
DONE	- Tabelle mit garantierten Erfolgen (100% bis zu Wert X)
	- Spezialtallente &"Primärspez." geben Schwigkeit -1
	- Ausgedehnte Proben: beschränken -> nicht beliebig oft ausführbar machen!!


		
Gegenstände:(siehe auch boni/Ausrüstung in "Das System v1.ods" -> Komplettüberarbeitung?!)
	- RÜSTUNGEN
		-> RS+RK tragen zur Rüstungsstufe bei
		-> je nach Rüstungsstufe sind die Mali/Voraussetzungen/Das Material(Leder,Metall,...)/Rüstungstyp(Platte/...) gegeben.
		-> Rüstungstyp MAX: Golemanzug(Stein)
	- mehr stufen?
	- unterteilung in Primäre QS und sekundäre QS
		-Primäre: Größe, Qualität(/Stabilität) und Dekowert: ändert die Kosten pro QS
		-Sekundär: Rest, wie jetzt
		-Gegenstände haben passive Kosten(Kosten=XQS+/-SekundärQS)
		-Bonusschaden neu balancieren(einen Grund für kleine Bonusschäden? z.b. Waffenbeschädigung=Fixschadenfaktor pro Schuss) => auch bei Nahkampf?(alle 2 Würfel)
		-Größe neu Balancieren(vorallem mit fernkampfwaffen -> warum sollte jemand eine kleine fernkampfwaffe wählen?)
		-Alle Größen von den Generischen Größenklassen ableiten(0,33/0,5/0,66/1/1,5/2/3/4,5/7/10,5/16)
			-Schadenswerte: <=0,33=>d4/0,33..0,66(2Klassen)=>d6/0,66..1=>d8/1..1,5=>d10/1,5..2=>d12/2..3=>d8+d6/3..4,5=>d20/...
			-Benötigte Hände(Körpergröße-Stufe=x):<=0,33 oder <x-2=>0,5/x-2=>1/x-1=>1,5/x=>2/x+1=>3/...
			-Inverntar: klein: <=0,33/normal:<=x-2/groß:<=x
  -Bonustabelle für generische Boni?!
	-> Einfache Boni/QS
	-> Normale Boni/QS
	-> Tech. Boni+QS
	-> Extreme Boni +QS
	e.v. Wie 'Gegenstandsding' von Lucas?
		-> alles ist ein Boni(Reichweite, Ladungen, ...)
		-> ALLE Boni kosten TSP UND QSP(können aber auch negative oder =0 kosten in einem oder beidem haben)
			-> TSP sind (üblicherweise) fest für einen Bonus/Modifikator(Schaden, Magazin, ...)
			-> QSP verändern die Stärke(1m, 3m, 100m) des Bonus/Modifikator
			=> Tabelle: |Bonus|TS|Effektstärke1&QSP1|Effektstärke2&QSP2|...
		-> Unterscheidung in Basisboni und Modifikatoren
			-> Basisboni: z.b. Schaden, Zeitschaden, Rüstung, Taschen, ...
				-> 'Funktionen'
			-> Modifikatoren: Reichweite, Ladung/Reload, Flächenwirkung, ...
				-> 'Metafunktionen' und 'Spezifikatoren'
				-> (Bel.?) menge and Modifikatoren pro Basisfunktion
				-> Aufgeteilt in Modifikatorgruppen(Reichweite, Schaden, ...)
					-> aus jeder Modifikatorgruppe kann nur max 1 pro Boni gewählt werden
					-> Bestimmte Modifikatorgruppen sind für bestimmte Basisboni nicht verfügbar(Spielleiterentscheid)
					-> z.b. Gruppe Reichweite besteht aus den Modifikatoren Selbst,Berührung/Nahkampf,Fernkampf
						-> QS(Nahkampf): Kurz,Mittel,Lang(jeweils mit Angriff/Verteidigungsmali gegen längere Nahkampfwaffen)
						-> QS(Fernkampf): 10m, 20m, ...
			-> 'Extender': e.v. 'Grundkosten'(TSP) der Basisboni
				-> Transformator als 'negativ' Modifikation -> -QSP je nach Dauer der Transformation, QSP einer Funktion(+Modifikatoren) können damit nicht unter 0 gebracht werden(-> an/aus/um Schalter).
		=> Gegenstandsding kann dann auch auf Magie angewendet werden?
			Basiseffekt(Magie):MSP je nach Wirkung(inklusive (MSP)Modifikatoren)
				=> =Basisboni?
			Verbesserungen:MSP pro Verbesserung+MKP je nach Bonus
				=> =Modifikatorgruppe(/Stärke)?
			Macht Verbesserung:
				=> =Modifikatorenstärke(/gruppe)?

			
  -Fernkampf:
	- Nachlanden überarbeiten?
		- Usage-dice/Munitions-dice(bei 1 Nachladen)
		- Nachladen dauert normal 1 Runde
		- ??
	- Kosten pro Schaden überarbeiten
	- Je nach Größe haben Waffen andere Reichweiten (min-max1/max2/max3) =>min behandeln wie max(:2->Schwierigkeit +1
	- Max Fixschaden=TechStufe
	- 
	
Stabilität:
	-Zufall weg!? -> Stabilität-#Würfel-Fixschaden*2=Schaden an der Waffe 
			=> Provisorisch=1
			=> 1QS=2 2QS=3 4QS=4 8QS=6 12QS=8 20=10
	- <0: kommplett Kaputt
	- O-1: unbrauchbar
	- 2-4: stark beschädigt(+1 auf alle Proben, ...)
	- 5 - 7: beschädigt(-1 auf Stabilität)
	- 8 -10: ganz/leicht beschädigt
	- Probe:
		- 0,5 pro würfel/O,5 Fixschaden
		- Sonst: je nach Einsatz/...
		
Tech Geschick:
	-Reparieren????
Reperatur/Konstruktion:
	-> einfluss durch QS
	- Rep: pro HP ~10% Herstellungskosten + 10% der QS Schwierigkeit je nach Beschädigung
	- Konstruktion: 
		- QS=Benötigte Erfolge
		- Herstellungs/Materialkosten je nach TS(
			1/2 der 'Kosten' bei TS0, 
			1/3 bei TS1, 
			1/4 bei TS2,
			1/6 bei TS3,
			1/10 bei TS4,
			1/15 bei TS5)
		- Material ist TS-1 und kann entweder mit Mechanikprobe mit QS Erfolgen hergestellt werden oder gekauft/angefertigt werden(Mat. Kosten erhöhen sich um 1 Stufe+Muss verfügbar sein)
		-Geschick ist benötigt?!(synergien?)
		
		
		
Zauberoption:
	- Zaubereffekte generischer: 
		mit Zauberweben lässt sich die Form verändern
		mit Zaubermacht lässt sich Radius/Stärke verbessern
	Effekt: ~8 Schaden pro erfolg bei St.1
	Effekt: *1,5 bei St2...
	Problem aktuell:??
		-> wenn SL Magie wirken will:
			SL würfelt gegen sich selbst(Magie vs Umgebungsprobe)
			-> bessere Lösung:?!		

Verführen: -> Einschmeicheln: vs Willenskraft/Wachsamkeit
Sozial: Fehlschläge bearbeiten?		

Magiteck mehr zeug machen(was/wie) -> nochmal in dem Regelteil schreiben
-> Technisches Wissen wichtiger machen!/lohnt sich aktuell nicht
=> nicht nur Basteln?

Einkaufsliste als Tabelle.

Treffer ~ 30-70% im schnitt
Schaden leicht;normal;krit;multikrit~50;30;10;10%

MAGIE!!!
Magie zu wenig/langweilig
Fernkampf:
	-> zielen auf Körperteile???
	-> Interessanter
	-> Nachladen & Co?(siehe oben)
Ausdauerkosten für Aktionen festlegen(1 pro Kr + extra pro Attacke)
Kampferschöpfung wirkt sich am ende des Kampfes aus?(Annahme kurzer Kampf -> Adlenarinschub)

Würfel/Wurf auswertung zu langsam
	-> sortieren braucht viel zeit
	-> Würfel raussuchen passt..
	-> auswertung heftig(2..3 vergleiche=1 Addition;2..3 Additionen=1 multi)
		-> 5 Würfel: 5 vergleiche + 5/7 für sortieren + 4/5 für Gegner vergleich + ...
		-> 4-5 Aktionen pro würfel(max 3 aktionen!)
		
		
		
TODO:
	Kampange: bei erstellung: festlegung einer Organisation(Elfen/Technohasser, Zwerge/Forscher/Erkunder, Abenteurer(Monster töten & co), Gnome/Diebe/..., Menschen/Söldner)
	Für mich: 2. Übersichtsseite mit 'Standartwerten' bzw. 'Standartzauber'/'Standartmanöver' aus denen Gegner auswählen können...
		Rüstung
		Gegnerwerte/Defaultgegner
		Standartzeuber in kurz
		Manöverliste(auch für andere)
		Bewegungs cheat sheat(auch für andere)
		...
GRUPPENKAMPF!!
	- Verteidigungsmali statt Angriffsboni für viele Gegner
	- Gruppenaktionsregeln(z.b. +Spez/2 der Unterstützer)
	- Erschwerung/Erleichterung nicht so sinnvoll? -> besser boni/mali? => Simulieren

PLOT=SÖLDNER!!


FERNKAMPFWAFFEN? -> verschieben da aktuell nicht benötigt...
!!- Kleine Fernwaffen!!

- Verteidigung: Keine Situationsboni/Schwierigkeiten(sonst blöd mit nur 1x Würfeln)
	- Angreifermenge(gegen wieviele man verteidigt) gibt Verteidigungsmalus
	- Schild gibt vert. Bonus
	- Größe gibt keinen Vert. bonus/malus(oder gibt statischen Bonus unabhängig von Gegnern)
		-> Tankige kleine Leute?
		-> e.v. LP=(Größe*4+1d8) pro KO punkt 
			dafür Verteidigungsbonus=X*(6-Größenkategorie)
				!!Gleichgewicht Faktor/Boni
	- ??Rüstung: bessere Tragbarkeit???
	- Kampfdistanz gibt keinen Verteidigungs bonus/malus -> Angriffs malus/bonus
	- Rüstung gibt keinen Verteidigerbonus dafür Angriffsmalus(für 'Umgehungstreffer')
		- Kleine Waffen: Bonus auf 'Rüstungsdurchdringung'/gezielte Angriffe
			- Umgehungstreffer erschwerniss/bonus= 2*(Rüstung+Waffengrößenkategorie-Y)
			!!Gleichgewicht Schaden/Trefferboni!!(incl. Kampfdistanz)

- Giftwurfdoch
- Heilung
- Waffen
	- Bogen mit spezialpfeilen...
- Spezialpfeile machen...
	-> Bastelset
- Schild/Schutzzeug?!
- Troll

Angriffsboni!!




Full renew:

Welt:
 - Teilweise revolutioniert aber zum großteil eher hinterwäldlich('Artefaktstädte' von Morodax in der entwicklung beschleunigt kenntniss über genauen aufbau größtenteils unbekannt)
 - Morodax ist vor +100? Jahren verschwunden

Gruppenabenteuertypus:
	- Erkundung von Dungeons -> aufspüren von Techno-magischen artefakten
	- Monsterjagten
	- ...
=> Kampf(zentral) + Erkundung/Handel/...
Priolist:
0. Würfeln -> (/)?
1. Kampf
2. Magie
3. Techno-magische Artefakte
4. Sozial/Interaktionsdöns
5. Konstruktion/Basteln??? -> eher nicht
 
System:
 - Kampf/Erkundungsorientiert
 
 
Würfeln's
=========
orig: komplex tabelle...
v2: vergl. mit gegenwurf
v3(prev): 1 Wurf(Aktionswurf) x(6) würfel 1=-2;2=-1;4+=1;10+=+2
=> x>3: auch eine 1(-2) noch kein zwingend schlechten Konsequenzen.
	=> x= 6(Würfelpool) 0 eher selten
=> x==3 -> eine 1 kann auf '0' aka semi Erfolg ausgeglichen werden
=> vergl. system 'Gruppengründung'
Attribut: max nutzbarer würfel((=2 d4) d6 d8 d10 d12 d20??)
Spezialisierung/bonus/malus/...: würfelkonvertierung 
=> 1 spez-punkt konvertiert 2(1) würfel um 1 stufe(immer die niedrigsten)
=> Attribute im bereich 0(depp)..5(Genie)
=> Spez im bereich 0(depp)..10(+)(Genie) bei erschaffung ~max 5(+- boni durch werkzeuge möglich)
Ergebnisse: 
	<Würfel -> Epic fail
	<0 -> fail; 
	0 -> Erfolg, aber: a) es dauert lange(*5/30min+)(oder kann abgebrochen werden) oder b) es passieren 'Schusselfehler'
	>0 -> 'guter Erfolg'
	>(>=/>)Würfel Epischer/Perfekter Erfolg
=> Option Scaling Würfel von 3->6 je nach Attribut?
	-> Attribut: (2/3-6) Würfel
	-> Spez/Malus: würfelqualität -> upgrade von ALLEN würfeln auf Bonus/2(3?) (,x -> hälfte auf/abgerundet(><0,5))

	
Magie's:
- Erfolgsorientierte Proben
- Zauberweben und Zaubermacht/Kraft getrennte Proben
- Zauber ohne macht macht nix
- Abgebrochender Zauber mit Macht macht etwas(e.v. nicht das gewünschte)
- Normal nix anderes möglich während Zaubern(full fokus)
- Schaden/Angriffe auf Zauberer benötigen Zähigkeit? probe damit Zauber dadurch nicht abbricht(relativ Schwer aber bei leichtem Schaden nicht unmöglich...)
- Zauberweben/Macht kann über mehrere Runden aufgebaut werden um Mächtigere Zauber zu wirken(p.e. '16 Erfolge')
- Erfolge sollte liniar o. leicht steigende Auswirkungen haben
=> Grenzwert für Zauberproben außerhalb von Kämpfen....(in Kampf durch Kampfrunden/Schadensrisiko gebalenced)
=> Erschöpfungssystem?... -> max X(Ausdauer/Konzentration) Zauberproben hintereinander möglich(dann Ruheaktion o.ä.)
=> E.v. erschöpfungssystem ähnlich wie Lukas Stamina/Fokus system?
	-> option 1: Kampf als 'Adlenarinrausch' -> keine Kosten aktion möglich wenn stamina/Fokus vorhanden -> nach kampf alles weg -> Regeneration mit Pause
	-> option 2: Fokus wird für Zauber verbraucht.
				 Spezialaktion stellt Fokus wieder her für Ausdauer
				 ander Spezialaktionen kosten Ausdauer für 'fanzy' Angriffe
				 Ausdauerregeneration benötigt Kampfpause(5/10? min Ruhe/Verschnaufen)
- Magiebaukasten? 
	-> Magieeigenschaften -> Kosten(Probemalus + Erfolgsmengen) ect...(Angelehnt an Butterfly worte?)
	-> Zauberoptionen('live anpassungen der Matrix') -> kosten Basis(Probenmalus) + Kosten stufen(Erfolge)
	-> Weben & Macht elemente
	-> ...
	
Kampf's:
- Probe mit malus (eingesetzter) Gegnerischer Fähigkeit
- Erfolg = Treffer 
	-> Waffenschaden * (Erfolgsmenge +(/*) gegnerische Rüstung) (+- Magieschutz) - Zähigkeit gegner = Schaden
	

