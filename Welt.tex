\raggedright
\chapter{Generelles}
Dieses Regelwerk soll keine Vollständige Beschreibung von Morodax darstellen. Es ist vielmehr dazu gedacht einige Grundregeln zu beschreiben und Beispiele/Orientierungen zur Balance zu liefern. Sollte ein Spieler/Meister neue Elemente hinzufügen wollen(Waffen/Ausrüstung/Zauber/...) soll dieses Dokument Anhaltspunkte zur Generierung eines solchen bieten.

\part{Die Welt}
\chapter{Die Geschichte}
\section{Entstehungsgeschichte/Religion}
Es gibt viele Religionen auf Morodax, sie alle haben eins gemeinsam: der Gott Morodis ist bei aller der (höchste) Gott, der Morodax erschuf bzw. der Welt leben brachte. Morodis wird dabei von vielen Religionen völlig unterschiedlich dargestellt, sowohl vom Aussehen als auch  von der Persönlichkeit her. Morodis selbst war/wird sein ein extrem mächtiger Magier-Cyborg/verrückter Wissenschaftler, der eine Zeitreise durchgeführt hat, um aus seiner Zeit zu fliehen. Durch die Zeitreise(oder vielleicht auch schon vorher) hat er seinen Verstand beschädigt, wodurch er eine vielzahl von Persönlichkeiten entwickelt hat. Mithilfe von Illusionsmagie hat er dann öfter sein Aussehen komplett verändert und durch die Technologie der Zukunft konnte er auch bedeutend länger leben als alle anderen in dieser Ära.

\section{Monde}
\begin{description}
\item[Zielel]\ 
\begin{description}
\item[Drehrichtung] wie unser Mond, parallel zur Rotationsachse der Welt
\item[Umlaufzeit] $\approx 18 Tage$
\item[wahrgenommene Größe] weniger als halb so groß wie unser Mond($\approx\frac{1}{3}$)
\item[Farbe] dunkelrot mit funkelnden(hellrot-glitzern) Stellen, bei Vollmond

 Leuchtet rot wenn er von hinten angestrahlt wird(einen anderen Mond/Sonne verdeckt).
 
 sonnst eher schwer zu sehen.
\item[Besonderheit] Riesiger roter Kristall, gilt als 'Magiemond'(ist in der Tat aus einem den Magiekristallen ähnlichem Material).
\end{description}
\item[Kochbalth]\ 
\begin{description}
\item[Drehrichtung] wie unser Mond, fast senkrecht($< 10^\circ$) zur Rotationsachse der Welt
\item[Umlaufzeit] $\approx 34 Tage$
\item[wahrgenommene Größe] $\approx$ so groß wie unser Mond
\item[Farbe] gelb (Vergleichbar mit Io)
\end{description}
\item[Althah]\ 
\begin{description}
\item[Drehrichtung] anders als unser Mond('(' = zunehmender Mond), fast senkrecht($< 10^\circ$) zur Rotationsachse der Welt
\item[Umlaufzeit] $\approx 33 Tage$
\item[wahrgenommene Größe] $\approx$ so groß wie unser Mond
\item[Farbe] lila
\end{description}
\item[Sariel]\ 
\begin{description}
\item[Drehrichtung] wie unser Mond, $\approx 30^\circ$ zur Rotationsachse der Welt
\item[Umlaufzeit] $\approx 46 Tage$
\item[wahrgenommene Größe] $\approx$doppelt so groß wie unser Mond
\item[Farbe] weiß-blau(gefrorenes Wasser)
\end{description}
\end{description}

\section{Kalender}
Ein Jahr hat 337 Tage bzw. 9 Monate(Jeha(33 Tage), Dailiel(46 Tage), Haelyr(34 Tage), Biniel(46 Tage), Saraq(33 Tage), Pyriel(46 Tage), Selik(34 Tage), Zaliel(19 Tage), Rielach(46 Tage)).

Eine Woche besteht aus 6 Tagen(Hielaph, Marach, Arah, Kusha, Sielyr, Zielah(Feiertag)).

\section{Zeitstrahl}
Jedes Zeitalter besteht aus 1042 Jahren genau nach dieser Zeit gibt es eine Besondere Konstellation der 4 Monde und der Sonne, die die Ausführung besonders mächtige Magie erlaubt(weshalb Morodis zu genau diesen Zeiten auch erschienen/verschwunden ist).
\begin{description}
	\item[Jahr 0(altes Zeitalter):] Ankunft von Morodis (Technischer Stand entspricht etwa der Stein- bis Bronzezeit).
	\item[Jahr 0-500(altes Zeitalter):] Massiv beschleunigte Entwicklung der Technologie vor allem im Bereich Rohstoffraffinerierung durch Morodis. Entwicklung der Morodis Religionen('Allwissendes Wesen'). Verschiedene Rassen/Gebiete werden von Morodis dabei vor allem in bestimmten Bereichen 'ausgebildet'.
	\item[Jahr ~500(altes Zeitalter):] Morodis beginnt mit ersten größeren Experimenten und verursacht u.a. die bis heute Existierenden 'Weltenrisse' durch hin und wieder diverse Monster aus anderen Dimensionen in diese Welt gelangen.
	\item[Jahr ~500-1000(altes Zeitalter):] Durch Morodis sind verschiedenste Rohstoffe bekannt/Raffinerierbar(inklusive Öl/Seltene Erze/...) jedoch ist die Bedeutung von vielen dieser Konstrukte die Morodis haben will recht unklar. Morodis selbst hat die Entwicklung für seine Zwecke weit genug vorangetrieben und kümmert sich nicht weiter groß um die weitere Förderung der Entwicklung. Morodis selbst kümmert sich in verschiedenen Laboren/Werkstätten um eigene Projekte.
	\item[Jahr 1042(altes Zeitalter)/Jahr 0(nach Morodis):] Morodis verschwindet. Viele seiner Labore/Werkstätten sind versiegelt und z.t. von diversen Monstern bewacht, da Morodis seine Geheimnisse mit niemandem teilen wollte.
	\item[Jahr -XXXX:] Nachdem die verschiedenen Rassen nicht weiter von Morodis gefördert werden versuchen sie nach und nach selber zu verstehen was Morodis sie bauen ließ/untersuchten Artefakte die Morodis zurückließ/vergessen hat.
	\item[Jahr XXXX:] 'Zwergenkrieg'
	\item[Jahr 1013:] 'Heute'(siehe unten) mehr oder weniger vergleichbar mit <1800 bei uns.
	\begin{hiddenWorld}
	\item[Jahr unbekannt:] Abbau von Zielel, da dieser ein hocheffizienter Magiekristall ist.
	\end{hiddenWorld}
	\item[Jahr unbekannt:] Morodis Geburt
\end{description}

\section{aktuelle Situation}
Nachdem Morodis auf der Welt erschienen ist(und seinem Wirken) sind die Grenzen zu anderen Dimensionen dünner als sie dies normalerweise sein sollten, dadurch ist Magie relativ verbreitet und hin und wieder kommen Wesen aus anderen Dimensionen in unsere Welt(oder umgekehrt).
Diese beiden Effekte führen zu einer relativ hohen Anzahl an Monstern(entweder durch Magie veränderte Tiere oder Wesen aus anderen Dimensionen).

Die hohe Anzahl an Monstern hat unter anderem zur folge, dass Abenteurer gebraucht werden, die gegen die Monster kämpfen um Karawanen oder Siedlungen zu schützen oder einfach um wertvolle Materialien zu erbeuten.
Ein beliebtes Material sind unter anderem Magiekristalle die sich in den Körpern von magischen Wesen bilden, vor allem bestimmte Monster haben große/hochwertige Magiekristalle die für die Herstellung magischer Hilfsmittel benötigt werden.

Zudem gibt es diverse unerforschte Dungeons die zum teil von Monstern bewohnt werden, von denen einige alte vergessene Labore von Morodis beinhalten und somit wertvolle Magisch/Technische Artefakte beherbergen.

\subsection{Kulturelle Situation}
Es gibt schon seit langem einen (schwelenden) Konflikt zwischen Elfen und Zwergen, in diesem waren die Elfen lange Zeit durch ihre Magie den Zwergen überlegen. Durch die Entwicklung von Arglardor(von anderen auch als Blutstahl bezeichnet, aufgrund seiner sehr dunkel roten Farbe), einem speziellen Magieresistenten Metall, konnten die Zwerge sich jedoch mittlerweile auch einen entscheidenden Vorteil verschaffen. Aufgrund der hohen Herstellungskosten kann dieses Metall jedoch nicht flächendeckend in einem Krieg verwendet werden. Aktuell kommt es daher nicht zu aktiven Auseinandersetzungen zwischen dem Reich der Zwerge und der Elfen, weshalb die Leute es den ruhenden Konflikt nennen.

Im Rest der Welt gibt es verschiedene Rassen, zwischen denen es auch z.t. zu Reibereien und Rassismus kommt. Jedoch gibt es auch das Land der Morodianer, Abkömmlinge von Teufeln und Dämonen, die von Morodis beschworen worden, in dem eine bunte Mischung aller Rassen lebt, die halbwegs funktioniert.

\subsection{Das Geld}
\begin{TODO}
Jeder eigene Währung mit ähnlichen Werten jedoch entsprechenden Umtauschkursen?!
\end{TODO}
Es gibt ein Währungsbündnis, hauptsächlich von den Morodianern initiiert, bei dem die angrenzenden Kulturen mitmachen, die allerdings z.t. eigene Prägungen haben(für Münzen)(vergleichbar mit dem System des Euros).Es gibt folgende Währungseinheiten:

\begin{tabular}{|c||c|c|c|}
\hline
Währung & \multicolumn{2}{c|}{ Wechselkurse } & Bemerkungen\\
\hline
\hline
1 Zinnmünze & - & $\frac{1}{1000}$ Goldmünzen & quasi ausgestorben\\
\hline
1 Kupfermünze & 10 Zinnmünzen & $\frac{1}{100}$ Goldmünzen & $\approx$1 Bier\\
\hline
1 Silbermünze & 10 Kupfermünzen & $\frac{1}{10}$ Goldmünzen & $\approx$1 Übernachtung\\
\hline
1 Goldmünze & 10 Silbermünzen & $\frac{1}{10}$ Goldmünzen & $\approx$1 Dolch\\
\hline
1 'Kupfer'-Geldkristall & 100 Goldmünzen & 100 Goldmünzen & $\approx$1 guter Wagen\\
\hline
1 'Silber'-Geldkristall & 10 'Kupfer'-Geldkristalle & 1K Goldmünzen & $\approx$1 einfache Yacht\\
\hline
1 'Gold'-Geldkristall & 10 'Silber'-Geldkristalle & 10K Goldmünzen & $\approx$1 Villa\\
\hline
1 'Mithril'-Geldkristall & 10 'Gold'-Geldkristalle & 100K Goldmünzen & $\approx$1 Schloss\\
\hline
1 'Adamantium'-Geldkristall & 10 'Mithril'-Geldkristalle & 1M Goldmünzen & $\approx$1 Berg\\
\hline
1 'Morodis'-Geldkristall & 10 'Adamantium'-Geldkristalle & 10M Goldmünzen & $\approx$1 Stadt\\
\hline
\end{tabular}
Die Geldkristalle sind winzige Magiekristalle(kaum größer als Münzen), die speziell angefertigt wurden um ihren Wert zu garantieren. Sie sind zudem fälschungssicher und diebessicher, da \begin{enumerate}
\item Jeder Kristall an seinen Besitzer gebunden ist.
\item Nur durch gegenseitiges Einverständnis kann die Bindung übertragen werden.
\item Manipulationsversuche/Fehlaktivierung/Schaden/... führen zur Aktivierung der Selbstzerstörung des Kristalls(Kristall \& Nutzer erhalten 200(d1) magischen, tödlichen und direkten Schaden(Direktangriff auf LP ohne Abwehrmöglichkeit))
\item Nutzen zur Identifikation den 'magischen Fingerabdruck' des Besitzers(die Bedienhand muss frei von Aglardor sein).
\item Besitzen einen Schutzschild, mit 50('Kupfer')-500('Morodax') Punkten Schutz \& Regeneration von 10('Kupfer')-100('Morodax') Punkten pro Kampfrunde.
\end{enumerate}
%Hinweis: 1 Gold~300€

\chapter{Die Magie TODO}
Für die die sich nicht dem langwierigen Studium der Magie hingeben wollen hier ein Auszug aus der beliebten Buchreihe "Magie für Zwerge": "Man kann sich einen Zauber vorstellen wie ein komplexes Rohsystem wenn man in das eine Ende Dampf einspeist und damit das System unter Druck setzt werden die Rohre mit Dampf befüllt. Je nach Konstruktion der Rohre(oder der angeschlossenen Systeme) können jetzt die verschiedensten Dinge passieren, z.B. könnte eine Düse genutzt werden um den Dampf zu fokussieren und als Waffe zu benutzten oder in ein Rohr könnten viele Löcher eingebaut werden um eine riesige Nebelwolke zu erzeugen."

\section{Magie wirken}
Die Magie von Morodax ist eine allumfassende Energie, alle sind mit der Magie verbunden manche stärker manche schwächer, manche sind sich der Magie bewusst, manche wissen vielleicht gar nicht welches Potential in ihnen schlummert.

Um die Magie verwenden zu können ist eine natürliche Begabung zur Manipulation der Magischen Energien notwendig(Magische Macht) und das wissen darüber wie diese Begabung genutzt werden kann(Magischer Verstand). Dann ist es einem Magier möglich durch Bewegungen die Energien in die gewünschten Bahnen zu lenken, manche Magier schwören auch auf den positiven Einfluss von Zaubersprüchen, manche schaffen es wohl sogar nur durch Zaubersprüche die Magie zu lenken, wahre Meistermagier sprechen sogar davon dass eigentlich nur ein Gedanke benötigt wird und alles andere nur Beiwerk ist um den eigenen Geist zu fokussieren.

Das wirken eines Zaubers beinhaltet 2 Schritte:\begin{enumerate}
\item Das erstellen der Zaubermatrix, hierbei werden die "Bahnen" gelegt durch die später die Magische Energie hindurch geleitet wird. Diese "Bahnen" bestimmen maßgeblich die Art des Effektes den der Zauber erreichen wird/soll.

In der Magietechnik sind diese Bahnen Konstruktiv fixiert, sodass keine Veränderungen hieran vorgenommen werden können(betrifft auch in verzauberte Gegenstände, die einen bestimmten Zauber bereitstellen sollen).
\item Das "befüllen" der Zaubermatrix mit Magischer Energie, hierbei wird die Magische Energie der Umgebung durch den Magier in die Zaubermatrix geleitet. Durch die Menge der Magischen Energie wird die stärke des Effekts bestimmt, bestimmte Zaubereffekte können auch durch die Überlastung einer Matrix erreicht werden(z.b. Explosionen, ...), oder durch die unterschiedliche Verteilung der Magie auf verschiedene Kanäle der Matrix(Dauer vs Stärke eines Lichtzaubers).

Dieser Vorgang ist für den Magier mit einer gewissen Erschöpfung verbunden.

In der Magietechnik werden z.t. Kristalle eingesetzt um Magische Energie zu speichern und bei bedarf in die entsprechende Zaubermatrix einzuleiten. Dadurch kann mitunter eine bessere Kontrolle über die Menge der Magischen Macht ausgeübt werden als dies einem normalen Magier möglich ist.
\end{enumerate}

\section{Zauber Schulen}
\begin{description}

\item[Angriffsmagie:] Reine Magische Energie wird konzentriert und genutzt um Schaden an Feinden zu verursachen, in der modernen Zeit kann diese Energie aber auch von Magischen Werkzeugen benutzt werden.

\item[Illusion:] Täuschungen werden erzeugt um Leute abzulenken, zu verwirren oder gar zu ängstigen. Sie können auch genutzt werden um z.b. Licht zu erzeugen.

\item[Verstärkung:] Klassischer Supportzweig der Magie, beinhaltet Heilung, Schilde und Haushaltsmagie(Säuberung,...).

\item[Verwandlung:] Magie zur (dauerhaften) Änderung von Personen und Gegenständen, Veränderung von Körperlichen Attributen, Verwandlung in Tiere etc..

\item[Elementar:] Manipulation der Elementarmächte(Luft,Erde,Feuer,Wasser). Kann z.b. für Mächtige Angriffsmagie benutzt werden, die schwer zu blocken ist.

\item[Beschwörung:] Erlaubt Beschwörung von Geistern(Seelen z.b. auf einem Schlachtfeld), Skeletten(Überreste), Dämonen(magische Zirkel), ... in Kombination mit Elemenarmagie(Kombinationszauber) sind sogar Elementarbeschwörungen(Element als Fokus) möglich.
\end{description}

\chapter{Technischer Stand TODO}

\section{Magietechnik}

Die Magietechnik wurde ursprünglich von den Elfen erfunden. Diese haben Magie statt sie direkt zu wirken in Gegenstände eingearbeitet, als Verzauberungen, die entweder dauerhafter Natur(magische Kleidung/Rüstung/Waffen/...) sind oder eine manuelle Magiezufuhr('Zauberstäbe'/...), zur Aktivierung der Zauberformel und den zugehörigen Effekten, erfordern.

Später wurde die Magietechnik allerdings von den Gnomen perfektioniert vor allem durch die Nutzung von Magiespeichern, womit auch wenig talentierte, diese Geräte nutzen können, bzw. ein konstanter Energiestrom über längere Zeit aufrechterhalten werden kann(was wiederum viel neue Anwendungsgebiete eröffnet). Mittlerweile haben die Gnome die Elfen in jeder Hinsicht bezüglich der Herstellung von Magietechnik überholt(auch was die Zaubermatrizen angeht).

Die größten Hersteller für Magietechnik sind bei den Gnomen zu finden Doktor Grordbort's magische Werkzeuge \& Waffen und Professor Felmorn's Kristallschmieden sind dort die beiden größten.
Doktor Gorodbort ist spezialisiert auf die Integration von Zaubermatrizen in technische Geräte, während Professor Felmorn auf die Herstellung und Entwicklung der Magiekristalle anführt.
Die beiden arbeiten aktuell gemeinsam an der serienreife des 'magischen Generators', der einen Energiekristall ohne manuelle zufuhr von magischer Energie aufladen kann. Aktuell werden besondere legendäre Magiekristalle ab der Kategorie Groß benötigt um einen magischen Generator zu erschaffen.

Magiekristalle werden in verschiedene Qualitäts-(einfach bis legendär) und Größenkategorien(winzig bis gigantisch) eingeteilt.

Einfache Kristalle werden z.t. künstlich hergestellt, zur Herstellung von Kristallen der Qualitätsstufe besser und höher werden die Überreste von magischen Monstern benötigt.

\begin{tabular}{|c|c|}
\hline
Größe & Verwendung\\
\hline
\hline
winzig & kleine Gegenstände\\
\hline
klein & normale Gegenstände(Handwaffen/Pistolen, ...)\\
\hline
mittel & große Gegenstände(Gewehre, ...)\\
\hline
größer & Vollrüstungen, ...\\
\hline
Groß & mobile bzw. (semi)statische Systeme z.B. Autos/Golems/Belagerungswaffen/...\\
\hline
Gigantisch & spezial(z.b. Versorgung einer Stadt)\\
\hline
\end{tabular}

\section{Technischer Stand nach Völkern}

\vbox{
\subsection{Die Zwerge}
\begin{itemize}
\item \textbf{Bewegungsmittel:} Hauptsächlich Dampfbetriebene Züge. \hidden{Als Verbindung zwischen Zwergenstädten(U-Bahn) mittlerweile hauptsächlich elektrische Züge.}
\item \textbf{Allgemeiner Eindruck:} Steampunk: \hidden{Innerhalb der Städte gibt es meist ein oder mehrere große Industriebezirke, die die Energieerzeugung übernehmen und Abgase gebündelt aus dem Berg heraus leiten. Die Energie wird dann über verschiedene Kanäle(Transmission, Rohre und z.t. auch Kabel) in die verschiedenen Ecken der Stadt transportiert.} Zwerge sind für hochwertige Mechanische Lösungen bekannt. Es gibt Geräte mit eigenen Energieerzeugern((kleine) Dampfmaschine, Handkurbel, ...), es gibt aber auch einige bei denen auswechselbare Energiespeicher(Sprengladung, Druckluftkapsel, etc.) benutzt werden.
\item \textbf{Benutzte Technologien:}\begin{enumerate}
	\item \textbf{Mechanik:} Komplexe mechanische Systeme (Taschenuhren,...) sind üblich.
	\item \textbf{Chemie:} Diverse Sprengstoffe, Gifte, Medikamente, ... sind den Zwergen bekannt.
	\item \textbf{Druck:} Primäre Energiequelle, entweder direkt(z.b. Presslufthammer) oder indirekt über z.b. Dampfmaschinen.
	\item \textbf{Magie:} nicht vorhanden
	\item \textbf{Elektrik:} \hidden{Hohe Verbreitung in größeren Zwergenstädten(insbesondere für Licht und erste Elektrifizierte Züge). Das Stromnetz wurde bisher allerdings nur innerhalb der Berge ausgebaut, weshalb die fortschritte der Zwerge in diesem Bereich }nicht bekannt\hidden{ sind. Mobile Energiequellen(mobile Generatoren oder Batterien) sind eher noch nicht besonders weit entwickelt. Insgesamt ist die Technik in diesem Bereich vergleichbar mit dem Anfang unseres 20. Jahrhunderts.}
\end{enumerate}
\end{itemize}
}

\vbox{
\subsection{Gnome und Co.}
\begin{itemize}
\item \textbf{Bewegungsmittel:} z.t. (Dampfbetriebene) Züge aber auch Kutschen(sowohl mit Magiekraft als auch mit Pferdekraft). Im Gegensatz zu den Zwergen interessieren sich Gnome auch für den Himmel/Sterne was u.a. zur Entwicklung einfacher Luftschiffen führt(Heißluftballons bzw. erste Zeppeline).
\item \textbf{Allgemeiner Eindruck:} Gnome sind eher Bastler und Erfinder als Ingenieure, daher sind die meisten gnomischen Geräte Einzelanfertigungen. Es gibt aber auch einige wenige große gnomische Massenproduzenten.

Die meisten gnomischen Handwerker und Erfinder bedienen sich der Magietechnik.

Insgesamt sind die Gnome technisch relativ hoch entwickelt.
\item \textbf{Benutzte Technologien:}\begin{enumerate}
	\item \textbf{Mechanik:} Etwas weniger komplex als bei den Zwergen aber trotzdem viel verwendet.
	\item \textbf{Chemie:} Zur Erstellung von Medikamenten, ...
	
			Gnomischer Sprengstoff wird mitunter durch Magie verstärkt(winzige Kristallsplitter), dieser ist jedoch deutlich teurer als regulärer Sprengstoff.
	\item \textbf{Druck:} Z.t. werden Dampfmaschinen als Energiequelle verwendet, jedoch eher selten, dementsprechend wird dieser Technikzweig von den Gnomen eher weniger benutzt. Hauptsächlich für die Luftfahrt relevant(Konstruktion der Luftschiffshüllen, Befüllung mit entsprechenden Gasen, ...).
	\item \textbf{Magie:} Vorreiter, wird für so gut wie alles benutzt.
	\item \textbf{Elektrik:} nicht vorhanden.
\end{enumerate}
\end{itemize}
}

\vbox{
\subsection{Die Allianz der Mannigfaltigkeit}
\begin{itemize}
\item \textbf{Bewegungsmittel:} z.t. (Dampfbetriebene) Züge aber auch Kutschen(meist mit Pferden) es sind auch hier erste Luftschiffe(experimentell oder militärisch) vorhanden.
\item \textbf{Allgemeiner Eindruck:} Es gibt technische Einflüsse und Gerätschaften der anderen Völker.
\item \textbf{Benutzte Technologien:}\begin{enumerate}
	\item \textbf{Mechanik:} Ähnlich wie bei den Gnomen.
	\item \textbf{Chemie:} Sprengstoffentwicklung liegt etwas hinter dem zwergischen Standard, Medizin/Gifte sind etwa auf dem gleichen Stand.
	\item \textbf{Druck:} Dampfmaschinen werden vor allem in Städten eingesetzt und häufig von Zwergen gebaut/gewartet.
	\item \textbf{Magie:} Wird in größeren Städten oft eingesetzt, viele Geräte stammen mehr oder weniger von den Gnomen.
	\item \textbf{Elektrik:} im Experimentalstatus.
\end{enumerate}
\end{itemize}
}
		
\vbox{
\subsection{Die Menschen}
\begin{itemize}
\item \textbf{Allgemeiner Eindruck:} Viele Einflüsse von anderen Kulturen, aber insgesamt weniger als bei der Allianz der Mannigfaltigkeit. Technisch somit in allen Punkten etwas unterhalb dieser, dafür ist jedoch vieles von dem was die Menschen verwenden auch von Menschen entwickelt und gebaut worden(was zu einem leicht anderen 'Grundstiel' im Design führt). Insgesamt sehr ähnlich zur Technologie Ende 18. Jhr.
\end{itemize}
}
		
\vbox{
\subsection{Die Elfen}
\begin{itemize}
\item \textbf{Bewegungsmittel:} Pferde, Segelboote oder Füße
\item \textbf{Allgemeiner Eindruck:} Elfen sind meist sehr naturverbunden und können mit Technik wenig anfangen. Die meisten Elfen nutzen Magie für die verschiedensten Aufgaben. Die einzigen technischen Geräte die bei Elfen anzutreffen sind, sind einfache mechanische Konstruktionen(Bögen, eventuell mal eine Armbrust oder Taschenuhr) oder magietechnische Geräte in ihrer Urform(so wie sie ursprünglich von den Elfen entwickelt wurden), was effektiv eine Zaubermatrix in einem Gegenstand ist, die manuell mit Magie befüllt wird(z.B. 'Zauberstäbe').
\item \textbf{Benutzte Technologien:}\begin{enumerate}
	\item \textbf{Mechanik:} rudimentär vorhanden(auch als Importware(Taschenuhren, ...)).
	\item \textbf{Chemie:} In Form von Heiltränken/Salben vorhanden(oft auch in Kombination mit Magie).
	\item \textbf{Druck:} nicht vorhanden.
	\item \textbf{Magie:} In Form von Zauberstäben, etc. vorhanden, jedoch meist ohne Kristalle benutzt. Kristalltechnik zum Teil als Lampen im Einsatz(in Städten).
	\item \textbf{Elektrik:} nicht vorhanden.
\end{enumerate}
\end{itemize}
}		
\section{Technischer Stand nach Anwendungskategorien}

\subsection{Waffentechnik}

Es gibt prinzipiell alles von klassischen Vorderladewaffen über Revolversysteme bis hin zu ersten Repetierwaffen.
Auch werden viele verschiedene Möglichkeiten der Waffenkonstruktion erdacht, neben Feuerwaffen gibt es auch welche, die Luftdruck, Federkraft oder Magie benutzten um Geschosse zu beschleunigen. Es gibt jedoch auch Waffen die unkonventionelle 'Munition' verschießen, so wäre z.b. eine Waffe denkbar, die einen bestimmten Zauber aktiviert.

Neben Schusswaffen wird Technik auch für andere Waffen verwendet z.b. eine Sprengladung an einem Hammer um dessen Schlagkraft zu erhöhen, ein Gefäß gefüllt mit explosivem Material oder Fallen.

\begin{Images}
\subsubsection{Beispiele}
Zuerst die beliebten Druckluftwaffen:

\includegraphics[width=0.333\textwidth]{Bilder/SteamWeapon/examples1.jpg}\nolinebreak
\includegraphics[width=0.333\textwidth]{Bilder/SteamWeapon/examples2.jpg}\nolinebreak
\includegraphics[width=0.333\textwidth]{Bilder/SteamWeapon/engine.jpg}

Das Herzstück dieser Waffen bildet ein Mechanismus, der von den Zwergen entwickelt und vertrieben wird, die meisten Hersteller von Druckwaffen nutzen dieses Modul als Basis.


Magische Waffen werden meist in Form von Strahlenwaffen konstruiert:

\includegraphics[width=0.27\textwidth]{Bilder/MagicWeapons/gun.jpg}\nolinebreak
\includegraphics[width=0.33\textwidth]{Bilder/MagicWeapons/examples.jpg}\nolinebreak
\includegraphics[width=0.4\textwidth]{Bilder/MagicWeapons/engine.jpg}

Es gibt auch 'reguläre' Feuerwaffen, diese sind allerdings seltener als die anderen Varianten.
\end{Images}

\subsection{Rüstungstechnik}

Es gibt wie auch bei Waffentechnik verschiedene Leute, die versuchen die Technologie der aktuellen Zeit zu nutzen um bessere Schutzmaßnahmen zu konstruieren. Von Rüstungen unterschiedlicher Machart über Schilde, die sich 'ausklappen' lassen, um die Vorteile verschiedener Schildtypen zu vereinen, bis hin zu neuen Techniken im Mauerbau wird auch hier viel probiert und experimentiert.

Es gibt sogar Leute, die versuchen Rüstungen durch den Einsatz von Magie zu verbessern, zum Beispiel durch Verzauberungen.

\subsection{Sonstige Technik}
Taschenuhren, Feuerzeuge aber auch Verbandsmaterial und vieles mehr besorgt sich der moderne Abenteurer um für jede Situation gewappnet zu sein.