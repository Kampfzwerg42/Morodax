\CharacterSheetUser\CharacterSheet{}

\newgeometry{left=0mm,right=0mm,top=10mm,bottom=10mm}
\begin{landscape}
\newpage
\thispagestyle{empty} %% Remove header and footer.

\begin{center}

{\Large \underline {Listen Blatt}}
\vskip 0.3cm

\begin{minipage}[t]{0.65\textwidth}
\begin{tabulary}{\textwidth}[t]{rL}
\multicolumn{2}{c}{Rassen(die verbreitetsten):}\\
\hline
Primärattribut(+Attr): & +1 Bonus auf den Endwert(1 pro Rasse)\\
Gegenattribut(-Attr): & -1 Bonus auf den Endwert(1 pro Rasse)\\
Primärspez.(+Spez): & +1 Bonus auf den Endwert(1 pro Rasse)\\
\hline
Ork($\approx 2,0m$): & +Stärke +Ausdauer\\
Katzenmenschen($\approx 1,7m$): & +Geschick, +Akrobatik\\
Echsenmenschen($\approx 1,8m$): & +Geschick, +Ausdauer\\
Troll($\approx 2,5m$): & +Konstitution, +Einschüchtern, *\\
Mensch($\approx 1,8m$): & +Wahrnehmung, +Heimlichkeit\\
Morodianer($\approx 1,9m$): & +Charisma, +beliebige Mag.Verstand Spez.\\
Zwerg($\approx 1,2m$): & +alle tech.Attribute, -mag.Macht, +Zähigkeit\\
Elf($\approx 1,7m$): & +alle mag.Attribute, -tech.Geschick, +Verführen\\
Gnom($\approx 0,9m$): & +Magietechik(wissen)(inc. Magietechik Spez.)\\
Andere($\approx ?,?m$): &...\\
\\
\end{tabulary}
\begin{tabulary}{\textwidth}[t]{rL}
\multicolumn{2}{c}{Besondere(leere) Spezialisierungen(2):}\\
\hline
\multicolumn{2}{c}{Spezial Fähigkeiten des Charakters}\\
Neue Fähigkeiten:&Erlaubt Proben zu tätigen.\\
z.b. Magietechnik:&Magische Geräte untersuchen/bauen/...\\
Spezialisierte Fähigkeiten:&Erleichterung für Proben\\
z.b. Schwertkampf:&+1 Erfolge für Kampfproben, Konter ab Def.Erfolge=1,5x Angriffserfolge\\
'Allgemeinere' Spezialisierungen:&oft Anwendbar aber Proben sind schwerer\\
z.b. Zauberkenntnis:& Allgemeine Kennriss von Zaubern(kein Zauberweben), Schwierigkeit + 1\\
\\
\end{tabulary}

\begin{tabulary}{\textwidth}[t]{rL}
\multicolumn{2}{c}{Startwerte:}\\
\hline
Attribute:& 40 Punkte(AttrP) zum verteilen(übrige Punkte werden 1:4 in SpezP umgewandelt).\\
Startwerte:& Alle Attribute starten 'Unterdurchschnittlich'.\\
Spezialisierungen:& 100 Punkte(SpezP) zum verteilen(Pro Attribut zw. 1x und 3x der Punkte, die im Attribut vorhanden sind).\\
Startwerte:& Alle Spezialisierungen starten bei 'keine Kenntnisse'.\\
Lebenspunkte:& $Lebenskraft*3+12$\\
Ausrüstungspunkte(AP):&5 * (5 oder 1d10).\\
\\
%\end{tabulary}

%\begin{tabulary}{\textwidth}[t]{rL}
\multicolumn{2}{c}{Eigenschaften(bis zu 2):}\\
\hline 
Ungläubiger: & Mag.Verstand(\&Spezialisierungen) für immer auf 0, +1 AttrP, +Spez\\
Idiot:& Tech.Wissen(\&Spezialisierungen) für immer auf 0, +1 AttrP, +Spez\\
Homosexuell: & siehe Regeln\\
Charaktermacke: & +Spezialisierung\\
Bsp. Vegetarier: & +Lebewesen\\
Bsp. Kleptomane: & +Heimlichkeit\\
...: & ...\\
\end{tabulary} 
\end{minipage}\nolinebreak
\begin{minipage}[t]{0.6\textwidth}\hfill
\begin{tabular}[t]{|c|c|c|c|c|}
\hline 
\multicolumn{2}{|c|}{\textbf{Klassifizierung}} & \multirow{2}{*}{\textbf{Wert}} & \multicolumn{2}{c|}{\textbf{Kosten}}\\ 
Attribut & Spezialisierung & & \textbf{AttrP} & \textbf{SpezP} \\ 
\hline 
\hline 
\multirow{2}{*}{Nicht Verfügbar} & \multirow{2}{*}{keine Kenntnisse} & \multirow{2}{*}{0} & - & - \\
\cline{4-5} 
&&&\multirow{2}{*}{0}&\multirow{2}{*}{1}\\
\cline{1-3} 
\multirow{2}{*}{Schlecht(max 3)} & \multirow{2}{*}{Grund Kenntnisse} & \multirow{2}{*}{1} & & \\
\cline{4-5} 
&&&\multirow{2}{*}{1}&\multirow{2}{*}{2}\\
\cline{1-3} 
\multirow{2}{*}{Unterdurchschnittlich} & \multirow{2}{*}{Grund Kenntnisse} & \multirow{2}{*}{2} & & \\
\cline{4-5} 
&&&\multirow{2}{*}{2}&\multirow{2}{*}{3}\\
\cline{1-3} 
\multirow{2}{*}{Durchschnitt(min 5)} & \multirow{2}{*}{Erweiterte Kenntnisse} & \multirow{2}{*}{3} & & \\
\cline{4-5} 
&&&\multirow{2}{*}{3}&\multirow{2}{*}{4}\\
\cline{1-3} 
\multirow{2}{*}{Trainiert} & \multirow{2}{*}{Erweiterte Kenntnisse} & \multirow{2}{*}{4} & & \\
\cline{4-5} 
&&&\multirow{2}{*}{4}&\multirow{2}{*}{5}\\
\cline{1-3} 
\multirow{2}{*}{Experte} & \multirow{2}{*}{Ausgebildet} & \multirow{2}{*}{5} & & \\
\cline{4-5} 
&&&\multirow{2}{*}{5}&\multirow{2}{*}{6}\\
\cline{1-3} 
\multirow{2}{*}{Genie} & \multirow{2}{*}{Ausgebildet} & \multirow{2}{*}{6} & & \\
\cline{4-5} 
&&&\multirow{2}{*}{6}&\multirow{2}{*}{7}\\
\cline{1-3} 
\multirow{2}{*}{...} & \multirow{2}{*}{Profi} & \multirow{2}{*}{7} & & \\
\cline{4-5} 
&&&\multirow{2}{*}{7}&\multirow{2}{*}{8}\\
\cline{1-3} 
\multirow{2}{*}{...} & \multirow{2}{*}{...} & \multirow{2}{*}{...} & & \\
\cline{4-5} 
&&&...&...\\
\hline 
\end{tabular}
\begin{tabulary}{\textwidth}[t]{rL}
\multicolumn{2}{c}{Spezialtalent(1):}\\
\hline 
$\bullet$ &Beidhändikeit (kann alles mit jeder Hand gleich gut \ref{SupportKampfaktionen})\\
$\bullet$ &Reaktiv (\ref{SupportKampfaktionen})\\
$\bullet$ &Zäher Mistkerl (nur bei Erschaffung, \ref{vital})\\
$\bullet$ &Talentiert: 1EP weniger bei Steigerungen(+20 Spez.Punkte bei Erschaffung) \\
$\bullet$ &Unerkannt (\ref{Unerkannt}) \\
%$\bullet$ &?Meisterschütze (Schnelleres Nachladen/?)\\
$\bullet$ &Mac Gyver(\ref{lange Proben})\\
$\bullet$ &Meistermagier(freischalten eines Basiszaubers ohne Buch(auch nicht Standardzauber wie z.b. Beschwörung), + 1 in Verstand- und Machtspezialisierung) \\
$\bullet$ &Kampfmagier(\ref{Kampfmagier}) \\
$\bullet$ &Nerd: nur bei Erschaffung, Wissensattribute auf 0, Wissensspezialisierungen beliebig wählbar, -3 AttrP, +15 SpezP(Wissen), 1EP weniger pro Spez.Punkt(Wissen) \\
$\bullet$ & ...(seid kreativ) \\
\end{tabulary} 
\end{minipage}

\end{center}
\end{landscape}
\restoregeometry

\newgeometry{left=5mm,right=5mm,top=10mm,bottom=10mm}
\begin{landscape}
\newpage
\thispagestyle{empty} %% Remove header and footer.

\begin{center}
{\Large \underline {Kurzreferenz}}
\vskip 0.3cm
\begin{minipage}[t]{0.45\paperheight}
\begin{description}
\item[Allgemeine Regeln:]\ 
	\begin{description}
	\item[Würfel:] Attributspunkte viele Würfel werden um Spezialisierungspunkte verbessert ($\xrightarrow{0}w20\xrightarrow{1}w12\xrightarrow{1}w8\xrightarrow{1}w4$ oder $-\xrightarrow{3}w12$).
	\item[Erfolge/Misserfolge:]Jeder Würfel der $\leq$ Probenschwierigkeit ist ist ein Erfolg.
	\item[Gegenprobe:]Beide machen eine Probe die Erfolge werden miteinander verglichen(die schlechtesten desjenigen der mehr erfolge hat werden als kritische erfolge aussortiert). Der beste Erfolg des einen gegen den schlechtesten Erfolg des anderen und so weiter. Der bessere Erfolg generiert jeweils 1 Teilerfolg, bei gleichstand gibts nix.
	\item[Passivität:]Wenn der Gegner nicht würfeln kann/will gelten alle Erfolge der Probe als kritische Erfolge.
	\item[Erschöpfung:] TODO.
	\end{description}
\item[Magie Regeln:]\ 
	\begin{description}
	\item[Voraussetzung:] Spezialisierungspunkte in der magischen Macht der Zauberkategorie.
	\item[Proben:] 1 Zauberweben- und 1 Zaubermachtprobe(2 Kampfrunden). Die Gegenprobe darf um je $1w20$ oder $1w12$ erschwert werden je 1 zusätzlichen Erfolg für Verbesserungen zu generieren(bzw. ein Kritischen Erfolg bei $1w12$).
	\item[Zauberweben:](Magischer Verstand)
		
	\textbf{Standardgegenprobe} 1w20 gegen Schwierigkeit 20.

	\textbf{Ungelernte Zauber} erfordern für alles die doppelte Anzahl an Zauberweben Erfolgen und werden gegen 2w20 gewürfelt.
	
	\item[Andauernde(temporäre) Zauber] benötigen in jeder Runde eine zusätzliche Zauberprobe(in der Zauberschwierigkeit). Falls weitere Zauber gewirkt werden sollen gelten die Regeln für mehrere Zauber.
	
	\item[Zaubermacht:] \textbf{Standardgegenprobe} gegen 0w20. Scheitert die Probe passiert nix.

	\item[Mehrere Zauber/Kombinationszauber:] als Spezialisierung wird die schwächere Spezialisierung benutzt Erfolge müssen auf beide Zauber verteilt werden. Pro Extrazauber wird die Gegenprobe um 1w12 erschwert.

	\item[Lebensdauer von Zaubern:]\
	 \begin{itemize}
		\item \textbf{Spontan:} direkter(dauerhafter) Effekt
		\item \textbf{Temporär:} andauernder Effekt
		\item \textbf{Permanent(Runden):} während der Aktivierungszeit wie Temporär, danach (fast) wie Spontan
	\end{itemize}
	\end{description}
\end{description}
\end{minipage}\hfill \begin{minipage}[t]{0.45\paperheight}
\begin{description}
\item[Kampf Regeln:]\ 
	\begin{description}
	\item[Kampfproben](Schwierigkeit 12) gegen gegnerische Abwehr(Nahkampf: 12/Fernkampf: 8) oder Ausweichen(Nahkampf: 8/Fernkampf: 12)
	\item[Nahkampfdistanz ändern] Akrobatikprobe(der der näher ran will) vs. Ausweichenprobe(der der weiter weg will)
	\item[zu nah für die eigene Waffe] gibt einen Spezialisierungsmalus auf Angriff(2*Falsche Distanz) und Verteidigung(Falsche Distanz)
	\item[zu weit weg für die eigene Waffe] gibt einen Spezialisierungsmalus auf Angriff(2*Falsche Distanz)
	\item[Schild] kann alles mit Abwehr oder Ausweichen(immer gegen 12) verteidigen. Überschreibt außerdem für Verteidigungsproben die eigene Waffenreichweite.
	\item[Trefferschaden] Leichte Treffer(kein kritischer Erfolg)= halber Schaden. Kritische Treffer(min 2 kritische Erfolge)= (kritische Erfolge - 1) * Schaden. 
	\item[Rüstung] Endschaden/Rüstungswert - Zähigkeit wird von den LP abgezogen. Wird die Rüstung umgangen zählt Rüstungswert=1.
	\end{description}	
\end{description}
\end{minipage}

\end{center}
\end{landscape}
\restoregeometry
